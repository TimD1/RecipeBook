\documentclass[11pt, twoside, openany]{book}
\usepackage[margin=0.5in]{geometry}
\usepackage{graphicx}
\usepackage{multicol}
\setlength{\parindent}{0pt}
\title{ \LARGE \textbf{Dunn Family Recipe Book}}
\begin{document}
\maketitle
\section*{Table of Contents}
{~\vspace{2mm}\\ \Large \textbf{Appetizers}}\hfill\textbf{\pageref{appetizers}}

Black Bean and Corn Salsa\hrulefill\pageref{black-bean-and-corn-salsa}\\
Chicken Wing Dip\hrulefill\pageref{chicken-wing-dip}\\
Crab Appetizers\hrulefill\pageref{crab-appetizers}\\
Jalapeno Poppers\hrulefill\pageref{jalapeno-poppers}\\
Pesto\hrulefill\pageref{pesto}\\
Pizza Bites\hrulefill\pageref{pizza-bites}\\
Power Balls\hrulefill\pageref{power-balls}\\
Pumpkin Buttermilk Biscuits\hrulefill\pageref{pumpkin-buttermilk-biscuits}\\
Pumpkin Dip\hrulefill\pageref{pumpkin-dip}\\
Pumpkin Muffins\hrulefill\pageref{pumpkin-muffins}\\
Spinach Balls\hrulefill\pageref{spinach-balls}\\
Stuffed Mushrooms\hrulefill\pageref{stuffed-mushrooms}\\
Swedish Meatballs\hrulefill\pageref{swedish-meatballs}\\
{~\vspace{2mm}\\ \Large \textbf{Breads}}\hfill\textbf{\pageref{breads}}

Almond Braid Bread\hrulefill\pageref{almond-braid-bread}\\
Almond Braid Filling\hrulefill\pageref{almond-braid-filling}\\
Almond Braid Icing\hrulefill\pageref{almond-braid-icing}\\
Amish Cinnamon Bread\hrulefill\pageref{amish-cinnamon-bread}\\
Apple Wheat Bread\hrulefill\pageref{apple-wheat-bread}\\
Banana Bread\hrulefill\pageref{banana-bread}\\
Cheese Casserole Bread\hrulefill\pageref{cheese-casserole-bread}\\
Cornbread\hrulefill\pageref{cornbread}\\
Pumpkin Chocolate Chip Bread\hrulefill\pageref{pumpkin-chocolate-chip-bread}\\
Zucchini Bread\hrulefill\pageref{zucchini-bread}\\
{~\vspace{2mm}\\ \Large \textbf{Breakfasts}}\hfill\textbf{\pageref{breakfasts}}

Ham and Broccoli Casserole\hrulefill\pageref{ham-and-broccoli-casserole}\\
Pancakes\hrulefill\pageref{pancakes}\\
Pumpkin Waffles\hrulefill\pageref{pumpkin-waffles}\\
Red Velvet French Toast\hrulefill\pageref{red-velvet-french-toast}\\
Waffles\hrulefill\pageref{waffles}\\
Egg Avocado Toast\hrulefill\pageref{egg-avocado-toast}\\
{~\vspace{2mm}\\ \Large \textbf{Cakes}}\hfill\textbf{\pageref{cakes}}

Almond Pound Cake\hrulefill\pageref{almond-pound-cake}\\
Amaretto Cheesecake\hrulefill\pageref{amaretto-cheesecake}\\
Angel Food Cake Base\hrulefill\pageref{angel-food-cake-base}\\
Angel Food Cake Strawberry Sauce\hrulefill\pageref{angel-food-cake-strawberry-sauce}\\
Apple Cake\hrulefill\pageref{apple-cake}\\
Carrot Cake Base\hrulefill\pageref{carrot-cake-base}\\
Carrot Cake Frosting\hrulefill\pageref{carrot-cake-frosting}\\
Chocolate Chip Cheesecake\hrulefill\pageref{chocolate-chip-cheesecake}\\
Chocolate Lava Cakes\hrulefill\pageref{chocolate-lava-cakes}\\
Devil's Food Cake\hrulefill\pageref{devil's-food-cake}\\
French Almond Cake\hrulefill\pageref{french-almond-cake}\\
Macademia Nut Cheesecake\hrulefill\pageref{macademia-nut-cheesecake}\\
Mini Cheesecakes\hrulefill\pageref{mini-cheesecakes}\\
Sour Cream Coffee Cake\hrulefill\pageref{sour-cream-coffee-cake}\\
Triple Chocolate Cheesecake\hrulefill\pageref{triple-chocolate-cheesecake}\\
{~\vspace{2mm}\\ \Large \textbf{Casseroles}}\hfill\textbf{\pageref{casseroles}}

Bacon Egg Spinach Casserole\hrulefill\pageref{bacon-egg-spinach-casserole}\\
Broccoli Chicken Casserole\hrulefill\pageref{broccoli-chicken-casserole}\\
Cauliflower Casserole\hrulefill\pageref{cauliflower-casserole}\\
Chile, Chicken, and Cheese Casserole\hrulefill\pageref{chile,-chicken,-and-cheese-casserole}\\
Spinach Artichoke Chicken Casserole\hrulefill\pageref{spinach-artichoke-chicken-casserole}\\
{~\vspace{2mm}\\ \Large \textbf{Chicken}}\hfill\textbf{\pageref{chicken}}

Broccoli Chicken Casserole\hrulefill\pageref{broccoli-chicken-casserole}\\
Buffalo Chicken Mac and Cheese\hrulefill\pageref{buffalo-chicken-mac-and-cheese}\\
Chicken Cordon Bleu, Baked\hrulefill\pageref{chicken-cordon-bleu,-baked}\\
Chicken Cordon Bleu\hrulefill\pageref{chicken-cordon-bleu}\\
Chicken Devan\hrulefill\pageref{chicken-devan}\\
Chicken Divan\hrulefill\pageref{chicken-divan}\\
Chicken Enchiladas\hrulefill\pageref{chicken-enchiladas}\\
Chicken French\hrulefill\pageref{chicken-french}\\
Chicken Pot Pie\hrulefill\pageref{chicken-pot-pie}\\
Chicken Taco Soup\hrulefill\pageref{chicken-taco-soup}\\
Chicken and Artichoke Bake\hrulefill\pageref{chicken-and-artichoke-bake}\\
Chicken and Shrimp Carbonara\hrulefill\pageref{chicken-and-shrimp-carbonara}\\
Chile, Chicken, and Cheese Casserole\hrulefill\pageref{chile,-chicken,-and-cheese-casserole}\\
Cream Cheese Spinach Stuffed Chicken\hrulefill\pageref{cream-cheese-spinach-stuffed-chicken}\\
Creamy Herb Chicken\hrulefill\pageref{creamy-herb-chicken}\\
Creamy Pesto Chicken\hrulefill\pageref{creamy-pesto-chicken}\\
Crockpot Shredded Chicken\hrulefill\pageref{crockpot-shredded-chicken}\\
Dill Chicken Salad\hrulefill\pageref{dill-chicken-salad}\\
Lemon Chicken With White Wine\hrulefill\pageref{lemon-chicken-with-white-wine}\\
Onion Cream Chicken\hrulefill\pageref{onion-cream-chicken}\\
Oriental Chicken Salad\hrulefill\pageref{oriental-chicken-salad}\\
Seared Chicken With Avocado\hrulefill\pageref{seared-chicken-with-avocado}\\
Sour Cream Salsa Chicken\hrulefill\pageref{sour-cream-salsa-chicken}\\
Spinach Artichoke Chicken Casserole\hrulefill\pageref{spinach-artichoke-chicken-casserole}\\
Coconut Chicken Curry\hrulefill\pageref{coconut-chicken-curry}\\
{~\vspace{2mm}\\ \Large \textbf{Cookies}}\hfill\textbf{\pageref{cookies}}

Almond Macaroons\hrulefill\pageref{almond-macaroons}\\
Bunco Bites\hrulefill\pageref{bunco-bites}\\
Chocolate Chip Cookies\hrulefill\pageref{chocolate-chip-cookies}\\
Christmas Crackle\hrulefill\pageref{christmas-crackle}\\
Cookies and Cream Oreo Bark\hrulefill\pageref{cookies-and-cream-oreo-bark}\\
Mint Meringues\hrulefill\pageref{mint-meringues}\\
Monster Cookies\hrulefill\pageref{monster-cookies}\\
Oatmeal Peanut Butter Chocolate Chip Cookies\hrulefill\pageref{oatmeal-peanut-butter-chocolate-chip-cookies}\\
Peanut Butter Balls\hrulefill\pageref{peanut-butter-balls}\\
Peanut Butter Blossoms\hrulefill\pageref{peanut-butter-blossoms}\\
Peanut Butter Cookies\hrulefill\pageref{peanut-butter-cookies}\\
Peppermint Bark\hrulefill\pageref{peppermint-bark}\\
Russian Tea Cakes\hrulefill\pageref{russian-tea-cakes}\\
Sugar Cookies\hrulefill\pageref{sugar-cookies}\\
{~\vspace{2mm}\\ \Large \textbf{Desserts}}\hfill\textbf{\pageref{desserts}}

Blueberry Cobbler\hrulefill\pageref{blueberry-cobbler}\\
Blueberry Kuchen\hrulefill\pageref{blueberry-kuchen}\\
Blueberry Maple Muffins\hrulefill\pageref{blueberry-maple-muffins}\\
Chocolate Almond Mousse\hrulefill\pageref{chocolate-almond-mousse}\\
Chocolate Brownies\hrulefill\pageref{chocolate-brownies}\\
Chocolate Chip Muffins\hrulefill\pageref{chocolate-chip-muffins}\\
Dark Chocolate Brownies\hrulefill\pageref{dark-chocolate-brownies}\\
Flourless Chocolate Torte\hrulefill\pageref{flourless-chocolate-torte}\\
Fudge\hrulefill\pageref{fudge}\\
Island Gem Bars\hrulefill\pageref{island-gem-bars}\\
Lemon Bars\hrulefill\pageref{lemon-bars}\\
Lemon Poppy Seed Muffins\hrulefill\pageref{lemon-poppy-seed-muffins}\\
No Bake Brownies\hrulefill\pageref{no-bake-brownies}\\
Oreo Truffles\hrulefill\pageref{oreo-truffles}\\
Pecan Tassies\hrulefill\pageref{pecan-tassies}\\
Raspberry Crisp\hrulefill\pageref{raspberry-crisp}\\
Rock Candy\hrulefill\pageref{rock-candy}\\
Sticky Buns\hrulefill\pageref{sticky-buns}\\
{~\vspace{2mm}\\ \Large \textbf{Pastas}}\hfill\textbf{\pageref{pastas}}

Buffalo Chicken Mac and Cheese\hrulefill\pageref{buffalo-chicken-mac-and-cheese}\\
Chicken and Shrimp Carbonara\hrulefill\pageref{chicken-and-shrimp-carbonara}\\
Lasagna\hrulefill\pageref{lasagna}\\
Mac and Cheese\hrulefill\pageref{mac-and-cheese}\\
Meatballs\hrulefill\pageref{meatballs}\\
Spaghetti Squash Gratin\hrulefill\pageref{spaghetti-squash-gratin}\\
{~\vspace{2mm}\\ \Large \textbf{Pies}}\hfill\textbf{\pageref{pies}}

Apple Crumb Pie\hrulefill\pageref{apple-crumb-pie}\\
Biscuit Pie Crust\hrulefill\pageref{biscuit-pie-crust}\\
Chicken Pot Pie\hrulefill\pageref{chicken-pot-pie}\\
Grasshopper Pie\hrulefill\pageref{grasshopper-pie}\\
{~\vspace{2mm}\\ \Large \textbf{Pizzas}}\hfill\textbf{\pageref{pizzas}}

Stromboli\hrulefill\pageref{stromboli}\\
Whole Wheat Pizza Dough\hrulefill\pageref{whole-wheat-pizza-dough}\\
{~\vspace{2mm}\\ \Large \textbf{Pork}}\hfill\textbf{\pageref{pork}}

Barbecue Sauce\hrulefill\pageref{barbecue-sauce}\\
Rib Sauce\hrulefill\pageref{rib-sauce}\\
Sausage Balls\hrulefill\pageref{sausage-balls}\\
Southern Succor Rub\hrulefill\pageref{southern-succor-rub}\\
{~\vspace{2mm}\\ \Large \textbf{Salads}}\hfill\textbf{\pageref{salads}}

Apple, Dried Cherry, and Walnut Salad Base\hrulefill\pageref{apple,-dried-cherry,-and-walnut-salad-base}\\
Apple, Dried Cherry, and Walnut Salad Maple Dressing\hrulefill\pageref{apple,-dried-cherry,-and-walnut-salad-maple-dressing}\\
Candied Walnut and Goat Cheese Salad\hrulefill\pageref{candied-walnut-and-goat-cheese-salad}\\
Candied Walnuts\hrulefill\pageref{candied-walnuts}\\
Dill Chicken Salad\hrulefill\pageref{dill-chicken-salad}\\
French Potato Salad\hrulefill\pageref{french-potato-salad}\\
Oriental Chicken Salad\hrulefill\pageref{oriental-chicken-salad}\\
Oriental Coleslaw Salad Base\hrulefill\pageref{oriental-coleslaw-salad-base}\\
Oriental Coleslaw Salad Dressing\hrulefill\pageref{oriental-coleslaw-salad-dressing}\\
Strawberry Salad with Poppy Seed Dressing\hrulefill\pageref{strawberry-salad-with-poppy-seed-dressing}\\
{~\vspace{2mm}\\ \Large \textbf{Seafood}}\hfill\textbf{\pageref{seafood}}

Bourbon Bacon Scallops\hrulefill\pageref{bourbon-bacon-scallops}\\
Chicken and Shrimp Carbonara\hrulefill\pageref{chicken-and-shrimp-carbonara}\\
Coconut Shrimp\hrulefill\pageref{coconut-shrimp}\\
Crab Casserole\hrulefill\pageref{crab-casserole}\\
Honey Balsamic Salmon\hrulefill\pageref{honey-balsamic-salmon}\\
New England Clam Chowder\hrulefill\pageref{new-england-clam-chowder}\\
Pan-Seared Scallops\hrulefill\pageref{pan-seared-scallops}\\
Shrimp Bisque\hrulefill\pageref{shrimp-bisque}\\
{~\vspace{2mm}\\ \Large \textbf{Sides}}\hfill\textbf{\pageref{sides}}

Baked Asparagus\hrulefill\pageref{baked-asparagus}\\
Blueberry Muffins\hrulefill\pageref{blueberry-muffins}\\
Mashed Sweet Potatoes\hrulefill\pageref{mashed-sweet-potatoes}\\
Twice Baked Potatoes\hrulefill\pageref{twice-baked-potatoes}\\
{~\vspace{2mm}\\ \Large \textbf{Soups}}\hfill\textbf{\pageref{soups}}

Butternut Squash Bisque\hrulefill\pageref{butternut-squash-bisque}\\
Chicken Taco Soup\hrulefill\pageref{chicken-taco-soup}\\
Chicken Taco Stew\hrulefill\pageref{chicken-taco-stew}\\
Chili\hrulefill\pageref{chili}\\
Corn Chowder\hrulefill\pageref{corn-chowder}\\
Country Vegetable Soup\hrulefill\pageref{country-vegetable-soup}\\
Egg Drop Soup\hrulefill\pageref{egg-drop-soup}\\
New England Clam Chowder\hrulefill\pageref{new-england-clam-chowder}\\
Potato Leek Soup\hrulefill\pageref{potato-leek-soup}\\
Roasted Cauliflower Soup\hrulefill\pageref{roasted-cauliflower-soup}\\
Shrimp Bisque\hrulefill\pageref{shrimp-bisque}\\
Spicy White Turkey Chili Soup\hrulefill\pageref{spicy-white-turkey-chili-soup}\\
Sweet Potato Soup\hrulefill\pageref{sweet-potato-soup}\\
Shakshuka\hrulefill\pageref{shakshuka}\\
{~\vspace{2mm}\\ \Large \textbf{Steak}}\hfill\textbf{\pageref{steak}}

Beef Tenderloin\hrulefill\pageref{beef-tenderloin}\\
Beef Wellington\hrulefill\pageref{beef-wellington}\\
Steak Teriyaki Marinade\hrulefill\pageref{steak-teriyaki-marinade}\\
{~\vspace{2mm}\\ \Large \textbf{Quiches}}\hfill\textbf{\pageref{quiches}}

Quiche Lorraine\hrulefill\pageref{quiche-lorraine}\\
Spinach Quiche\hrulefill\pageref{spinach-quiche}\\
Vegetable Gruyere Quiche\hrulefill\pageref{vegetable-gruyere-quiche}\\
{\newpage \LARGE \textbf{Appetizers}} \label{appetizers}\vspace{4mm}\\
\noindent\begin{minipage}[t]{\linewidth}%
{\Large\textbf{Black Bean and Corn Salsa}} \label{black-bean-and-corn-salsa}\hfill\textit{Sue Dunn}\\
\textbf{Yield:} \textit{10 servings}\\
\noindent\begin{minipage}[t]{0.78\linewidth}%
\textbf{Ingredients}:\vspace{-3mm}
\begin{multicols}{2}
\begin{itemize}\setlength\itemsep{-1mm}
\item 1 (15 oz) can black beans, drained
\item 1 (11 oz) can corn, drained
\item 1 jalapeno pepper, minced (optional)
\item 1 avacado, chopped
\item 2 medium tomatoes, chopped
\item 1 red bell pepper, chopped
\item 1/2 bunch fresh cilantro, chopped
\item 1/2 onion, diced
\item 2 limes, squeezed
\item 1 tsp salt
\item tortilla chips, for dipping
\end{itemize}
\end{multicols}
\end{minipage}
\noindent\begin{minipage}[t]{0.18\linewidth}
\centering \strut\vspace*{-\baselineskip}\newline
\includegraphics[width=0.9\linewidth]{/home/tim/Documents/projects/recipes/img/salsa.jpg}\\
\end{minipage}\vspace{3mm}
\textbf{Directions}:
\vspace{-3mm}\begin{enumerate}\setlength\itemsep{-1mm}
\item Combine all ingredients except avocado and chips.
\item Cover and chill. Add avacodo just before serving with chips.
\end{enumerate}
\end{minipage}\vspace{8mm}
\noindent\begin{minipage}[t]{\linewidth}%
{\Large\textbf{Chicken Wing Dip}} \label{chicken-wing-dip}\hfill\textit{Tom Dunn}\\
\noindent\begin{minipage}[t]{0.78\linewidth}%
\textbf{Ingredients}:\vspace{-3mm}
\begin{multicols}{2}
\begin{itemize}\setlength\itemsep{-1mm}
\item 4-6 oz Frank's hot sauce
\item 2 (8 oz) packages regular cream cheese
\item 1 (8 oz) bottle bleu cheese dressing
\item 1 chicken breast, cooked and shredded
\item 8 oz Monterey Jack, shredded
\end{itemize}
\end{multicols}
\end{minipage}
\noindent\begin{minipage}[t]{0.18\linewidth}
\centering \strut\vspace*{-\baselineskip}\newline
\includegraphics[width=0.9\linewidth]{/home/tim/Documents/projects/recipes/img/31362A38-BD46-4715-9B26-0AC1C3991F65.jpg}\\
\end{minipage}\vspace{3mm}
\textbf{Directions}:
\vspace{-3mm}\begin{enumerate}\setlength\itemsep{-1mm}
\item Cook and shred chicken.
\item In a medium-size saucepan, bring hot sauce to a simmer over medium-low heat. Add cream cheese and stir continuously until fully melted. Stir in dressing.
\item Add chicken and stir until fully coated. Add shredded cheese and stir until fully melted.
\item Transfer mixture to a small crockpot and heat on low. Or transfer to a small casserole dish and bake at 350F until bubbly (20-25 minutes). 
\item Serve warm with tortilla chips.
\end{enumerate}
\end{minipage}\vspace{8mm}
\noindent\begin{minipage}[t]{\linewidth}%
{\Large\textbf{Crab Appetizers}} \label{crab-appetizers}\hfill\textit{Barb McKinley}\\
\textbf{Yield:} \textit{makes 24}\\
\noindent\begin{minipage}[t]{0.78\linewidth}%
\textbf{Ingredients}:\vspace{-3mm}
\begin{multicols}{2}
\begin{itemize}\setlength\itemsep{-1mm}
\item 1 (7 oz) can of crab meat
\item 1 stick of butter, softened
\item 1 jar (5 oz) Kraft Old English Cheese
\item 1/2 tsp garlic powder
\item 2 tsp mayonnaise
\item 6 English muffins
\end{itemize}
\end{multicols}
\end{minipage}
\noindent\begin{minipage}[t]{0.18\linewidth}
\centering \strut\vspace*{-\baselineskip}\newline
\includegraphics[width=0.9\linewidth]{/home/tim/Documents/projects/recipes/img/crab-appetizers.jpeg}\\
\end{minipage}\vspace{3mm}
\textbf{Directions}:
\vspace{-3mm}\begin{enumerate}\setlength\itemsep{-1mm}
\item Open English muffins, then cut in half again, making 24 quarter pieces. Wash the crab meat, looking for loose pieces of shell.
\item Mix all ingredients together and place atop English muffins.
\item Bake for 20 minutes at 400F.
\end{enumerate}
\end{minipage}\vspace{8mm}
\noindent\begin{minipage}[t]{\linewidth}%
{\Large\textbf{Jalapeno Poppers}} \label{jalapeno-poppers}\hfill\textit{Trey Dunn}\\
\textbf{Yield:} \textit{makes 24 peppers}\\
\noindent\begin{minipage}[t]{0.78\linewidth}%
\textbf{Ingredients}:\vspace{-3mm}
\begin{multicols}{2}
\begin{itemize}\setlength\itemsep{-1mm}
\item 12 jalapeno peppers, halved and seeded
\item 2 packages of cream cheese
\item 1 1/2 lb bacon
\item 2 handfuls sharp cheddar cheese
\item garlic powder
\end{itemize}
\end{multicols}
\end{minipage}
\noindent\begin{minipage}[t]{0.18\linewidth}
\centering \strut\vspace*{-\baselineskip}\newline
\includegraphics[width=0.9\linewidth]{/home/tim/Documents/projects/recipes/img/jalapeno-poppers.jpeg}\\
\end{minipage}\vspace{3mm}
\textbf{Directions}:
\vspace{-3mm}\begin{enumerate}\setlength\itemsep{-1mm}
\item Heat oven to 400F. Prepare peppers.
\item Heat cream cheese in microwave until soft (30 seconds). Add cheddar cheese and garlic powder. Combine, and add the mixture into the peppers, filling them to the top.
\item Cut the bacon strips in half and wrap around the pepper, securing with a toothpick. Cook for about 30 minutes or until done.
\end{enumerate}
\end{minipage}\vspace{8mm}
\noindent\begin{minipage}[t]{\linewidth}%
{\Large\textbf{Pesto}} \label{pesto}\hfill\textit{Sue Dunn}\\
\noindent\begin{minipage}[t]{0.78\linewidth}%
\textbf{Ingredients}:\vspace{-3mm}
\begin{multicols}{2}
\begin{itemize}\setlength\itemsep{-1mm}
\item 1/3 cup pine nuts (or walnuts)
\item 3 garlic cloves, minced
\item 2 cups fresh basil
\item 1/2 tsp salt
\item 1/4 tsp ground black pepper
\item 2/3 cup extra virgin olive oil
\item 1/2 cup grated Parmesan
\end{itemize}
\end{multicols}
\end{minipage}
\noindent\begin{minipage}[t]{0.18\linewidth}
\centering \strut\vspace*{-\baselineskip}\newline
\includegraphics[width=0.9\linewidth]{/home/tim/Documents/projects/recipes/img/pesto.jpeg}\\
\end{minipage}\vspace{3mm}
\textbf{Directions}:
\vspace{-3mm}\begin{enumerate}\setlength\itemsep{-1mm}
\item Place the pine nuts and garlic in a food processor. Process until coarsely chopped, about 10 seconds. Add the basil leaves, salt, and pepper and process until mixture resembles a paste, about 1 minute. With the processor running, slowly pour the olive oil through the feed tube and process until the pesto is thoroughly blended. Add the Parmesan and process a minute more.
\item Use pesto immediately or store in an air-tight container, covered with a thin layer of olive oil (this seals out the air and prevents the pesto from oxidizing, which would turn it an ugly brown color). It will keep in the refrigerator for about a week.
\end{enumerate}
\end{minipage}\vspace{8mm}
\noindent\begin{minipage}[t]{\linewidth}%
{\Large\textbf{Pizza Bites}} \label{pizza-bites}\hfill\textit{Sue Dunn}\\
\noindent\begin{minipage}[t]{0.78\linewidth}%
\textbf{Ingredients}:\vspace{-3mm}
\begin{multicols}{2}
\begin{itemize}\setlength\itemsep{-1mm}
\item unknown
\end{itemize}
\end{multicols}
\end{minipage}
\noindent\begin{minipage}[t]{0.18\linewidth}
\centering \strut\vspace*{-\baselineskip}\newline
\includegraphics[width=0.9\linewidth]{/home/tim/Documents/projects/recipes/img/none.jpg}\\
\end{minipage}\vspace{3mm}
\textbf{Directions}:
\vspace{-3mm}\begin{enumerate}\setlength\itemsep{-1mm}
\item unknown
\end{enumerate}
\end{minipage}\vspace{8mm}
\noindent\begin{minipage}[t]{\linewidth}%
{\Large\textbf{Power Balls}} \label{power-balls}\hfill\textit{Sue Dunn}\\
\noindent\begin{minipage}[t]{0.78\linewidth}%
\textbf{Ingredients}:\vspace{-3mm}
\begin{multicols}{2}
\begin{itemize}\setlength\itemsep{-1mm}
\item 4 cups oatmeal
\item 1 cup peanut butter
\item 2/3 cup honey
\item 1 cup chocolate chips
\item 2 tsp vanilla
\item 1/4 cup ground flax seed
\end{itemize}
\end{multicols}
\end{minipage}
\noindent\begin{minipage}[t]{0.18\linewidth}
\centering \strut\vspace*{-\baselineskip}\newline
\includegraphics[width=0.9\linewidth]{/home/tim/Documents/projects/recipes/img/881F7E37-ABBD-497D-A638-12EFC9985D4A.jpg}\\
\end{minipage}\vspace{3mm}
\textbf{Directions}:
\vspace{-3mm}\begin{enumerate}\setlength\itemsep{-1mm}
\item In a large mixing bowl, add together the dry ingredients (oatmeal, flax seed, chocolate chips). Mix together.
\item Add in peanut butter, honey, and vanilla. Mix together thoroughly. Refrigerate to make your dough a bit firmer.
\item Roll bits of dough into 1 inch balls. Place in an air tight container and store in the refrigerator. They can be stored for up to one week.
\end{enumerate}
\end{minipage}\vspace{8mm}
\noindent\begin{minipage}[t]{\linewidth}%
{\Large\textbf{Pumpkin Buttermilk Biscuits}} \label{pumpkin-buttermilk-biscuits}\hfill\textit{Sue Dunn}\\
\textbf{Yield:} \textit{15 biscuits}\\
\noindent\begin{minipage}[t]{0.78\linewidth}%
\textbf{Ingredients}:\vspace{-3mm}
\begin{multicols}{2}
\begin{itemize}\setlength\itemsep{-1mm}
\item 2 cups all-purpose flour
\item 2 1/2 tsp baking powder
\item 1/2 tsp salt
\item 5 Tbsp butter, chilled
\item 1/3 cup buttermilk
\item 3/4 cup pumpkin
\item 3 Tbsp honey
\end{itemize}
\end{multicols}
\end{minipage}
\noindent\begin{minipage}[t]{0.18\linewidth}
\centering \strut\vspace*{-\baselineskip}\newline
\includegraphics[width=0.9\linewidth]{/home/tim/Documents/projects/recipes/img/none.jpg}\\
\end{minipage}\vspace{3mm}
\textbf{Directions}:
\vspace{-3mm}\begin{enumerate}\setlength\itemsep{-1mm}
\item Preheat oven to 400F. Combine flour, baking powder, and salt in large bowl; cut in butter with a pastry blender until it resembles a coarse meal. Chill ofr 10 minutes.
\item Combine buttermilk and honey, stirring with a whisk until blended. Add buttermilk to flour mixture; stir just until moist.
\item Turn dough out on a lightly floured surface; knead lightly 4 times. Roll dough into a 9 x 5 x 1/2 inch rectangle; dust top of dough with flour. Fold dough crosswise into thirds. Re-roll dough into same rectangle. Dust again. Repeat that again, folding the other direction.
\item Cut 3 x 5 to form 15 rounds. Place 1 inch apart on a large baking sheet lined with parchment paper. Bake at 400F for 12 minutes or until golden. Remove from pan; cool 2 minutes and serve warm.
\end{enumerate}
\end{minipage}\vspace{8mm}
\noindent\begin{minipage}[t]{\linewidth}%
{\Large\textbf{Pumpkin Dip}} \label{pumpkin-dip}\hfill\textit{Donna Knights}\\
\textit{``Half of this recipe is usually enough''}\\
\noindent\begin{minipage}[t]{0.78\linewidth}%
\textbf{Ingredients}:\vspace{-3mm}
\begin{multicols}{2}
\begin{itemize}\setlength\itemsep{-1mm}
\item 2 cups confectioners sugar
\item 8 oz cream cheese, softened
\item 15 oz canned pumpkin
\item 1 tsp cinnamon
\item 1/2 tsp ginger
\item 1 dash nutmeg
\end{itemize}
\end{multicols}
\end{minipage}
\noindent\begin{minipage}[t]{0.18\linewidth}
\centering \strut\vspace*{-\baselineskip}\newline
\includegraphics[width=0.9\linewidth]{/home/tim/Documents/projects/recipes/img/B96CF91B-E4BA-4702-A0BA-1BDE96A6206A.jpg}\\
\end{minipage}\vspace{3mm}
\textbf{Directions}:
\vspace{-3mm}\begin{enumerate}\setlength\itemsep{-1mm}
\item Beat all ingredients until smooth, and refrigerate.
\item Serve with ginger snaps/apples, etc.
\end{enumerate}
\end{minipage}\vspace{8mm}
\noindent\begin{minipage}[t]{\linewidth}%
{\Large\textbf{Pumpkin Muffins}} \label{pumpkin-muffins}\hfill\textit{Sue Dunn}\\
\textbf{Yield:} \textit{12 muffins}\\
\noindent\begin{minipage}[t]{0.78\linewidth}%
\textbf{Ingredients}:\vspace{-3mm}
\begin{multicols}{2}
\begin{itemize}\setlength\itemsep{-1mm}
\item 1 3/4 cups all-purpose flour
\item 1 cup granulated sugar
\item 1/2 cup brown sugar
\item 1 tsp baking soda
\item 1/2 tsp baking powder
\item 1/2 tsp salt
\item 2 tsp ground cinnamon
\item 1/4 tsp ground cloves
\item 1/4 tsp ground nutmeg
\item 2 eggs
\item 1 (15 oz) can pumpkin puree
\item 1/2 cup coconut oil, melted
\item 1 Tbsp whole milk
\item 1 tsp vanilla extract
\end{itemize}
\end{multicols}
\end{minipage}
\noindent\begin{minipage}[t]{0.18\linewidth}
\centering \strut\vspace*{-\baselineskip}\newline
\includegraphics[width=0.9\linewidth]{/home/tim/Documents/projects/recipes/img/none.jpg}\\
\end{minipage}\vspace{3mm}
\textbf{Directions}:
\vspace{-3mm}\begin{enumerate}\setlength\itemsep{-1mm}
\item Preheat oven to 375F. Line a muffin pan with paper liners or grease with non-stick spray.
\item In a medium bowl, combine the flour, sugar, brown sugar, baking soda, baking powder, salt, and spices. Whisk to combine.
\item In a small bowl, combine the eggs, pumpkin puree, coconut oil, mlik, and vanilla extract. Whisk to combine. Pour the wet mixture into the dry ingredients and fold to combine. The batter will be thick.
\item Scopp the batter mix into the prepared muffin pan. Bake for 22 minutes, or 10 minutes for mini muffins.
\end{enumerate}
\end{minipage}\vspace{8mm}
\noindent\begin{minipage}[t]{\linewidth}%
{\Large\textbf{Spinach Balls}} \label{spinach-balls}\hfill\textit{Sue Dunn}\\
\noindent\begin{minipage}[t]{0.78\linewidth}%
\textbf{Ingredients}:\vspace{-3mm}
\begin{multicols}{2}
\begin{itemize}\setlength\itemsep{-1mm}
\item 1 (10 oz) package frozen chopped spinach, thawed and drained
\item 2 cups finely crushed herb-seasoned dry bread stuffing mix
\item 1/2 cup grated parmesan cheese
\item 2 tsp garlic powder
\item 1/2 tsp ground black pepper
\item 1 tsp italian seasoning
\item 1/4 cup melted butter
\item 3 eggs, beaten
\item 3/4 cup chopped onions
\end{itemize}
\end{multicols}
\end{minipage}
\noindent\begin{minipage}[t]{0.18\linewidth}
\centering \strut\vspace*{-\baselineskip}\newline
\includegraphics[width=0.9\linewidth]{/home/tim/Documents/projects/recipes/img/13577261-2CCD-4BCD-BD18-734529F16541.jpg}\\
\end{minipage}\vspace{3mm}
\textbf{Directions}:
\vspace{-3mm}\begin{enumerate}\setlength\itemsep{-1mm}
\item Preheat oven to 350F.
\item In a large bowl, combine spinach, stuffing mix, Parmesan cheese, garlic powder, black pepper, Italian seasoning, melted butter, eggs, and onions. Shape into walnut-sized balls and place on a baking sheet.
\item Bake for 20 minutes, or until lightly browned.
\end{enumerate}
\end{minipage}\vspace{8mm}
\noindent\begin{minipage}[t]{\linewidth}%
{\Large\textbf{Stuffed Mushrooms}} \label{stuffed-mushrooms}\hfill\textit{Sue Dunn}\\
\textbf{Yield:} \textit{12 servings}\\
\noindent\begin{minipage}[t]{0.78\linewidth}%
\textbf{Ingredients}:\vspace{-3mm}
\begin{multicols}{2}
\begin{itemize}\setlength\itemsep{-1mm}
\item 12 fresh mushrooms
\item 1/2 lb ground beef
\item 1 Tbsp onion, minced
\item 1 clove garlic, minced
\item 1 Tbsp butter
\item 1/4 cup bread crumbs
\item salt and pepper
\item 1/4 cup heavy cream
\item 1/4 cup butter, melted
\item 1 tsp chili powder
\end{itemize}
\end{multicols}
\end{minipage}
\noindent\begin{minipage}[t]{0.18\linewidth}
\centering \strut\vspace*{-\baselineskip}\newline
\includegraphics[width=0.9\linewidth]{/home/tim/Documents/projects/recipes/img/none.jpg}\\
\end{minipage}\vspace{3mm}
\textbf{Directions}:
\vspace{-3mm}\begin{enumerate}\setlength\itemsep{-1mm}
\item Preheat oven to 425F. Remove and chop mushroom stems.
\item In a saucepan over medium heat, cook ground beef, onion, and garlic. Mix in chopped mushroom stems, 1 Tbsp butter, bread crumbs, salt, and pepper. Cook, stirring frequently, for 5 minutes. Remove from heat and stir in cream.
\item Dip mushroom caps in 1/4 cup melted butter, and stuff generously with meat mixture. Arrange stuffed mushrooms in a baking dish. Sprinkle with chili powder.
\item Bake for 20-25 minutes.
\end{enumerate}
\end{minipage}\vspace{8mm}
\noindent\begin{minipage}[t]{\linewidth}%
{\Large\textbf{Swedish Meatballs}} \label{swedish-meatballs}\hfill\textit{Sue Dunn}\\
\textbf{Yield:} \textit{serves 6}\\
\noindent\begin{minipage}[t]{0.78\linewidth}%
\textbf{Ingredients}:\vspace{-3mm}
\begin{multicols}{2}
\begin{itemize}\setlength\itemsep{-1mm}
\item 1 1/2 cups soft bread crumbs, about 3 slices bread
\item 1/4 cup finely chopped onion
\item 1 1/2 tsp salt
\item 1/4 tsp pepper
\item 1/4 tsp nutmeg
\item 3/4 cups milk
\item 2 lbs ground beef
\item 2 Tbsp butter
\item 2 Tbsp vegetable oil
\item 2 Tbsp flour
\item 1 (10.5 oz) can condensed beef broth
\item 1 cup half-and-half or light cream
\end{itemize}
\end{multicols}
\end{minipage}
\noindent\begin{minipage}[t]{0.18\linewidth}
\centering \strut\vspace*{-\baselineskip}\newline
\includegraphics[width=0.9\linewidth]{/home/tim/Documents/projects/recipes/img/swedish_meatballs.jpg}\\
\end{minipage}\vspace{3mm}
\textbf{Directions}:
\vspace{-3mm}\begin{enumerate}\setlength\itemsep{-1mm}
\item Combine bread crumbs, onion, salt, pepper, nutmeg and 3/4 cup milk in a large mixing bowl. Let milk soak into crumbs for a few minutes. Gently stir in ground beef until well blended; form into balls about 1 to 1 1/2 inches in diameter. Brown meatballs in butter and oil in a large skillet; remove with a slotted spoon to a 2 1/2-quart baking dish. Drain off all but 2 tablespoons of drippings; stir flour into drippings. Cook, stirring constantly, until bubbly. Stir in beef broth and cream. Continue cooking, stirring constantly, until sauce thickens and boils for a minute. Pour over Swedish meatballs in baking dish. Bake Swedish meatballs at 325F for 35-45 minutes.
\end{enumerate}
\end{minipage}\vspace{8mm}

{\newpage \LARGE \textbf{Breads}} \label{breads}\vspace{4mm}\\
\noindent\begin{minipage}[t]{\linewidth}%
{\Large\textbf{Almond Braid Bread}} \label{almond-braid-bread}\hfill\textit{Sue Dunn}\\
\noindent\begin{minipage}[t]{0.78\linewidth}%
\textbf{Ingredients}:\vspace{-3mm}
\begin{multicols}{2}
\begin{itemize}\setlength\itemsep{-1mm}
\item 2 packages active dry yeast
\item 1 cup warm (105-115F) water
\item 1/2 cup sugar
\item 3/4 cup warm (105-115F) milk
\item 1 tsp salt
\item 1 tsp vanilla extract
\item 1/4 cup butter, softened
\item 5 cups flour
\item 2 eggs
\item 1 egg white
\item 1 tsp water
\item sliced almonds
\end{itemize}
\end{multicols}
\end{minipage}
\noindent\begin{minipage}[t]{0.18\linewidth}
\centering \strut\vspace*{-\baselineskip}\newline
\includegraphics[width=0.9\linewidth]{/home/tim/Documents/projects/recipes/img/none.jpg}\\
\end{minipage}\vspace{3mm}
\textbf{Directions}:
\vspace{-3mm}\begin{enumerate}\setlength\itemsep{-1mm}
\item Sprinkle yeast over the water in a large bowl. Add 1 Tbsp sugar. Let stand until yeast is soft (5 minutes).
\item Add remaining sugar, milk, salt, vanilla, and butter.
\item Add half the flour. Mix to blend, then beat at meduim speed until smooth and elastic (5 minutes). Beat in eggs, one at a time, beating until smooth after each addition. Stir in about 2 cups more flour to make a soft dough.
\item Turn dough out onto a board coated with some of the remaining flour. Knead until dough is smooth and satiny and small bubbles form just under surface (15-20 minutes), adding just enough flour to prevent dough from being sticky.
\item Turn dough in a greased bowl. Cover with plastic wrap and a towel; let rise in a warm place until doubled in bulk (1 - 1 1/2 hours). While dough rises, prepare ``Almond Braid Filling''.
\item Punch dough down. Cover with inverted bowl and let rest for 10 minutes.
\item Divide dough into three equal portions. Roll each out to a 6 x 18 inch rectangle. Spread a third of the filling down the center of each, leaving about a 1/2 inch margin all the way around. Starting from an 18 inch side, roll each strip; pinch ends and long edges to seal.
\item Place filled rolls side by side on a large greased baking sheet; braid, being careful not to stretch strands. Pinch ends of braid and tuck under slightly to seal.
\item Let rise until almost doubled in bulk (35-45 minutes). Preheat oven to 350F.
\item Make egg white mixture: add teaspoon of water to beaten egg white. Brush braid lightly with egg white mixture. Bake until braid is richly browned and sounds hollow when tapped (35-45 minutes). Slide carefully onto a wire rack. Drizzle warm braid with ``Almond Braid Icing''; decorate with sliced almonds. Let cool to room temperature before slicing.
\end{enumerate}
\end{minipage}\vspace{8mm}
\noindent\begin{minipage}[t]{\linewidth}%
{\Large\textbf{Almond Braid Filling}} \label{almond-braid-filling}\hfill\textit{Sue Dunn}\\
\noindent\begin{minipage}[t]{0.78\linewidth}%
\textbf{Ingredients}:\vspace{-3mm}
\begin{multicols}{2}
\begin{itemize}\setlength\itemsep{-1mm}
\item 1 1/2 cups unblanched almonds
\item 1/4 cup fine dry bread crumbs
\item 3/4 cup sugar
\item 1/4 cup butter, melted
\item 3/4 tsp almond extract
\item 1/2 tsp vanilla extract
\item 1 egg, slightly beaten
\end{itemize}
\end{multicols}
\end{minipage}
\noindent\begin{minipage}[t]{0.18\linewidth}
\centering \strut\vspace*{-\baselineskip}\newline
\includegraphics[width=0.9\linewidth]{/home/tim/Documents/projects/recipes/img/none.jpg}\\
\end{minipage}\vspace{3mm}
\textbf{Directions}:
\vspace{-3mm}\begin{enumerate}\setlength\itemsep{-1mm}
\item Whirl almonds in food processor or blender until powdery. Mix almonds, crumbs, and sugar to combine thoroughly. Stir in butter and almond and vanilla extracts. Mix in egg to moisten mixture evenly.
\end{enumerate}
\end{minipage}\vspace{8mm}
\noindent\begin{minipage}[t]{\linewidth}%
{\Large\textbf{Almond Braid Icing}} \label{almond-braid-icing}\hfill\textit{Sue Dunn}\\
\noindent\begin{minipage}[t]{0.78\linewidth}%
\textbf{Ingredients}:\vspace{-3mm}
\begin{multicols}{2}
\begin{itemize}\setlength\itemsep{-1mm}
\item 1 cup confectioner's sugar
\item 1 tsp butter, softened
\item 1/8 tsp almond extract
\item 4 tsp warm water
\end{itemize}
\end{multicols}
\end{minipage}
\noindent\begin{minipage}[t]{0.18\linewidth}
\centering \strut\vspace*{-\baselineskip}\newline
\includegraphics[width=0.9\linewidth]{/home/tim/Documents/projects/recipes/img/none.jpg}\\
\end{minipage}\vspace{3mm}
\textbf{Directions}:
\vspace{-3mm}\begin{enumerate}\setlength\itemsep{-1mm}
\item Mix all ingredients in a small bowl until smooth.
\end{enumerate}
\end{minipage}\vspace{8mm}
\noindent\begin{minipage}[t]{\linewidth}%
{\Large\textbf{Amish Cinnamon Bread}} \label{amish-cinnamon-bread}\hfill\textit{}\\
\textit{``also known as 'friendship bread'''}\\
\noindent\begin{minipage}[t]{0.78\linewidth}%
\textbf{Ingredients}:\vspace{-3mm}
\begin{multicols}{2}
\begin{itemize}\setlength\itemsep{-1mm}
\item 2 1/2 Cups milk
\item 3 Cups sugar
\item 4 Cups flour
\item 1 Cup vegetable oil
\item 1/2 Cup milk
\item 3 eggs
\item 1 tsp vanilla
\item 2 tsp cinnamon
\item 1 box vanilla instant pudding mix (6 ounce)
\item 1/2 Cup nuts (optional)
\item 1/2 Cup raisins (optional)
\end{itemize}
\end{multicols}
\end{minipage}
\noindent\begin{minipage}[t]{0.18\linewidth}
\centering \strut\vspace*{-\baselineskip}\newline
\includegraphics[width=0.9\linewidth]{/home/tim/Documents/projects/recipes/img/amish_cinnamon_bread.jpg}\\
\end{minipage}\vspace{3mm}
\textbf{Directions}:
\vspace{-3mm}\begin{enumerate}\setlength\itemsep{-1mm}
\item Day One: For those making the starter from scratch: combine 1 cup milk, 1 cup sugar, and 1 cup flour in a large zip lock bag and mush to mix ingredients. For those receiving the fermented batter in a gallon zip lock bag: Do nothing. Leave it to sit on the counter.
\item On days 2-4: Squeeze the bag several times during the day. (If air builds up in the bag, open the zip lock slightly and remove the air). I took mine to work, laid it on my desk, and to relieve stress squeezed the bag several times during the day. Ha ha ha!
\item On day 5: add 1 cup milk, 1 cup sugar, and 1 cup self-rising flour to the bag. Squeeze the bag several times during the day.
\item On days 6-8: Squeeze the bag several times during the day.(remove air).
\item On day 9: Add 1 cup milk, 1 cup sugar, and 1 cup self-rising flour into the bag. Close zip lock. Squeeze the bag several times during the day.
\item Day 10: Pour 1/2 cup "starter" in four (4) separate gallon zip lock bags. These starters replace the milk, flour, and sugar used to start the very first batch from scratch. Give the four bags to friends along with the steps on how to finish making their own starters and bread, or freeze the starters for future use if desired, just be sure that once you take a starter out of the freezer, you let it sit out one day before starting your steps.
\item In a large glass bowl add 2 cups self-rising flour, 1 cup of sugar, 3 eggs, 1 cup oil, 2 tsp cinnamon, 1/2 cup milk, 1 tsp vanilla, 1 large box (or 2 small boxes) of instant vanilla pudding, 1/2 cup of either raisins, nuts, chocolate chips or fruit (optional) or 1/4 cup of any two of these ingredients; mix well.
\item Spray well 2 large loaf pans with cooking spray. In a small bowl or cup, mix 1 tsp cinnamon and 2 tbsp sugar. Sprinkle about 1/2 to 2/3 in loaf pans, reserving about 1/3 to 1/2 of the mix. Pour batter into pans.
\item Sprinkle remaining cinnamon and sugar mix across the tops of the batter. (You may choose to sprinkle the remaining mix after baking the bread).
\item Bake at 325F for one hour. You may also make small loaves. If you do, bake at the same temperature, but for 25-30 minutes.
\item Notes: Do not use metal spoon or metal bowl for mixing. Do not refrigerate at any time during the process. Keep on the counter. If air builds up in the zip lock. Open the zipper slightly and squeeze the air out, being careful not to let any of the batter out. Quickly reseal. It is normal for the batter to thicken and bubble during the time it sits on the counter. This is called the fermentation process.
\end{enumerate}
\end{minipage}\vspace{8mm}
\noindent\begin{minipage}[t]{\linewidth}%
{\Large\textbf{Apple Wheat Bread}} \label{apple-wheat-bread}\hfill\textit{Sue Dunn}\\
\textit{``Shredded apple in this mild, wheaty bread adds an elusive flavor and keeps it moist.''}\\
\textbf{Yield:} \textit{2 loaves}\\
\noindent\begin{minipage}[t]{0.78\linewidth}%
\textbf{Ingredients}:\vspace{-3mm}
\begin{multicols}{2}
\begin{itemize}\setlength\itemsep{-1mm}
\item 1 package active dry yeast
\item 1 1/4 cups warm (105-115F) water
\item 1/4 cup firmly packed brown sugar
\item 1 cup warm (105-115F) milk
\item 1 1/2 tsp salt
\item 2 Tbsp salad oil
\item 5 cups unbleached all-purpose flour
\item 1 1/2 cups whole wheat flour
\item 1 large apple, peeled, cored, and shredded
\end{itemize}
\end{multicols}
\end{minipage}
\noindent\begin{minipage}[t]{0.18\linewidth}
\centering \strut\vspace*{-\baselineskip}\newline
\includegraphics[width=0.9\linewidth]{/home/tim/Documents/projects/recipes/img/apple-wheat-bread.jpg}\\
\end{minipage}\vspace{3mm}
\textbf{Directions}:
\vspace{-3mm}\begin{enumerate}\setlength\itemsep{-1mm}
\item Sprinkle yeast over 1/4 cup of the water in a large bowl or electric mixer. Add 1 tsp of the brown sugar. Let stand until soft (5 minutes).
\item Stir in remaining water, milk, remaining brown sugar, salt, and oil.
\item Add 3 1/2 cups of the unbleached flour. Mix to blend, then beat at medium speed until smooth and elastic (5 minutes). Stir in whole wheat flour and apple. Then stir in about 3/4 cup more unbleached flour to make a soft dough.
\item Turn dough out onto a board or pastry cloth coated with some of the remaining 3/4 cup unbleached flour. Knead until dough is smooth and springy and small bubbles form just under surface (12-15 minutes), adding just enough flour to prevent dough from being sticky.
\item Turn dough in a greased bowl. Cover with plastic wrap and a towel; let rise in a warm place until doubled in bulk (1 1/4 - 1 1/2 hours).
\item Punch dough down and divide into two equal portions. Shape each into a loaf. Place loaves in greased 4 1/2 x 8 1/2 inch loaf pans. Let rise until almost doubled in bulk (40-45 minutes).
\item Preheat oven to 350F. Bake until loaves are well browned and sound hollow when tapped (40-45 minutes). Remove loaves from pans and let cool on wire racks.
\end{enumerate}
\end{minipage}\vspace{8mm}
\noindent\begin{minipage}[t]{\linewidth}%
{\Large\textbf{Banana Bread}} \label{banana-bread}\hfill\textit{Sue Dunn}\\
\textit{``from the Alpha Bakery Children's (ABC) cookbook''}\\
\noindent\begin{minipage}[t]{0.78\linewidth}%
\textbf{Ingredients}:\vspace{-3mm}
\begin{multicols}{2}
\begin{itemize}\setlength\itemsep{-1mm}
\item 3/4 cup sugar
\item 1 1/2 cups mashed bananas (3 large)
\item 3/4 cup vegetable oil
\item 2 eggs
\item 2 cups all-purpose flour
\item 1/2 cup chocolate chips
\item 1 tsp baking soda
\item 2 tsp vanilla
\item 1/2 tsp baking powder
\item 1/2 tsp salt
\end{itemize}
\end{multicols}
\end{minipage}
\noindent\begin{minipage}[t]{0.18\linewidth}
\centering \strut\vspace*{-\baselineskip}\newline
\includegraphics[width=0.9\linewidth]{/home/tim/Documents/projects/recipes/img/A1DC5656-FAA8-460B-9E34-05D4EEAB7F19.jpg}\\
\end{minipage}\vspace{3mm}
\textbf{Directions}:
\vspace{-3mm}\begin{enumerate}\setlength\itemsep{-1mm}
\item Heat the oven to 350F. Grease a 9x5x3 loaf pan.
\item Mix sugar, bananas, oil and eggs in a large bowl with a wooden spoon. Stir in remaining ingredients. Pour into pan.
\item Bake until a wooden pick inserted in the center of the bread comes out clean (60-70 minutes). Let cool 10 minutes, then loosen sides of loaf from pan and remove from pan. Let cool completely before slicing. 
\end{enumerate}
\end{minipage}\vspace{8mm}
\noindent\begin{minipage}[t]{\linewidth}%
{\Large\textbf{Cheese Casserole Bread}} \label{cheese-casserole-bread}\hfill\textit{Sue Dunn}\\
\noindent\begin{minipage}[t]{0.78\linewidth}%
\textbf{Ingredients}:\vspace{-3mm}
\begin{multicols}{2}
\begin{itemize}\setlength\itemsep{-1mm}
\item 2 Tbsp sugar
\item 2 tsp salt
\item 2 packages active dry yeast
\item 5 cups all-purpose flour
\item 2 1/4 cups milk
\item 1 cup cheddar, shredded
\item 1 Tbsp butter, melted
\end{itemize}
\end{multicols}
\end{minipage}
\noindent\begin{minipage}[t]{0.18\linewidth}
\centering \strut\vspace*{-\baselineskip}\newline
\includegraphics[width=0.9\linewidth]{/home/tim/Documents/projects/recipes/img/none.jpg}\\
\end{minipage}\vspace{3mm}
\textbf{Directions}:
\vspace{-3mm}\begin{enumerate}\setlength\itemsep{-1mm}
\item In a large bowl, combine sugar, salt, yeast, and 2 cups flour. In 2 quart saucepan over low heat, heat milk and cheese until very warm (120-130F). The cheese doesn't need to melt.
\item Slowly beat liquid mixture into dry ingredients until just blended. At medium speed, beat 2 minutes, scraping bowl with rubber spatula. Beat in 1 cup flour to make a thick batter; beat 2 minutes more, scraping bowl often. Stir in enough additional flour (2 cups) to make a stiff dough that leaves the sides of the bowl.
\item Cover bowl with towel; let rise in warm place until doubled, about 45 minutes.
\item Stir down dough; turn into casserole dish. Cover; let rise in warm place until doubled, about 45 minutes.
\item Preheat oven to 375F. Brush loaf with melted butter. Bake 30-35 minutes until golden.
\end{enumerate}
\end{minipage}\vspace{8mm}
\noindent\begin{minipage}[t]{\linewidth}%
{\Large\textbf{Cornbread}} \label{cornbread}\hfill\textit{Sue Dunn}\\
\textbf{Yield:} \textit{9 servings}\\
\noindent\begin{minipage}[t]{0.78\linewidth}%
\textbf{Ingredients}:\vspace{-3mm}
\begin{multicols}{2}
\begin{itemize}\setlength\itemsep{-1mm}
\item 1 cup cornmeal
\item 1 cup all-purpose flour
\item 1 tsp baking powder
\item 1/2 tsp baking soda
\item 1/8 tsp salt
\item 1/2 cup unsalted butter, melted
\item 1/3 cup dark brown sugar
\item 2 Tbsp honey
\item 1 egg, room temp
\item 1 cup buttermilk, room temp
\end{itemize}
\end{multicols}
\end{minipage}
\noindent\begin{minipage}[t]{0.18\linewidth}
\centering \strut\vspace*{-\baselineskip}\newline
\includegraphics[width=0.9\linewidth]{/home/tim/Documents/projects/recipes/img/none.jpg}\\
\end{minipage}\vspace{3mm}
\textbf{Directions}:
\vspace{-3mm}\begin{enumerate}\setlength\itemsep{-1mm}
\item Preheat oven to 400F. Grease and lightly flour a 9 inch baking pan. Set aside.
\item Whisk the cornmeal, flour, baking powder, baking soda, and salt together in a large bowl. Set aside.
\item In a medium bowl, whisk the melted butter, brown sugar, and honey together until completely smooth. Then, whisk in the egg until combined. Finally, whisk in the buttermilk. Pour the wet ingredients into the dry ingredients and whisk until combined, avoiding overmixing.
\item Pour batter into prepared baking pan. Bake for 20 minutes or until golden brrown on top and the center is cooked through. Allow to slightly cool before slicing and serving. Serve combined with butter, honey, jam, or whatever you like.
\end{enumerate}
\end{minipage}\vspace{8mm}
\noindent\begin{minipage}[t]{\linewidth}%
{\Large\textbf{Pumpkin Chocolate Chip Bread}} \label{pumpkin-chocolate-chip-bread}\hfill\textit{BLC LYO}\\
\textbf{Yield:} \textit{four loaves}\\
\noindent\begin{minipage}[t]{0.78\linewidth}%
\textbf{Ingredients}:\vspace{-3mm}
\begin{multicols}{2}
\begin{itemize}\setlength\itemsep{-1mm}
\item 2 cups sugar
\item 1/2 cup butter, softened
\item 3 large eggs
\item 2 tsp vanilla extract
\item 1 (15 oz) can pumpkin
\item 2 3/4 cups all-purpose flour
\item 3/2 tsp baking powder
\item 1 1/2 tsp baking soda
\item 1 tsp salt
\item 3/4 tsp ground cinnamon
\item 1/4 tsp ground cloves
\item 1/4 tsp ground nutmeg
\item 1/2 cup milk
\item 12 ounce package chocolate chips
\end{itemize}
\end{multicols}
\end{minipage}
\noindent\begin{minipage}[t]{0.18\linewidth}
\centering \strut\vspace*{-\baselineskip}\newline
\includegraphics[width=0.9\linewidth]{/home/tim/Documents/projects/recipes/img/pumpkin-choc-chip-bread.jpeg}\\
\end{minipage}\vspace{3mm}
\textbf{Directions}:
\vspace{-3mm}\begin{enumerate}\setlength\itemsep{-1mm}
\item Grease and flour four mini loaf pans and preheat oven to 350F.
\item Add the sugar and butter to a mixing bowl and beat them with an electric mixer until well combined. Add eggs, pumpkin and vanilla and mix.
\item In a separate bowl mix together flour, baking powder, baking soda, salt, cinnamon, cloves, and nutmeg.
\item Add half of the flour mixture to pumpkin mixture, then add milk, then remaining flour mixture. Fold in chocolate chips at the end.
\item Pour the batter into pans. Bake at 350F for 35-45 minutes or until a toothpick inserted comes out clean. Check every 5 minutes after 35 min.
\item Cool for a few minutes in the pan before inverting onto a wire rack to cool.
\end{enumerate}
\end{minipage}\vspace{8mm}
\noindent\begin{minipage}[t]{\linewidth}%
{\Large\textbf{Zucchini Bread}} \label{zucchini-bread}\hfill\textit{allrecipes.com}\\
\noindent\begin{minipage}[t]{0.78\linewidth}%
\textbf{Ingredients}:\vspace{-3mm}
\begin{multicols}{2}
\begin{itemize}\setlength\itemsep{-1mm}
\item 3 cups all-purpose flour
\item 1 tsp salt
\item 1 tsp baking soda
\item 1 tsp baking powder
\item 3 tsp ground cinnamon
\item 3 eggs
\item 1 cup vegetable oil
\item 2 1/4 cups white sugar
\item 3 tsp vanilla extract
\item 2 cups grated zucchini
\item 1 cup chopped walnuts
\end{itemize}
\end{multicols}
\end{minipage}
\noindent\begin{minipage}[t]{0.18\linewidth}
\centering \strut\vspace*{-\baselineskip}\newline
\includegraphics[width=0.9\linewidth]{/home/tim/Documents/projects/recipes/img/3671A514-1C82-49BB-B842-F8A855CF46B6.jpg}\\
\end{minipage}\vspace{3mm}
\textbf{Directions}:
\vspace{-3mm}\begin{enumerate}\setlength\itemsep{-1mm}
\item Grease and flour two 8 x 4 inch pans. Preheat oven to 325 degrees F (165 degrees C).
\item Sift flour, salt, baking powder, soda, and cinnamon together in a bowl.
\item Beat eggs, oil, vanilla, and sugar together in a large bowl. Add sifted ingredients to the creamed mixture, and beat well. Stir in zucchini and nuts until well combined. Pour batter into prepared pans.
\item Bake for 40 to 60 minutes, or until tester inserted in the center comes out clean. Cool in pan on rack for 20 minutes. Remove bread from pan, and completely cool.
\end{enumerate}
\end{minipage}\vspace{8mm}

{\newpage \LARGE \textbf{Breakfasts}} \label{breakfasts}\vspace{4mm}\\
\noindent\begin{minipage}[t]{\linewidth}%
{\Large\textbf{Ham and Broccoli Casserole}} \label{ham-and-broccoli-casserole}\hfill\textit{}\\
\textbf{Yield:} \textit{Serves 10-12}\\
\noindent\begin{minipage}[t]{0.78\linewidth}%
\textbf{Ingredients}:\vspace{-3mm}
\begin{multicols}{2}
\begin{itemize}\setlength\itemsep{-1mm}
\item 12 slices white bread, crusts removed, torn into pieces
\item 3 cups ham, cooked and cubed
\item 3 cups (2 10 oz pkgs) chopped frozen broccoli, thawed
\item 8 oz shredded sharp cheddar cheese
\item 1/2 cup milk
\item 6 eggs, beaten
\item 1 1/4 tsp salt
\item 1 tsp dry mustard
\item 2 Tbsp melted butter
\end{itemize}
\end{multicols}
\end{minipage}
\noindent\begin{minipage}[t]{0.18\linewidth}
\centering \strut\vspace*{-\baselineskip}\newline
\includegraphics[width=0.9\linewidth]{/home/tim/Documents/projects/recipes/img/D08AE61F-C771-49C4-8DD1-111FCF4EA38E.jpg}\\
\end{minipage}\vspace{3mm}
\textbf{Directions}:
\vspace{-3mm}\begin{enumerate}\setlength\itemsep{-1mm}
\item In a 9x13x2 inch baking dish, put enough of the bread pieces into the bottom of the pan to make a thin layer. Make a layer of ham, broccoli, and cheese. Whisk together milk, eggs, salt, and dry mustard. Pour over top. Toss remaining bread with butter; sprinkle over the top of the casserole. Bake at 325F for 45 minutes. Let cool for about 20 minutes.
\end{enumerate}
\end{minipage}\vspace{8mm}
\noindent\begin{minipage}[t]{\linewidth}%
{\Large\textbf{Pancakes}} \label{pancakes}\hfill\textit{Carolyn Benjamin}\\
\noindent\begin{minipage}[t]{0.78\linewidth}%
\textbf{Ingredients}:\vspace{-3mm}
\begin{multicols}{2}
\begin{itemize}\setlength\itemsep{-1mm}
\item 1 1/2 cups all-purpose flour
\item 3 1/2 tsp baking powder
\item 1 tsp salt
\item 1 Tbsp white sugar
\item 1 1/4 cups milk
\item 1 egg
\item 3 Tbsp butter, melted
\end{itemize}
\end{multicols}
\end{minipage}
\noindent\begin{minipage}[t]{0.18\linewidth}
\centering \strut\vspace*{-\baselineskip}\newline
\includegraphics[width=0.9\linewidth]{/home/tim/Documents/projects/recipes/img/8345AC64-F6A9-48FE-ACE0-4D19E230A1B3.jpg}\\
\end{minipage}\vspace{3mm}
\textbf{Directions}:
\vspace{-3mm}\begin{enumerate}\setlength\itemsep{-1mm}
\item In a large bowl, sift together the flour, baking powder, salt and sugar. Make a well in the center and pour in the milk, egg and melted butter; mix until smooth.
\item Heat a lightly oiled griddle or frying pan over medium high heat. Pour or scoop the batter onto the griddle, using approximately 1/4 cup for each pancake. Brown on both sides and serve hot.
\end{enumerate}
\end{minipage}\vspace{8mm}
\noindent\begin{minipage}[t]{\linewidth}%
{\Large\textbf{Pumpkin Waffles}} \label{pumpkin-waffles}\hfill\textit{Carolyn Benjamin}\\
\textbf{Yield:} \textit{8 waffles}\\
\noindent\begin{minipage}[t]{0.78\linewidth}%
\textbf{Ingredients}:\vspace{-3mm}
\begin{multicols}{2}
\begin{itemize}\setlength\itemsep{-1mm}
\item 1 cup flour
\item 2 tsp baking powder
\item 3/4 tsp ground cinnamon
\item 1/8 tsp salt
\item 1/8 tsp ground cloves
\item 1 cup low fat milk
\item 1/2 cup pumpkin puree
\item 1/4 cup packed dark brown sugar
\item 1 Tbsp vegetable oil
\item 1 large egg, slightly beaten
\end{itemize}
\end{multicols}
\end{minipage}
\noindent\begin{minipage}[t]{0.18\linewidth}
\centering \strut\vspace*{-\baselineskip}\newline
\includegraphics[width=0.9\linewidth]{/home/tim/Documents/projects/recipes/img/6F6656B5-56C4-49D2-A435-402CEA169684.jpg}\\
\end{minipage}\vspace{3mm}
\textbf{Directions}:
\vspace{-3mm}\begin{enumerate}\setlength\itemsep{-1mm}
\item Combine flour and next four ingredients (through cloves) in a large bowl. Make a well in the center of the mixture.
\item Combine milk and next four ingredients (through egg) in a bowl and add flour mixture. Stir until moist.
\item Coat waffle iron with cooking spray (if necessary). Spoon about 1/4 cup batter per waffle into hot waffle iron, spreading batter to edges
\end{enumerate}
\end{minipage}\vspace{8mm}
\noindent\begin{minipage}[t]{\linewidth}%
{\Large\textbf{Red Velvet French Toast}} \label{red-velvet-french-toast}\hfill\textit{Sue Dunn}\\
\textbf{Yield:} \textit{4 servings}\\
\noindent\begin{minipage}[t]{0.78\linewidth}%
\textbf{Ingredients}:\vspace{-3mm}
\begin{multicols}{2}
\begin{itemize}\setlength\itemsep{-1mm}
\item 4 eggs
\item 1 cup buttermilk
\item 2 Tbsp sugar
\item 2 tsp vanilla extract
\item 2 Tbsp cocoa powder
\item 2 Tbsp red food coloring (optional)
\item 8 slices bread, halved
\item 1 (8 oz) pkg cream cheese
\item 2 Tbsp sugar
\item 1 tsp vanilla bean paste
\item fresh raspberries
\end{itemize}
\end{multicols}
\end{minipage}
\noindent\begin{minipage}[t]{0.18\linewidth}
\centering \strut\vspace*{-\baselineskip}\newline
\includegraphics[width=0.9\linewidth]{/home/tim/Documents/projects/recipes/img/none.jpg}\\
\end{minipage}\vspace{3mm}
\textbf{Directions}:
\vspace{-3mm}\begin{enumerate}\setlength\itemsep{-1mm}
\item Whisk together eggs, buttermilk, sugar, vanilla, coloring, and cocoa in a shallow bowl until smooth. In place of buttermilk, you can combine 1 cup milk with 1 Tbsp vinegar and allow to sour for 5 minutes.
\item Dip bread slices into egg wash and press lightly a few times to soak through completely. Flip each slice and repeat until evenly coated.
\item Heat a larget nonstick skillet over low-medium heat and fry the bread in batches of 3 or 4, turning once until just cooked through. Remove and place on warmed plate.
\item While frying french toash, make the cheesecake filling. Combine the cream cheese, sugar, and vanilla paste in a medium sized bowl and whip until light and fluffy. Spoon 1-2 Tbsp of filling onto 6 toast halves; spread to evenly coat; top with remaining halves to create a sandwich. Drizzle with melted chocolate, top with remaining cream, dust with confectioner's sugar, and serve with berries.
\end{enumerate}
\end{minipage}\vspace{8mm}
\noindent\begin{minipage}[t]{\linewidth}%
{\Large\textbf{Waffles}} \label{waffles}\hfill\textit{Sue Dunn}\\
\textbf{Yield:} \textit{4 servings}\\
\noindent\begin{minipage}[t]{0.78\linewidth}%
\textbf{Ingredients}:\vspace{-3mm}
\begin{multicols}{2}
\begin{itemize}\setlength\itemsep{-1mm}
\item 2 cups all-purpose flour
\item 1 Tbsp granulated sugar
\item 4 tsp baking powder
\item 1/4 tsp salt
\item 2 eggs
\item 1 tsp vanilla extract
\item 1/2 cup butter, melted
\item 1 3/4 cups milk
\end{itemize}
\end{multicols}
\end{minipage}
\noindent\begin{minipage}[t]{0.18\linewidth}
\centering \strut\vspace*{-\baselineskip}\newline
\includegraphics[width=0.9\linewidth]{/home/tim/Documents/projects/recipes/img/none.jpg}\\
\end{minipage}\vspace{3mm}
\textbf{Directions}:
\vspace{-3mm}\begin{enumerate}\setlength\itemsep{-1mm}
\item Preheat the waffle iron and grease. Whisk together the dry ingredients - flour, sugar, baking powder, and salt.
\item In another bowl, beat the eggs until light and fluffy. Add the vanilla to the egg mixture, then combine with dry ingredients. Add the butter and milk, blending just until smooth. Let sit for 5 minutes before cooking.
\item Pour onto hot iron and cook until golden brown.
\end{enumerate}
\end{minipage}\vspace{8mm}
\noindent\begin{minipage}[t]{\linewidth}%
{\Large\textbf{Egg Avocado Toast}} \label{egg-avocado-toast}\hfill\textit{Tim Dunn}\\
\noindent\begin{minipage}[t]{0.78\linewidth}%
\textbf{Ingredients}:\vspace{-3mm}
\begin{multicols}{2}
\begin{itemize}\setlength\itemsep{-1mm}
\item olive oil
\item 2 slices bread
\item 3 eggs
\item salt and pepper
\item Italian seasoning
\item fresh mozzarella
\item 1/2 - 1 avocado
\item 1/4 - 1/2 tomato
\item cheddar cheese, shredded
\end{itemize}
\end{multicols}
\end{minipage}
\noindent\begin{minipage}[t]{0.18\linewidth}
\centering \strut\vspace*{-\baselineskip}\newline
\includegraphics[width=0.9\linewidth]{/home/tim/Documents/projects/recipes/img/egg-avocado-toast.jpg}\\
\end{minipage}\vspace{3mm}
\textbf{Directions}:
\vspace{-3mm}\begin{enumerate}\setlength\itemsep{-1mm}
\item Coat a small skillet in olive oil and heat. Toast the bread.
\item Crack the eggs into the heated skillet. Add seasonings. While it cooks, layer the bread with sliced mozzarella, avocado, and tomato.
\item Flip the eggs, and immediately grate cheddar cheese on top. Once cooked, remove the eggs from the skillet and place as the final layer on the toast.
\end{enumerate}
\end{minipage}\vspace{8mm}

{\newpage \LARGE \textbf{Cakes}} \label{cakes}\vspace{4mm}\\
\noindent\begin{minipage}[t]{\linewidth}%
{\Large\textbf{Almond Pound Cake}} \label{almond-pound-cake}\hfill\textit{Sue Dunn}\\
\textbf{Yield:} \textit{serves 8}\\
\noindent\begin{minipage}[t]{0.78\linewidth}%
\textbf{Ingredients}:\vspace{-3mm}
\begin{multicols}{2}
\begin{itemize}\setlength\itemsep{-1mm}
\item 1 cup sugar
\item 3/4 cup butter
\item 3 large eggs
\item 1 tsp almond extract
\item 3 Tbsp water
\item 1 cup flour
\item 1 cup ground almonds
\item 1 1/2 tsp baking powder
\item 1/4 tsp salt
\end{itemize}
\end{multicols}
\end{minipage}
\noindent\begin{minipage}[t]{0.18\linewidth}
\centering \strut\vspace*{-\baselineskip}\newline
\includegraphics[width=0.9\linewidth]{/home/tim/Documents/projects/recipes/img/none.jpg}\\
\end{minipage}\vspace{3mm}
\textbf{Directions}:
\vspace{-3mm}\begin{enumerate}\setlength\itemsep{-1mm}
\item Preheat oven to 325F. Butter and flour a loaf pan. Cream the butter and sugar. Beat in the eggs (individually), almond extract, and water.
\item Stir together the flour, ground almonds, and baking powder. Then combine with wet ingredients. Pour evenly into the pan and bake for 45 minutes.
\end{enumerate}
\end{minipage}\vspace{8mm}
\noindent\begin{minipage}[t]{\linewidth}%
{\Large\textbf{Amaretto Cheesecake}} \label{amaretto-cheesecake}\hfill\textit{Sue Dunn}\\
\textit{``Crust: graham crackers - Amaretto; Filling: cream cheese - white chocolate; Topping: sour cream - almonds''}\\
\noindent\begin{minipage}[t]{0.78\linewidth}%
\textbf{Ingredients}:\vspace{-3mm}
\begin{multicols}{2}
\begin{itemize}\setlength\itemsep{-1mm}
\item 2 1/2 cups graham cracker crumbs (2 bags)
\item 1/4 cup plain almonds, ground finely
\item 1/4 cup sugar
\item 1/2 cup unsalted butter, melted
\item 1 Tbsp Amaretto
\item 2 (8 oz) pkgs cream cheese, softened
\item 2 (8 oz) pkgs Neufchatel cheese, softened
\item 1 1/2 cups sugar
\item 4 eggs
\item 3 Tbsp amaretto
\item 1 tsp almond extract
\item 3/4 tsp vanilla extract
\item pinch of salt
\item 1/4 cup almonds, ground finely
\item 2 oz white chocolate, ground finely
\item 1 (8 oz) container sour cream
\item 1/8 cup sugar
\item 1/2 tsp almond extract
\item 1/4 cup sliced almonds
\end{itemize}
\end{multicols}
\end{minipage}
\noindent\begin{minipage}[t]{0.18\linewidth}
\centering \strut\vspace*{-\baselineskip}\newline
\includegraphics[width=0.9\linewidth]{/home/tim/Documents/projects/recipes/img/none.jpg}\\
\end{minipage}\vspace{3mm}
\textbf{Directions}:
\vspace{-3mm}\begin{enumerate}\setlength\itemsep{-1mm}
\item Grease a 9 inch spring-form pan. Preheat oven to 350F. Combine the graham cracker crumbs, melted butter, almonds, sugar, and amaretto for the crust. Spread the curst into the pan evently and halfway up the sides, pressing down firmly.
\item Cream the cheese with the sugar. Add the eggs one at a time, beating just until combined. Add the almonds, amaretto, almond extract, vanilla extract, salt, and white chocolate. Pour the filling into the crust.
\item Bake for 55-65 minutes, or until the cheesecake is almost set. Meanwhile, mix together the topping (remaining ingredients) in a small bowl. 
\item Remove the cheesecake from the oven and let cool for 5 minutes before spreading topping. Bake another 5 minutes to set the topping. Remove from oven, garnish with almonds, and refrigerate overnight.
\end{enumerate}
\end{minipage}\vspace{8mm}
\noindent\begin{minipage}[t]{\linewidth}%
{\Large\textbf{Angel Food Cake Base}} \label{angel-food-cake-base}\hfill\textit{}\\
\noindent\begin{minipage}[t]{0.78\linewidth}%
\textbf{Ingredients}:\vspace{-3mm}
\begin{multicols}{2}
\begin{itemize}\setlength\itemsep{-1mm}
\item 1 1/4 cups confectioners sugar
\item 1 cup cake flour
\item 1 1/2 cups egg (whites) at room temerature (12-14 egg whites)
\item 1 1/2 tsp cream of tartar
\item 1 1/2 tsp vanilla extract
\item 1/4 tsp salt
\item 1/4 tsp almond extract
\item 1 cup sugar
\end{itemize}
\end{multicols}
\end{minipage}
\noindent\begin{minipage}[t]{0.18\linewidth}
\centering \strut\vspace*{-\baselineskip}\newline
\includegraphics[width=0.9\linewidth]{/home/tim/Documents/projects/recipes/img/CC11502A-C21B-463E-9CCC-F8A88F278887.jpg}\\
\end{minipage}\vspace{3mm}
\textbf{Directions}:
\vspace{-3mm}\begin{enumerate}\setlength\itemsep{-1mm}
\item Preheat oven to 375F. In small bowl, stir confectioner's sugar and cake flour; set aside.
\item Add egg whites, cream of tartar, vanilla extract, salt, and almond extract to large bowl and, with mixer at high speed, beat until well mixed.
\item Beating at high speed, sprinkle in sugar, 2 Tbsps at a time; beat just until sugar dissolves and whites form stiff peaks. Do not scrape bowl during beating.
\item With rubber spatula, fold in flour mixture, about 1/4 at a time, just until flour disappears.
\item Pour mixture into ungreased 10-inch tube pan with spatula, cut through batter to break any large air bubbles.
\item Bake 35 minutes or until top of cake springs back when lightly touched with finger. Any cracks on surface should look dry.
\item Invert cake in pan on funnel; cool completely. With spatula loosen cake from pan and remove to plate.
\end{enumerate}
\end{minipage}\vspace{8mm}
\noindent\begin{minipage}[t]{\linewidth}%
{\Large\textbf{Angel Food Cake Strawberry Sauce}} \label{angel-food-cake-strawberry-sauce}\hfill\textit{Sue Dunn}\\
\noindent\begin{minipage}[t]{0.78\linewidth}%
\textbf{Ingredients}:\vspace{-3mm}
\begin{multicols}{2}
\begin{itemize}\setlength\itemsep{-1mm}
\item 1 Cup sliced fresh strawberries
\item 1 Tbsp sugar
\item 3/4 tsp cornstarch
\item 1/8 tsp almond extract
\item Ice cream or angel food cake
\end{itemize}
\end{multicols}
\end{minipage}
\noindent\begin{minipage}[t]{0.18\linewidth}
\centering \strut\vspace*{-\baselineskip}\newline
\includegraphics[width=0.9\linewidth]{/home/tim/Documents/projects/recipes/img/D49FF126-9D65-4612-ABB0-6F8E70D44160.jpg}\\
\end{minipage}\vspace{3mm}
\textbf{Directions}:
\vspace{-3mm}\begin{enumerate}\setlength\itemsep{-1mm}
\item Combine the strawberries and sugar in a small bowl; cover and refrigerate for 2-3 hours. Drain, reserving juice. Set berries aside. Add water to juice to measure 1/2 cup; pour into a saucepan. Stir in cornstarch until smooth. Bring to a boil; boil and stir for 2 minutes. Remove from the heat; stir in extract. Pour over berries; fold gently. Chill. Serve over ice cream or cake. Yield: 3/4 cup.
\end{enumerate}
\end{minipage}\vspace{8mm}
\noindent\begin{minipage}[t]{\linewidth}%
{\Large\textbf{Apple Cake}} \label{apple-cake}\hfill\textit{Carolyn}\\
\noindent\begin{minipage}[t]{0.78\linewidth}%
\textbf{Ingredients}:\vspace{-3mm}
\begin{multicols}{2}
\begin{itemize}\setlength\itemsep{-1mm}
\item 4 cups sliced apples
\item 2 cups sugar
\item 3/4 cup oil
\item 2 eggs
\item 2 tsp vanilla
\item 3 cups flour
\item 1/4 cup chopped walnuts
\item 1/4 cup raisins
\item 1 tsp salt
\item 2 tsp cinnamon
\item 1 1/2 tsp baking soda
\end{itemize}
\end{multicols}
\end{minipage}
\noindent\begin{minipage}[t]{0.18\linewidth}
\centering \strut\vspace*{-\baselineskip}\newline
\includegraphics[width=0.9\linewidth]{/home/tim/Documents/projects/recipes/img/631F9375-BE85-42D8-B385-1DCA7BDCB152.jpg}\\
\end{minipage}\vspace{3mm}
\textbf{Directions}:
\vspace{-3mm}\begin{enumerate}\setlength\itemsep{-1mm}
\item Dice apples. Beat together eggs, oil, vanilla and sugar. Then add flour, baking soda, salt and cinnamon. Mix well, then add apples, nuts and raisins. Mixture will be thick. Put in well greased (9x13) oblong pan. 
\item Bake a@ 325 for 45-55 minutes. 
OR:
Can be made into muffins (~24-36) grease pans. Bake @ 325 for 25 minutes or until done.
\end{enumerate}
\end{minipage}\vspace{8mm}
\noindent\begin{minipage}[t]{\linewidth}%
{\Large\textbf{Carrot Cake Base}} \label{carrot-cake-base}\hfill\textit{Sue Dunn}\\
\noindent\begin{minipage}[t]{0.78\linewidth}%
\textbf{Ingredients}:\vspace{-3mm}
\begin{multicols}{2}
\begin{itemize}\setlength\itemsep{-1mm}
\item 2 cups all-purpose flour
\item 2 tsp baking powder
\item 1 tsp baking soda
\item 1 1/2 tsp ground cinnamon
\item 1/2 tsp ground ginger
\item 1/4 tsp ground nutmeg
\item 1/2 tsp salt
\item 3/4 cup vegetable oil
\item 4 eggs, room temperature
\item 1 1/2 cups light brown sugar
\item 1/2 cup granulated sugar
\item 1/2 cup unsweetened applesauce
\item 1 tsp vanilla extract
\item 3 cups grated carrots
\end{itemize}
\end{multicols}
\end{minipage}
\noindent\begin{minipage}[t]{0.18\linewidth}
\centering \strut\vspace*{-\baselineskip}\newline
\includegraphics[width=0.9\linewidth]{/home/tim/Documents/projects/recipes/img/none.jpg}\\
\end{minipage}\vspace{3mm}
\textbf{Directions}:
\vspace{-3mm}\begin{enumerate}\setlength\itemsep{-1mm}
\item Preheat the oven to 350F. Spray two 9 inch round cake pans with non-stick spray and set aside.
\item In a large mixing bowl, whisk together the flour, baking powder, baking soda, cinnamon, ginger, nutmeg, and salt until well-combined. Set aside.
\item In a separate large mixing bowl, whisk together the oil, eggs, brown sugar, granulated sugar, applesauce, and vanilla extract until fully combined. Add the grated carrots into the wet ingredients and mix until well combined. Pour the wet ingredients into the dry ingredients and mix until just combined.
\item Pour the cake batter evenly into both pans. Bake at 350F for 30-35 minutes or until set. Remove from oven and cool on wire rack for 20-25 minutes. Once cooled, layer with cream cheese frosting.
\end{enumerate}
\end{minipage}\vspace{8mm}
\noindent\begin{minipage}[t]{\linewidth}%
{\Large\textbf{Carrot Cake Frosting}} \label{carrot-cake-frosting}\hfill\textit{Sue Dunn}\\
\noindent\begin{minipage}[t]{0.78\linewidth}%
\textbf{Ingredients}:\vspace{-3mm}
\begin{multicols}{2}
\begin{itemize}\setlength\itemsep{-1mm}
\item 1 (8 oz) pkg cream cheese, softened
\item 1/2 cup unsalted butter, softened
\item 2 cups powdered sugar
\item 1 tsp vanilla extract
\end{itemize}
\end{multicols}
\end{minipage}
\noindent\begin{minipage}[t]{0.18\linewidth}
\centering \strut\vspace*{-\baselineskip}\newline
\includegraphics[width=0.9\linewidth]{/home/tim/Documents/projects/recipes/img/none.jpg}\\
\end{minipage}\vspace{3mm}
\textbf{Directions}:
\vspace{-3mm}\begin{enumerate}\setlength\itemsep{-1mm}
\item In a large mixing bowl, beat the cream cheese until smooth. Add the butter and mix for one minute until smooth. Add in the powdered sugar and vanilla extract, again mixing until smooth.
\end{enumerate}
\end{minipage}\vspace{8mm}
\noindent\begin{minipage}[t]{\linewidth}%
{\Large\textbf{Chocolate Chip Cheesecake}} \label{chocolate-chip-cheesecake}\hfill\textit{}\\
\noindent\begin{minipage}[t]{0.78\linewidth}%
\textbf{Ingredients}:\vspace{-3mm}
\begin{multicols}{2}
\begin{itemize}\setlength\itemsep{-1mm}
\item 2 cups graham cracker crumbs
\item 1/4 cup sugar
\item 3/4 cup butter (melted)
\item 2 1/4 lb cream cheese, room temperature
\item 1 2/3 cups sugar
\item 5 eggs, room temperature
\item 1 cup bauley's original irish cream
\item 1 cup semisweet chocolate chips
\item 1 cup chilled whipping cream
\item 2 Tbsp sugar
\item 1 tsp instant coffee powder
\end{itemize}
\end{multicols}
\end{minipage}
\noindent\begin{minipage}[t]{0.18\linewidth}
\centering \strut\vspace*{-\baselineskip}\newline
\includegraphics[width=0.9\linewidth]{/home/tim/Documents/projects/recipes/img/8B39CF86-9814-48F4-8032-38018AAB816F.jpg}\\
\end{minipage}\vspace{3mm}
\textbf{Directions}:
\vspace{-3mm}\begin{enumerate}\setlength\itemsep{-1mm}
\item For Crust: preheat oven to 325 F. Coat 9 inch springform pan with nonstick vegetable oil spray. Combine sugar and crumbs in a pan. Stir in butter. Press mixture into bottom and 1 inch up sides of pan. Bake until light brown about 7 minutes. Maintain over at 325 F.
\item For Filling: using electric mixer, beat cream cheese until smooth. Gradually mix in sugar. Beat in 1 egg at a time. Blend in Baileys and vanilla. 
Sprinkle half of chocolate chips over crust. Spoon in filing. Sprinkle remaining chocolate chips. Bake cake until puffed, springy in center and golden brown, about 1 hour and 20 minutes. Cool cake completely. 
\item For Cream: beat cream, sugar and coffee powder until peaks form. Spread mixture over cooled cake. Garnish cheesecake with chocolate curls. Cut in thin slices to serve.
\end{enumerate}
\end{minipage}\vspace{8mm}
\noindent\begin{minipage}[t]{\linewidth}%
{\Large\textbf{Chocolate Lava Cakes}} \label{chocolate-lava-cakes}\hfill\textit{Brian Dunn}\\
\textbf{Yield:} \textit{4 cakes}\\
\noindent\begin{minipage}[t]{0.78\linewidth}%
\textbf{Ingredients}:\vspace{-3mm}
\begin{multicols}{2}
\begin{itemize}\setlength\itemsep{-1mm}
\item 1 stick unsalted butter
\item 6 oz bittersweet chocolate
\item 2 eggs
\item 2 egg yolks
\item 1/4 cup sugar
\item pinch of salt
\item 2 Tbsp all-purpose flour
\end{itemize}
\end{multicols}
\end{minipage}
\noindent\begin{minipage}[t]{0.18\linewidth}
\centering \strut\vspace*{-\baselineskip}\newline
\includegraphics[width=0.9\linewidth]{/home/tim/Documents/projects/recipes/img/none.jpg}\\
\end{minipage}\vspace{3mm}
\textbf{Directions}:
\vspace{-3mm}\begin{enumerate}\setlength\itemsep{-1mm}
\item Preheat the oven to 450F. Butter and lightly flour four ramekins; tap out the excess flour. Set aside.
\item In a double boiler, melt the butter with the chocolate. In a medium bowl, beat the eggs with the yolks, sugar, and salt at high speed until thickened.
\item Whisk the chocolate until smooth. Quickly fold it into the egg mixture along with the flour. Spoon the batter into the prepared ramekins and bake for 12 minutes, or until the sides are firm but center is soft. Le the cakes cool for 1 minute, then cover each with an inverted dinner plate. Carefully turn each over, let stand for 10 seconds, then unmold. Serve immediately.
\end{enumerate}
\end{minipage}\vspace{8mm}
\noindent\begin{minipage}[t]{\linewidth}%
{\Large\textbf{Devil's Food Cake}} \label{devil's-food-cake}\hfill\textit{Andrew Kulawiec}\\
\noindent\begin{minipage}[t]{0.78\linewidth}%
\textbf{Ingredients}:\vspace{-3mm}
\begin{multicols}{2}
\begin{itemize}\setlength\itemsep{-1mm}
\item 2 1/4 cups flour
\item 1/2 cup unsweetened cocoa powder
\item 1 1/2 tsp baking soda
\item 1/2 cup shortening
\item 1 cup sugar
\item 1 tsp vanilla
\item 3 egg yolks
\item 1 1/3 cups cold water
\item 3 egg whites
\item 3/4 cup sugar
\end{itemize}
\end{multicols}
\end{minipage}
\noindent\begin{minipage}[t]{0.18\linewidth}
\centering \strut\vspace*{-\baselineskip}\newline
\includegraphics[width=0.9\linewidth]{/home/tim/Documents/projects/recipes/img/EB8F81C2-A7F0-4FCC-83D0-EBD470961FEC.jpg}\\
\end{minipage}\vspace{3mm}
\textbf{Directions}:
\vspace{-3mm}\begin{enumerate}\setlength\itemsep{-1mm}
\item Grease and flour two 9x1 1/2 inch round pans. Stir together flour, cocoa powder, baking soda and salt. In large mixer bowl, beat shortening on medium for 30 seconds. Add sugar and vanilla, beat until fluffy. Add yolks one at a time, beating on medium for 1 minute after each. Ad dry ingredients and water alternately to beaten mixture, beating on low after each until just combined.
\item Thoroughly wash beaters. In a small bowl beat egg whites 'til soft peaks form; gradually add the 3/4 cups sugar, beating until stiff peaks form. Fold egg white mixture into batter. Combine well. Turn batter into pans. Bake at 350, 30-35 minutes. Cool for 10 minutes on wire racks. Remove from pans. Fill and frost with sea foam frosting. Melt one (1 oz) square of unsweetened chocolate with 1/2 tsp of shortening and drizzle on top.
\end{enumerate}
\end{minipage}\vspace{8mm}
\noindent\begin{minipage}[t]{\linewidth}%
{\Large\textbf{French Almond Cake}} \label{french-almond-cake}\hfill\textit{Sue Dunn}\\
\textbf{Yield:} \textit{12 servings}\\
\noindent\begin{minipage}[t]{0.78\linewidth}%
\textbf{Ingredients}:\vspace{-3mm}
\begin{multicols}{2}
\begin{itemize}\setlength\itemsep{-1mm}
\item 3 1/2 cups almond flour
\item 1/2 Tbsp baking powder
\item 1/4 tsp salt
\item 1/3 cup butter, softened
\item 1/2 cup monk fruit sweetener
\item 4 eggs
\item 3/4 cup sour cream
\item 1/2 tsp vanilla extract
\item 1/2 tsp almond extract
\item 3 Tbsp butter
\item 2 Tbsp monk fruit sweetener
\item 1/4 tsp almond extract
\item 1/4 tsp vanilla extract
\item 1/2 cup sliced almonds
\end{itemize}
\end{multicols}
\end{minipage}
\noindent\begin{minipage}[t]{0.18\linewidth}
\centering \strut\vspace*{-\baselineskip}\newline
\includegraphics[width=0.9\linewidth]{/home/tim/Documents/projects/recipes/img/french-almond-cake.jpeg}\\
\end{minipage}\vspace{3mm}
\textbf{Directions}:
\vspace{-3mm}\begin{enumerate}\setlength\itemsep{-1mm}
\item Preheat the oven to 350 F. Line the bottom of a springform pan or cake pan with parchment paper.
\item Arrange almonds in a single layer on a baking sheet. Toast for 3-4 minutes, until golden. Remove from the oven and allow to cool, leaving the oven on.
\item Beat butter and sweetener together. Mix in the almond flour, baking powder, and sea salt. Then beat in the eggs, sour cream, vanilla extract, and almond extract.
\item Bake at 350 F for 30-45 minutes, until the top is golden and springs back, and inserted toothpick comes out clean.
\item Allow the cake to cool for at least 10 minutes in the pan, until warm but no longer hot.
\end{enumerate}
\end{minipage}\vspace{8mm}
\noindent\begin{minipage}[t]{\linewidth}%
{\Large\textbf{Macademia Nut Cheesecake}} \label{macademia-nut-cheesecake}\hfill\textit{Sue Dunn}\\
\textbf{Yield:} \textit{10 servings}\\
\noindent\begin{minipage}[t]{0.78\linewidth}%
\textbf{Ingredients}:\vspace{-3mm}
\begin{multicols}{2}
\begin{itemize}\setlength\itemsep{-1mm}
\item 1 3/4 cups Graham cracker crumbs (20 crackers)
\item 1/4 cup chopped macademia nuts
\item 1/2 tsp cinnamon
\item 1/2 cup butter, melted
\item 3 eggs, beaten
\item 2 (8 oz) pkgs cream cheese, softened
\item 1 cup sugar
\item 1/4 tsp salt
\item 2 tsp vanilla extract
\item 1/2 tsp almond extract
\item 3 cups sour cream
\item 1/4 cup macademia nuts, finely chopped
\item 1 pint strawberries
\item 1 cup water
\item 1 1/2 Tbsp cornstarch
\item 1/2 cup sugar
\end{itemize}
\end{multicols}
\end{minipage}
\noindent\begin{minipage}[t]{0.18\linewidth}
\centering \strut\vspace*{-\baselineskip}\newline
\includegraphics[width=0.9\linewidth]{/home/tim/Documents/projects/recipes/img/none.jpg}\\
\end{minipage}\vspace{3mm}
\textbf{Directions}:
\vspace{-3mm}\begin{enumerate}\setlength\itemsep{-1mm}
\item Thoroughly mix ingredients for crust (crackers, nuts, cinnamon, butter). Press on bottom and sides of 9 inch springform pan.
\item Combine eggs, cream cheese, sugar, salt, vanilla, and almond extracts. Beat until smooth. Blend in sour cream and nuts. Pour into crumb crust. Bake at 375F for about 35 minutes or until just set. Cool and chill thoroughly, for 4-5 hours.
\item Crush 1 cup strawberries. Add water and cook for 2 minutes. Sieve. Mix cornstarch with sugar and stir into hot berry mixture. Bring to a boil, stirring constantly. Cook and stir until thick and clear. Cool to room temperature. Place remaining strawberries on top of chilled cheesecake. Pour glaze over cake. Chill about 2 hours.
\end{enumerate}
\end{minipage}\vspace{8mm}
\noindent\begin{minipage}[t]{\linewidth}%
{\Large\textbf{Mini Cheesecakes}} \label{mini-cheesecakes}\hfill\textit{Grandma Claire Dunn}\\
\noindent\begin{minipage}[t]{0.78\linewidth}%
\textbf{Ingredients}:\vspace{-3mm}
\begin{multicols}{2}
\begin{itemize}\setlength\itemsep{-1mm}
\item 1 (12 oz) pkg vanilla wafers
\item 2 (8 oz) pkg cream cheese
\item 3/4 cup sugar
\item 2 eggs
\item 1 tsp vanilla extract
\item 1 (21 oz) can cherry pie filling
\end{itemize}
\end{multicols}
\end{minipage}
\noindent\begin{minipage}[t]{0.18\linewidth}
\centering \strut\vspace*{-\baselineskip}\newline
\includegraphics[width=0.9\linewidth]{/home/tim/Documents/projects/recipes/img/25366BCB-A3E5-4589-ACF1-B4B19443572D.jpg}\\
\end{minipage}\vspace{3mm}
\textbf{Directions}:
\vspace{-3mm}\begin{enumerate}\setlength\itemsep{-1mm}
\item Preheat oven to 350 degrees F (175 degrees C). Line miniature muffin tins (tassie pans) with miniature paper liners.
\item Crush the vanilla wafers, and place 1/2 teaspoon of the crushed vanilla wafers into each paper cup.
\item In a mixing bowl, beat cream cheese, sugar, eggs and vanilla until light and fluffy. Fill each miniature muffin liner with this mixture, almost to the top.
\item Bake for 15 minutes. Cool. Top with a teaspoonful of cherry pie filling.
\end{enumerate}
\end{minipage}\vspace{8mm}
\noindent\begin{minipage}[t]{\linewidth}%
{\Large\textbf{Sour Cream Coffee Cake}} \label{sour-cream-coffee-cake}\hfill\textit{Sue Dunn}\\
\textbf{Yield:} \textit{16 servings}\\
\noindent\begin{minipage}[t]{0.78\linewidth}%
\textbf{Ingredients}:\vspace{-3mm}
\begin{multicols}{2}
\begin{itemize}\setlength\itemsep{-1mm}
\item 1/2 cup brown sugar
\item 1/2 cup pecans, chopped
\item 1 1/2 tsp ground cinnamon
\item 3 cups all-purpose flour
\item 1 1/2 tsp baking powder
\item 1 1/2 tsp baking soda
\item 3/4 tsp salt
\item 1 1/2 cup granulated sugar
\item 3/4 cup butter, softened
\item 1 1/2 tsp vanilla
\item 3 eggs
\item 1 1/2 cup sour cream
\item 1/4 cup butter
\item 2 cups powdered sugar
\item 1 tsp vanilla
\item 1-2 Tbsp milk
\end{itemize}
\end{multicols}
\end{minipage}
\noindent\begin{minipage}[t]{0.18\linewidth}
\centering \strut\vspace*{-\baselineskip}\newline
\includegraphics[width=0.9\linewidth]{/home/tim/Documents/projects/recipes/img/sour-cream-coffee-cake.jpg}\\
\end{minipage}\vspace{3mm}
\textbf{Directions}:
\vspace{-3mm}\begin{enumerate}\setlength\itemsep{-1mm}
\item Heat oven to 350F. Grease 10 inch angle food (tube) cake pan.
\item In a smal bowl, mix sugar, pecan, and cinnamon for the filling. Set aside. In a medium bowl, mix the flour, baking powder, bakind soda, and salt; set aside. In a large bowl, beat granulated sugar, butter, vanilla, and eggs. Beat in flour mixture alternately with sour cream slowly.
\item Spread 1/3 of the batter (about 2 cups) in pan; sprinkle with 1/3 of the filling. Repeat twice.
\item Bake for 1 hour. Cool for 10 minutes before removing from pan.
\item Meanwhile, in a saucepan, brown butter. Remove from heat and mix with powdered sugar and vanilla. Stir in just enough milk until glaze is smooth and thin enough to drizzle.
\end{enumerate}
\end{minipage}\vspace{8mm}
\noindent\begin{minipage}[t]{\linewidth}%
{\Large\textbf{Triple Chocolate Cheesecake}} \label{triple-chocolate-cheesecake}\hfill\textit{Sue Dunn}\\
\textit{``Crust: Oreos and butter; Filling: cream cheese - 10oz chocolate; Topping: heavy cream - sugar''}\\
\noindent\begin{minipage}[t]{0.78\linewidth}%
\textbf{Ingredients}:\vspace{-3mm}
\begin{multicols}{2}
\begin{itemize}\setlength\itemsep{-1mm}
\item 24 Oreo cookies, finely crushed
\item 1/4 cup unsalted butter, melted
\item 2 lbs cream cheese, room temp
\item 1 1/3 cups powdered sugar
\item 3 Tbsp cocoa powder
\item 4 eggs, room temp
\item 10 oz bittersweet chocolate, chopped
\item 3/4 cup heavy cream
\item 6 oz bittersweet chocolate, finely chopped
\item 1 Tbsp granulated sugar
\end{itemize}
\end{multicols}
\end{minipage}
\noindent\begin{minipage}[t]{0.18\linewidth}
\centering \strut\vspace*{-\baselineskip}\newline
\includegraphics[width=0.9\linewidth]{/home/tim/Documents/projects/recipes/img/43658F4A-0D1D-407D-9879-F76AD6FB686B.jpg}\\
\end{minipage}\vspace{3mm}
\textbf{Directions}:
\vspace{-3mm}\begin{enumerate}\setlength\itemsep{-1mm}
\item Crust: Preheat oven to 350F. Grease a 9 inch springform pan and set aside. Finely crush cookies, add melted butter, and mix. Press crumb mixture onto the bottom of the prepared pan and bake for 8 minutes. Remove from oven and set on while rack to cool.
\item Filling: Melt chocolate and set aside to cool. Mix cream cheese and sugar until smooth; mix in cocoa powder. Add the eggs on at a time, mixing on low speed and not overbeating. Add melted chocolate and keep mixing on low. Pour the filling over the crust and smooth the top. Bake until the center looks dry (about 1 hour 10 minutes). Cool on a wire rack for 5 minutes, then run a thin knife around the sides of the pan. Let it set in a fridge for at least 8 hours.
\item In a medium saucepan, stir together cream, chocolate, and sugar on low heat until the chocolate is completely melted and the texture is smooth. Cool and pour over the cheesecake.
\end{enumerate}
\end{minipage}\vspace{8mm}

{\newpage \LARGE \textbf{Casseroles}} \label{casseroles}\vspace{4mm}\\
\noindent\begin{minipage}[t]{\linewidth}%
{\Large\textbf{Bacon Egg Spinach Casserole}} \label{bacon-egg-spinach-casserole}\hfill\textit{Sue Dunn}\\
\noindent\begin{minipage}[t]{0.78\linewidth}%
\textbf{Ingredients}:\vspace{-3mm}
\begin{multicols}{2}
\begin{itemize}\setlength\itemsep{-1mm}
\item 10 eggs
\item 2 1/2 cups frozen spinach
\item 6 slices bacon, cooked and crumbled
\item 1 cup mushrooms, sliced
\item 1/2 cup red onion, chopped
\item 1/2 green pepper, chopped
\item 1/2 red pepper, chopped
\item 1 1/4 cup shredded cheddar
\item salt and pepper to taste
\end{itemize}
\end{multicols}
\end{minipage}
\noindent\begin{minipage}[t]{0.18\linewidth}
\centering \strut\vspace*{-\baselineskip}\newline
\includegraphics[width=0.9\linewidth]{/home/tim/Documents/projects/recipes/img/none.jpg}\\
\end{minipage}\vspace{3mm}
\textbf{Directions}:
\vspace{-3mm}\begin{enumerate}\setlength\itemsep{-1mm}
\item Preheat oven to 375 F. Spray a 9x13 baking dish with cooking spray.
\item Place a skillet on medium-high heat. Add the chopped veggies (excluding spinach) to the pan. Saute for a few minutes until the veggies are soft. Pour into baking dish.
\item Add the spinach, in another layer. 
\item In a medium bowl, whisk the eggs. Season with salt and pepper. Pour the egg mixture over the veggies.
\item Add one last layer of bacon and shredded cheese. Bake for 35 minutes.
\end{enumerate}
\end{minipage}\vspace{8mm}
\noindent\begin{minipage}[t]{\linewidth}%
{\Large\textbf{Broccoli Chicken Casserole}} \label{broccoli-chicken-casserole}\hfill\textit{Nancy Feth}\\
\noindent\begin{minipage}[t]{0.78\linewidth}%
\textbf{Ingredients}:\vspace{-3mm}
\begin{multicols}{2}
\begin{itemize}\setlength\itemsep{-1mm}
\item 2 (10 oz) packages proccoli cuts
\item 3 whole chicken breasts (cooked, boned and cut in pieces)
\item 2 cans condensed cream of chicken soup
\item 1 cup mayonnaise
\item 1 tsp lemon juice
\item 1/2 tsp curry powder
\item 1 cup shredded sharp cheddar cheese
\end{itemize}
\end{multicols}
\end{minipage}
\noindent\begin{minipage}[t]{0.18\linewidth}
\centering \strut\vspace*{-\baselineskip}\newline
\includegraphics[width=0.9\linewidth]{/home/tim/Documents/projects/recipes/img/B22FE9D1-F626-4C5B-B0BD-B4F423DB41FA.jpg}\\
\end{minipage}\vspace{3mm}
\textbf{Directions}:
\vspace{-3mm}\begin{enumerate}\setlength\itemsep{-1mm}
\item Place uncooked broccoli in 9x13 baking dish. Season with salt and pepper.
\item Place cut-up chicken pieces on top. Make a sauce with soup, mayonnaise, lemon juice and curry powder. Pour over chicken. Sprinkle with cheese.
\item Bake for 30 minutes at 350F
\item If made 1 day in advance, refrigerate and bake slightly longer
\end{enumerate}
\end{minipage}\vspace{8mm}
\noindent\begin{minipage}[t]{\linewidth}%
{\Large\textbf{Cauliflower Casserole}} \label{cauliflower-casserole}\hfill\textit{Sue Dunn}\\
\textbf{Yield:} \textit{8 servings, 1/2 cup each}\\
\noindent\begin{minipage}[t]{0.78\linewidth}%
\textbf{Ingredients}:\vspace{-3mm}
\begin{multicols}{2}
\begin{itemize}\setlength\itemsep{-1mm}
\item 1 large head cauliflower, cut into small florets
\item 2 Tbsp butter, melted
\item sea salt
\item black pepper
\item 2/3 cup sour cream
\item 1/4 cup heavy cream
\item 2 cloves garlic, minced
\item 1 1/2 cup cheddar cheese, shredded
\item 6 Tbsp bacon bits
\item 1/4 cup green onions
\end{itemize}
\end{multicols}
\end{minipage}
\noindent\begin{minipage}[t]{0.18\linewidth}
\centering \strut\vspace*{-\baselineskip}\newline
\includegraphics[width=0.9\linewidth]{/home/tim/Documents/projects/recipes/img/cauliflower.jpg}\\
\end{minipage}\vspace{3mm}
\textbf{Directions}:
\vspace{-3mm}\begin{enumerate}\setlength\itemsep{-1mm}
\item Preheat the oven to 450F.
\item In a large bowl, toss the cauliflower florets with butter. Season with salt and black pepper.
\item Roast on a baking sheet for 15-20 minutes, until crisp and tender.
\item Meanwhile, in the same bowl, whisk together the sour and heavy creams, until smooth. Stir in the minced garlic, half of the cheddar cheese, half of the bacon bits, and half of the green onions. If desired, season sauce with sea salt and black pepper (keep in mind, the cheese will make it saltier as it melts).
\item When the cauliflower is done roasting, take it out and leave the oven on. Add the cauliflower to the bowl and mix with the sauce. Transfer to the casserole dish, and top with the remaining cheese and bacon bits.
\item Bake for 5-10 minutes, until the cheese melts. Top with remaining green onions.
\end{enumerate}
\end{minipage}\vspace{8mm}
\noindent\begin{minipage}[t]{\linewidth}%
{\Large\textbf{Spinach Artichoke Chicken Casserole}} \label{spinach-artichoke-chicken-casserole}\hfill\textit{Sue Dunn}\\
\textbf{Yield:} \textit{6 servings}\\
\noindent\begin{minipage}[t]{0.78\linewidth}%
\textbf{Ingredients}:\vspace{-3mm}
\begin{multicols}{2}
\begin{itemize}\setlength\itemsep{-1mm}
\item 1 Tbsp butter, salted
\item 2 Tbsp red onion, diced
\item 2 cups spinach, raw
\item 1 (14 oz) can artichoke hearts, chopped
\item 1/2 jalapeno, chopped
\item salt and pepper
\item 1 (8 oz) pkg cream cheese, softened
\item 1/3 cup mayonnaise
\item 1/3 cup sour cream
\item 1/4 tsp garlic powder
\item 3 cups chicken breast, cooked
\item 1 cup mozzarella cheese, shredded
\item 1/4 cup Parmesan, shredded
\end{itemize}
\end{multicols}
\end{minipage}
\noindent\begin{minipage}[t]{0.18\linewidth}
\centering \strut\vspace*{-\baselineskip}\newline
\includegraphics[width=0.9\linewidth]{/home/tim/Documents/projects/recipes/img/none.jpg}\\
\end{minipage}\vspace{3mm}
\textbf{Directions}:
\vspace{-3mm}\begin{enumerate}\setlength\itemsep{-1mm}
\item Preheat oven to 400F. In a large skillet over medium heat, melt butter and saute onion until translucent. Add spinach, artichokes, and jalapeno. Saute until spinach is wilted. Sprinkle in salt and pepper.
\item In a large bowl, mix cream cheese, mayo, sour cream, and garlic powder until smooth. Add mixture to skillet.
\item Add cooked cubed chicken and mozzarella to skillet. Stir. Pour contents into baking dish or oven-safe skillet. Top with Parmesan and bake for 20 minutes or until bubbly and brown.
\end{enumerate}
\end{minipage}\vspace{8mm}

{\newpage \LARGE \textbf{Chicken}} \label{chicken}\vspace{4mm}\\
\noindent\begin{minipage}[t]{\linewidth}%
{\Large\textbf{Buffalo Chicken Mac and Cheese}} \label{buffalo-chicken-mac-and-cheese}\hfill\textit{Sue Dunn}\\
\textbf{Yield:} \textit{serves 8}\\
\noindent\begin{minipage}[t]{0.78\linewidth}%
\textbf{Ingredients}:\vspace{-3mm}
\begin{multicols}{2}
\begin{itemize}\setlength\itemsep{-1mm}
\item 7 Tbsp unsalted butter
\item salt
\item 1 lb elbow macaroni
\item 1 onion, finely chopped
\item 2 stalks celery, finely chopped
\item 3 cups shredded chicken (rotisserie)
\item 2 cloves garlic, minced
\item 3/4 cup hot sauce (Frank's or Louisiana)
\item 2 Tbsp all-purpose flour
\item 2 tsp dry mustard
\item 2 1/2 cups half-and-half
\item 1 lb sharp cheddar, cubed
\item 8 oz pepper jack cheese, shredded
\item 2/3 cup sour cream
\item 1 cup Panko breadcrumbs
\item 3/4 cup crumbled blue cheese
\item 2 Tbsp parsley, freshly chopped
\end{itemize}
\end{multicols}
\end{minipage}
\noindent\begin{minipage}[t]{0.18\linewidth}
\centering \strut\vspace*{-\baselineskip}\newline
\includegraphics[width=0.9\linewidth]{/home/tim/Documents/projects/recipes/img/none.jpg}\\
\end{minipage}\vspace{3mm}
\textbf{Directions}:
\vspace{-3mm}\begin{enumerate}\setlength\itemsep{-1mm}
\item Preheat the oven to 350F and butter a 9x13 baking dish. Bring a large pot of salted water to a boil; add the pasta and cook until al dente, about 7 minutes. Drain.
\item Meanwhile, melt 3 Tbsp butter in a saucepan over medium heat. Add the onion and celery and cook until soft, about 5 minutes. Stir in the chicken and garlic and cook 2 minutes, then add 1/2 cup hot sauce and simmer until slightly thickened, about one more minute.
\item Stir in the flour and mustard until smooth. Whisk in the half-and-half, then add the remaining 1/4 cup hot sauce and stir until thick, about 2 minutes. Whisk in the cheddar and pepper jack cheeses, then whisk in the sour cream until smooth.
\item Spread half of the macaroni in the prepared baking dish, then top with the chicken mixture and the remaining macaroni. Pour the cheese sauce evenly on top.
\item Melt the remaining 2 Tbsp butter, then stir in the breadcrumbs, blue cheese, and parsley. Sprinkle over the macaroni and bake until bubbly, 30-40 minutes. Let rest for 10 minutes before serving.
\end{enumerate}
\end{minipage}\vspace{8mm}
\noindent\begin{minipage}[t]{\linewidth}%
{\Large\textbf{Chicken Cordon Bleu, Baked}} \label{chicken-cordon-bleu,-baked}\hfill\textit{Sue Dunn}\\
\noindent\begin{minipage}[t]{0.78\linewidth}%
\textbf{Ingredients}:\vspace{-3mm}
\begin{multicols}{2}
\begin{itemize}\setlength\itemsep{-1mm}
\item cooking spray
\item 12 chicken breasts
\item salt and pepper
\item 1 egg
\item 2 egg whites
\item 1 Tbsp water
\item 1/2 cup seasoned breadcrumbs
\item 1/4 cup grated Parmesan cheese
\item 5 oz (6 slices) deli ham
\item 6 slices Swiss cheese
\end{itemize}
\end{multicols}
\end{minipage}
\noindent\begin{minipage}[t]{0.18\linewidth}
\centering \strut\vspace*{-\baselineskip}\newline
\includegraphics[width=0.9\linewidth]{/home/tim/Documents/projects/recipes/img/none.jpg}\\
\end{minipage}\vspace{3mm}
\textbf{Directions}:
\vspace{-3mm}\begin{enumerate}\setlength\itemsep{-1mm}
\item Preheat oven to 450F. Spray a large baking sheet with cooking spray. Wash and dry chicken cutlets; lightly pound the chicken to make thinner and lightly season with salt and pepper.
\item Lay the chicken on a working surface and place a slice of ham and cheese on top of the chicken. Roll, then set them aside with the seam down, securing with toothpicks as necessary.
\item In a medium bowl, whisk eggs and whites with water to make an egg wash. In a separate bowl, combine the breadcrumbs and parmesan cheese. Dip the chicken into the egg wash, and then the breadcrumbs.
\item Place chicken onto the baking sheet seams side down. Spray the top of the chicken with cooking spray and bake about 25 minutes, or until cooked.
\end{enumerate}
\end{minipage}\vspace{8mm}
\noindent\begin{minipage}[t]{\linewidth}%
{\Large\textbf{Chicken Cordon Bleu}} \label{chicken-cordon-bleu}\hfill\textit{Sue Dunn}\\
\noindent\begin{minipage}[t]{0.78\linewidth}%
\textbf{Ingredients}:\vspace{-3mm}
\begin{multicols}{2}
\begin{itemize}\setlength\itemsep{-1mm}
\item 6 chicken breast halves (3 breasts)
\item 6 slices of swiss cheese
\item 6 slices of maple ham
\item 3 Tbsp all-purpose flour
\item 1 tsp paprika
\item 6 Tbsp butter
\item 1/2 cup dry white wine
\item 1 tsp chicken bouillon (1 cube)
\item 1 Tbsp corn starch
\item 1 cup heavy whipping cream
\end{itemize}
\end{multicols}
\end{minipage}
\noindent\begin{minipage}[t]{0.18\linewidth}
\centering \strut\vspace*{-\baselineskip}\newline
\includegraphics[width=0.9\linewidth]{/home/tim/Documents/projects/recipes/img/chicken_cordon_bleu.jpg}\\
\end{minipage}\vspace{3mm}
\textbf{Directions}:
\vspace{-3mm}\begin{enumerate}\setlength\itemsep{-1mm}
\item Pound chicken breasts if too thick. Place cheese and ham on each breast. Fold chicken over filling and secure with toothpicks. Mix the flour and paprika in a small bowl and coat the chicken pieces.
\item Heat the butter in a large skillet over medium-high heat and cook the chicken until browned on all sides. Add the wine and bouillon. Reduce heat to low, cover and simmer for 30 minutes until chicken is no longer pink.
\item Remove the toothpicks and transfer the breasts to a warm platter. Blend cornstarch with cream in a small bowl and whisk slowly into skillet. Cook, stirring until thickened, and pour over chicken. Serve warm. 
\end{enumerate}
\end{minipage}\vspace{8mm}
\noindent\begin{minipage}[t]{\linewidth}%
{\Large\textbf{Chicken Devan}} \label{chicken-devan}\hfill\textit{Nancy Feth}\\
\noindent\begin{minipage}[t]{0.78\linewidth}%
\textbf{Ingredients}:\vspace{-3mm}
\begin{multicols}{2}
\begin{itemize}\setlength\itemsep{-1mm}
\item 2 pkgs broccoli cuts
\item 3 whole chicken breasts (cooked, boned and cut in pieces)
\item 2 cans condensed cream of chicken soup
\item 1 cup mayonnaise
\item 1 tsp lemon juice
\item 1/2 tsp curry powder
\item 1 cup shredded cheddar cheese
\end{itemize}
\end{multicols}
\end{minipage}
\noindent\begin{minipage}[t]{0.18\linewidth}
\centering \strut\vspace*{-\baselineskip}\newline
\includegraphics[width=0.9\linewidth]{/home/tim/Documents/projects/recipes/img/BEFEED18-A837-4235-853D-6933B6FB94FE.jpg}\\
\end{minipage}\vspace{3mm}
\textbf{Directions}:
\vspace{-3mm}\begin{enumerate}\setlength\itemsep{-1mm}
\item Place broccoli in 9x13 pan season with salt and pepper. Place chicken on top, make a sauce of soup, mayo, lemon juice and curry. Pour over above, sprinkle with cheese.
\item Bake 30 mins at 350
\end{enumerate}
\end{minipage}\vspace{8mm}
\noindent\begin{minipage}[t]{\linewidth}%
{\Large\textbf{Chicken Divan}} \label{chicken-divan}\hfill\textit{Sue Dunn}\\
\noindent\begin{minipage}[t]{0.78\linewidth}%
\textbf{Ingredients}:\vspace{-3mm}
\begin{multicols}{2}
\begin{itemize}\setlength\itemsep{-1mm}
\item 1 lb broccoli cuts
\item 1 (8 oz) pkg cream cheese
\item 1 cup mayonnaise
\item 3/4 cup heavy cream
\item 1/2 tsp garlic powder
\item 1/2 tsp onion powder
\item 1/2 tsp ground mustard
\item 1 tsp salt
\item 2 bouillon cubes
\item 4 cups chicken breast, shredded
\item 2 cups cheddar cheese, shredded
\item cooked rice (optional)
\end{itemize}
\end{multicols}
\end{minipage}
\noindent\begin{minipage}[t]{0.18\linewidth}
\centering \strut\vspace*{-\baselineskip}\newline
\includegraphics[width=0.9\linewidth]{/home/tim/Documents/projects/recipes/img/none.jpg}\\
\end{minipage}\vspace{3mm}
\textbf{Directions}:
\vspace{-3mm}\begin{enumerate}\setlength\itemsep{-1mm}
\item Bring a large pot of water to a rapid boil. Add in broccoli and cook for 3 minutes. Remove from heat and drain. Set broccoli aside.
\item In a medium bowl, microwave cream cheese, mayo, and heavy cream with spices, salt, and bouillon for 1 minute at a time for up to 3 minutes. Whisk until smooth.
\item In a 9x13 pan, mix together chicken with sauce and broccoli. Top with cheddar cheese and bake at 375 F for 25 minutes. Serve warm over rice.
\end{enumerate}
\end{minipage}\vspace{8mm}
\noindent\begin{minipage}[t]{\linewidth}%
{\Large\textbf{Chicken Enchiladas}} \label{chicken-enchiladas}\hfill\textit{Sue Dunn}\\
\textbf{Yield:} \textit{4 servings}\\
\noindent\begin{minipage}[t]{0.78\linewidth}%
\textbf{Ingredients}:\vspace{-3mm}
\begin{multicols}{2}
\begin{itemize}\setlength\itemsep{-1mm}
\item 1 small onion, chopped
\item vegetable oil
\item 2 cloves garlic, minced
\item 1 (15.5 oz) can roasted tomatoes
\item 2 Tbsp red chili powder
\item 1 tsp sugar
\item 1/2 to a cup of water
\item 12 corn tortillas
\item 2 cups of cooked chicken, shredded or chopped
\item salt
\item 2 cups grated cheese
\end{itemize}
\end{multicols}
\end{minipage}
\noindent\begin{minipage}[t]{0.18\linewidth}
\centering \strut\vspace*{-\baselineskip}\newline
\includegraphics[width=0.9\linewidth]{/home/tim/Documents/projects/recipes/img/FF7A35FC-D2F8-4181-B882-42AA5ADC8179.jpg}\\
\end{minipage}\vspace{3mm}
\textbf{Directions}:
\vspace{-3mm}\begin{enumerate}\setlength\itemsep{-1mm}
\item Preheat the oven to 350F.
\item Prepare the sauce: Coat a large skillet with oil and saut the onions on medium heat until translucent, a few minutes. Add the garlic for a minute more. While the onions are cooking, puree the canned tomatoes in a blender. Add the tomatoes to the onions and garlic. Bring to a low simmer. Start adding the chili powder, one teaspoon at a time, tasting after each addition, until you get to the desired level of heat and chili flavor. For us that's around 2 Tablespoons. But it depends on your taste and how strong the chili powder is that you are using. Note that the tortillas and chicken will absorb some of the heat, so allow for that and let it be a little bit spicier than what you want in the finished dish. Add a teaspoon of sugar if necessary to cut down on the acid from the tomatoes. You want more of the taste of the chili and less of the tomatoes for this sauce. As the sauce simmers, dilute it with water to keep it from getting too thick as it simmers. Remove from heat.
\item Mix in 1/4 cup of the sauce with the cooked chicken, and a 1/4 cup of the cheese. Sprinkle with a little salt. Set aside.
\item Prepare the tortillas: Heat a small light skillet on med-high heat. Add a teaspoon of oil (high smoke point oil as indicated above, we use grapeseed oil) to coat the pan. Dip a tortilla in the sauce to coat the tortilla with sauce on both sides. Place the tortilla in the skillet and heat for a few seconds, until the tortilla begin to show some air bubbles. Use a metal spatula to flip to the other side for a few more seconds. Set aside on a plate. Repeat with remaining tortillas.
\item Assemble the enchiladas. Use an 8x12 inch baking dish. Place a couple spoonfuls of the chicken mixture in the center of a tortilla and roll it up. Place in the baking dish and repeat until all dozen of your tortillas are neatly placed in rows in the casserole dish. Cover the tortillas rolls with the remaining sauce. Sprinkle with the remaining grated cheese. Note that I recall often eating these chicken enchiladas with very little cheese on them. Instead we had probably 2/3 cup of chopped fresh onion that had been soaked in vinegar sprinkled over the top.
\item Place in the oven and cook for 10 minutes, or until cheese is bubbly.
\item Serve with thinly sliced iceberg lettuce that has been seasoned with vinegar and salt (no oil), guacamole or avocado slices, and sour cream. Garnish with cilantro.
\end{enumerate}
\end{minipage}\vspace{8mm}
\noindent\begin{minipage}[t]{\linewidth}%
{\Large\textbf{Chicken French}} \label{chicken-french}\hfill\textit{Sue Dunn}\\
\textbf{Yield:} \textit{4 servings}\\
\noindent\begin{minipage}[t]{0.78\linewidth}%
\textbf{Ingredients}:\vspace{-3mm}
\begin{multicols}{2}
\begin{itemize}\setlength\itemsep{-1mm}
\item 1/4 cup all-purpose flour
\item salt and pepper
\item 2 eggs, beaten
\item 1 Tbsp white sugar
\item 1 Tbsp grated Parmesan
\item 2 Tbsp olive oil
\item 4 chicken breasts
\item 1/4 cup butter
\item 2 tsp garlic, minced
\item 1/4 cup dry sherry
\item 1/4 cup lemon juice
\item 2 tsp low-sodium chicken base
\end{itemize}
\end{multicols}
\end{minipage}
\noindent\begin{minipage}[t]{0.18\linewidth}
\centering \strut\vspace*{-\baselineskip}\newline
\includegraphics[width=0.9\linewidth]{/home/tim/Documents/projects/recipes/img/none.jpg}\\
\end{minipage}\vspace{3mm}
\textbf{Directions}:
\vspace{-3mm}\begin{enumerate}\setlength\itemsep{-1mm}
\item Mix together the flour, salt, and pepper in a shallow bowl. In another bowl, whisk eggs, sugar, and Parmesan cheese until the mixture is thoroughly blended and the sugar has dissolved.
\item Heat olive oil over medium heat in a large skillet until the oil shimmers. Dip the chicken breasts in the flour mixture, then into the egg mixture, and gently lay them in the skillet. Pan-fry the chicken breasts until golden brown and no longer pink in the middle, about 6 minutes per side. Remove from the skillet and set aside.
\item In the same skilled over medium-low heat, melt the butter and stir in garlic, sherry, lemon juice, and chicken base. Bring the sauce to a simmer, and stir until smooth and slightly thickened, about 5 minutes. Be sure to dissolve any brown flavorful bits from the bottom of the skillet as you stir. Return the chicken breasts to the sauce, and gently simmer for about 15 minutes.
\end{enumerate}
\end{minipage}\vspace{8mm}
\noindent\begin{minipage}[t]{\linewidth}%
{\Large\textbf{Chicken Pot Pie}} \label{chicken-pot-pie}\hfill\textit{Sue Dunn}\\
\noindent\begin{minipage}[t]{0.78\linewidth}%
\textbf{Ingredients}:\vspace{-3mm}
\begin{multicols}{2}
\begin{itemize}\setlength\itemsep{-1mm}
\item 4 chicken breasts
\item 1/2 cup butter
\item salt and pepper
\item 2 large carrots, peeled and diced
\item 1 onion, chopped
\item 3 cloves garlic, minced
\item 3/4 cup all-purpose flour
\item 3 cups low-sodium chicken broth
\item 1/4 cup heavy cream
\item 1 cup frozen peas
\item 2 Tbsp parsley, freshly chopped
\item 2 tsp thyme, freshly chopped
\item egg wash
\end{itemize}
\end{multicols}
\end{minipage}
\noindent\begin{minipage}[t]{0.18\linewidth}
\centering \strut\vspace*{-\baselineskip}\newline
\includegraphics[width=0.9\linewidth]{/home/tim/Documents/projects/recipes/img/none.jpg}\\
\end{minipage}\vspace{3mm}
\textbf{Directions}:
\vspace{-3mm}\begin{enumerate}\setlength\itemsep{-1mm}
\item Make ``Biscuit Pie Crust''.
\item Preheat oven to 400 F. Grease a large baking dish with butter and grease one side of a large piece of parchment with butter. Season chicken all over with salt and pepper then place in baking dish. Place buttered side of parchment paper over chicken, so that chicken is completely covered. Bake until chicken is cooked through, 30 to 40 minutes. Let rest 10 minutes before cutting into cubes.
\item Meanwhile, start making the filling: In a large pot over medium heat, melt butter. Add onions and carrots and cook until vegetables begin to soften, about 10 minutes. Stir in garlic, then stir in flour and cook until the flour mixture is golden and begins to bubble. Gradually whisk in chicken broth. Bring mixture to a boil and cook until thickened, about 5 minutes. Stir in heavy cream, chicken, peas, parsley, and thyme. Season with salt and pepper.
\item Place half of rolled dough in pie dish. Add filling, then place remaining dough on top. Crimp edges, slit to vent. Brush with egg wash and sprinkle with flaky sea salt.
\item Reduce heat to 375 F and bake pie until crust is golden, about 45 minutes. Let cool before serving.
\end{enumerate}
\end{minipage}\vspace{8mm}
\noindent\begin{minipage}[t]{\linewidth}%
{\Large\textbf{Chicken Taco Soup}} \label{chicken-taco-soup}\hfill\textit{Sue Dunn}\\
\textbf{Yield:} \textit{8 servings}\\
\noindent\begin{minipage}[t]{0.78\linewidth}%
\textbf{Ingredients}:\vspace{-3mm}
\begin{multicols}{2}
\begin{itemize}\setlength\itemsep{-1mm}
\item 2 lb chicken breasts
\item 2 (8 oz) pkg cream cheese
\item 1 oz ranch dressing
\item 3 Tbsp Chipotle seasoning
\item 2 (10 oz) cans diced tomatoes and green chiles
\item 4 cups chicken broth
\item grated cheddar cheese
\item cilantro
\item sour cream
\end{itemize}
\end{multicols}
\end{minipage}
\noindent\begin{minipage}[t]{0.18\linewidth}
\centering \strut\vspace*{-\baselineskip}\newline
\includegraphics[width=0.9\linewidth]{/home/tim/Documents/projects/recipes/img/none.jpg}\\
\end{minipage}\vspace{3mm}
\textbf{Directions}:
\vspace{-3mm}\begin{enumerate}\setlength\itemsep{-1mm}
\item Place all ingredients in crockpot. Cook on low for 6-8 hours.
\item Remove chicken and shred with two forks. Return to crockpot. Serve with cheese, cilantro, and sour cream.
\end{enumerate}
\end{minipage}\vspace{8mm}
\noindent\begin{minipage}[t]{\linewidth}%
{\Large\textbf{Chicken and Artichoke Bake}} \label{chicken-and-artichoke-bake}\hfill\textit{Sue Dunn}\\
\textit{``The leftovers make a delicious panini sandwich with rosemary olive oil bread and fontina cheese.''}\\
\noindent\begin{minipage}[t]{0.78\linewidth}%
\textbf{Ingredients}:\vspace{-3mm}
\begin{multicols}{2}
\begin{itemize}\setlength\itemsep{-1mm}
\item 1 (14 oz) can water-packed artichokes
\item 3/4 cup grated parmesan cheese
\item 3/4 cup light mayonnaise
\item dash of garlic powder
\item 4 chicken breasts, cut lengthwise in half
\end{itemize}
\end{multicols}
\end{minipage}
\noindent\begin{minipage}[t]{0.18\linewidth}
\centering \strut\vspace*{-\baselineskip}\newline
\includegraphics[width=0.9\linewidth]{/home/tim/Documents/projects/recipes/img/chicken-artichoke.jpeg}\\
\end{minipage}\vspace{3mm}
\textbf{Directions}:
\vspace{-3mm}\begin{enumerate}\setlength\itemsep{-1mm}
\item Drain artichokes. Squeeze all water out of each artichoke. Chop.
\item Combine artichokes, cheese, mayo, and garlic in a bowl. Mix thoroughly.
\item Place chicken in a greased 11x7 inch baking dish. Spread artichoke mixture over chicken. Bake uncovered at 375F for 30-35 minutes.
\end{enumerate}
\end{minipage}\vspace{8mm}
\noindent\begin{minipage}[t]{\linewidth}%
{\Large\textbf{Chile, Chicken, and Cheese Casserole}} \label{chile,-chicken,-and-cheese-casserole}\hfill\textit{Andrew Kulawiec}\\
\noindent\begin{minipage}[t]{0.78\linewidth}%
\textbf{Ingredients}:\vspace{-3mm}
\begin{multicols}{2}
\begin{itemize}\setlength\itemsep{-1mm}
\item 12-16 small flour tortillas
\item 16 oz sour cream
\item 2 cans condensed cream of chicken soup
\item 1 lb Cooked chicken
\item 2 small cans of chopped green chilies
\item 1 block extra sharp cheddar cheese
\item Monterey jack cheese (optional)
\end{itemize}
\end{multicols}
\end{minipage}
\noindent\begin{minipage}[t]{0.18\linewidth}
\centering \strut\vspace*{-\baselineskip}\newline
\includegraphics[width=0.9\linewidth]{/home/tim/Documents/projects/recipes/img/F6769DDE-B871-4966-97FC-4D0ABF315688.jpg}\\
\end{minipage}\vspace{3mm}
\textbf{Directions}:
\vspace{-3mm}\begin{enumerate}\setlength\itemsep{-1mm}
\item Layer Twice: tortillas to cover the bottom of a 9x13 pan, cream of chicken soup, chicken, sour cream, chiles, cheese.
\item Bake at 325F for 25 minutes. Then turn to 350F until golden brown.
\end{enumerate}
\end{minipage}\vspace{8mm}
\noindent\begin{minipage}[t]{\linewidth}%
{\Large\textbf{Cream Cheese Spinach Stuffed Chicken}} \label{cream-cheese-spinach-stuffed-chicken}\hfill\textit{Sue Dunn}\\
\textit{``Chicken breasts stuffed with a creamy spinach, parmesan, mozerella, and cream cheese filling and pan seared to perfection.''}\\
\textbf{Yield:} \textit{8 servings}\\
\noindent\begin{minipage}[t]{0.78\linewidth}%
\textbf{Ingredients}:\vspace{-3mm}
\begin{multicols}{2}
\begin{itemize}\setlength\itemsep{-1mm}
\item 8 (4 oz) chicken breasts
\item 2 tsp chili powder
\item 2 tsp Italian seasoning
\item 1 tsp black pepper
\item 1 tsp salt
\item 1 Tbsp olive oil
\item 4 cups spinach (one 10 oz pkg, thawed and drained)
\item 1 (8 oz) pkg cream cheese (room temp)
\item 1/2 cup Parmesan cheese
\item 1/2 cup mozarella cheese
\item 2 Tbsp minced garlic
\item 1/2 tsp pepper
\item salt to taste
\end{itemize}
\end{multicols}
\end{minipage}
\noindent\begin{minipage}[t]{0.18\linewidth}
\centering \strut\vspace*{-\baselineskip}\newline
\includegraphics[width=0.9\linewidth]{/home/tim/Documents/projects/recipes/img/cream-cheese-spinach-stuffed-chicken.jpeg}\\
\end{minipage}\vspace{3mm}
\textbf{Directions}:
\vspace{-3mm}\begin{enumerate}\setlength\itemsep{-1mm}
\item To make the filling: in a medium bowl, combine the spinach, cream cheese, parmesan cheese, mozzarella cheese, garlic, salt, and pepper.
\item To butterfly the chicken breasts: lay them flat on a sturdy surface. Place one hand on top to hold it in place and slice 3/4 of the way through the chicken breast.
\item To stuff the chicken: season the outside of the chicken with chili powder, Italian seasoning, salt, and pepper. Spoon 1/4 of the cheese mixture into the middle of the cut chicken breasts and fold the chicken so the cream cheese is sealed inside. Use toothpicks if necessary.
\item To cook: Heat a non-stick skillet on medium-high and add olive oil. Cook the chicken, covering the pan with a lid, for about 9-10 minutes per side or until the chicken is cooked through.
\end{enumerate}
\end{minipage}\vspace{8mm}
\noindent\begin{minipage}[t]{\linewidth}%
{\Large\textbf{Creamy Herb Chicken}} \label{creamy-herb-chicken}\hfill\textit{Sue Dunn}\\
\noindent\begin{minipage}[t]{0.78\linewidth}%
\textbf{Ingredients}:\vspace{-3mm}
\begin{multicols}{2}
\begin{itemize}\setlength\itemsep{-1mm}
\item 4 chicken breasts (pounded 1/2 inch thin)
\item 2 Tbsp onion powder
\item 2 Tbsp garlic powder
\item 1 tsp parsley, freshly chopped
\item 1/2 tsp thyme
\item 1/2 tsp rosemary
\item salt and pepper
\item 4 cloves garlic
\item 1 tsp parsley, freshly chopped
\item 1/2 tsp thyme
\item 1/2 tsp rosemary
\item 1 cup milk
\item salt and pepper
\item 1 tsp cornstarch
\item 1 Tbsp water
\end{itemize}
\end{multicols}
\end{minipage}
\noindent\begin{minipage}[t]{0.18\linewidth}
\centering \strut\vspace*{-\baselineskip}\newline
\includegraphics[width=0.9\linewidth]{/home/tim/Documents/projects/recipes/img/none.jpg}\\
\end{minipage}\vspace{3mm}
\textbf{Directions}:
\vspace{-3mm}\begin{enumerate}\setlength\itemsep{-1mm}
\item Coat chicken breasts with onion powder, garlic powder, and herbs. Season generously with salt and pepper.
\item Heat 1 Tbsp of oil in a large pan over medium-high heat and cook chicken breasts. Set chicken aside.
\item To the same pan, heat another 2 tsp of olive oil, and saute garlic with parsley, thyme, and rosemary for a minute until fragrant. Stir in milk; season with salt and pepper to taste.
\item Bring to a boil. Mix together the cornstarch and water before adding to the pan, stirring quickly, until sauce has thickened slightly. Reduce heat and simmer gently for another minute to thicken sauce. Return chicken to the skillet, and sprinkle with extra herbs if desired. Serve immediately.
\end{enumerate}
\end{minipage}\vspace{8mm}
\noindent\begin{minipage}[t]{\linewidth}%
{\Large\textbf{Creamy Pesto Chicken}} \label{creamy-pesto-chicken}\hfill\textit{Sue Dunn}\\
\textbf{Yield:} \textit{4 servings}\\
\noindent\begin{minipage}[t]{0.78\linewidth}%
\textbf{Ingredients}:\vspace{-3mm}
\begin{multicols}{2}
\begin{itemize}\setlength\itemsep{-1mm}
\item 1 lb chicken breasts
\item salt and pepper
\item 2 Tbsp pesto
\item 2 Tbsp olive oil
\item 2 cups cherry tomatoes, halved
\item 1 Tbsp all-purpose flour
\item 2 garlic cloves, minced
\item 1/2 Tbsp pesto
\item 1 cup evaporated milk
\item freshly shaved Parmesan (optional, for garnish)
\end{itemize}
\end{multicols}
\end{minipage}
\noindent\begin{minipage}[t]{0.18\linewidth}
\centering \strut\vspace*{-\baselineskip}\newline
\includegraphics[width=0.9\linewidth]{/home/tim/Documents/projects/recipes/img/creamy-pesto-chicken.jpg}\\
\end{minipage}\vspace{3mm}
\textbf{Directions}:
\vspace{-3mm}\begin{enumerate}\setlength\itemsep{-1mm}
\item Season chicken with salt and pepper, and rub chicken with pesto. Heat olive oil in a large skillet. Add chicken to skillet and cook for 5 minutes on each side, or until thoroughly cooked.
\item In the meantime, prepare the sauce by combining flour, garlic, pesto, evaporated milk, salt, and pepper in a bowl; whisk until thoroughly combined and set aside. Remove chicken from skillet.
\item Return skillet to heat and stir in the tomatoes. If pan is dry, add more olive oil. Cook the tomatoes over medium-high heat for about a minute, or until beginning to soften. Add the prepared sauce to the skillet and bring to a boil.
\item Place the chicken back in the skillet and cook for 2-3 minutes, or until heated through. Remove from heat and garnish with cheese. Serve with pasta or rice.
\end{enumerate}
\end{minipage}\vspace{8mm}
\noindent\begin{minipage}[t]{\linewidth}%
{\Large\textbf{Crockpot Shredded Chicken}} \label{crockpot-shredded-chicken}\hfill\textit{Sue Dunn}\\
\textbf{Yield:} \textit{7 cups}\\
\noindent\begin{minipage}[t]{0.78\linewidth}%
\textbf{Ingredients}:\vspace{-3mm}
\begin{multicols}{2}
\begin{itemize}\setlength\itemsep{-1mm}
\item 3 lb chicken breasts
\item 3/4 cup low-sodium chicken broth
\item 1 tsp oregano
\item 1/2 tsp garlic powder
\item 1/2 tsp salt
\item 1/4 tsp black pepper
\end{itemize}
\end{multicols}
\end{minipage}
\noindent\begin{minipage}[t]{0.18\linewidth}
\centering \strut\vspace*{-\baselineskip}\newline
\includegraphics[width=0.9\linewidth]{/home/tim/Documents/projects/recipes/img/none.jpg}\\
\end{minipage}\vspace{3mm}
\textbf{Directions}:
\vspace{-3mm}\begin{enumerate}\setlength\itemsep{-1mm}
\item Place all ingredients in crockpot. Cover and cook on high for 2-3 hours or on low for 4-6 hours, or until chicken is cooked through and shreds easily with a fork.
\item Remove the chicken from the slow cooker, leaving the broth in the pot. Shred. Stir in 1/2 cup of broth from the pot from flavor and moisture.
\item Mix in 1/4 cup of the sauce with the cooked chicken, and a 1/4 cup of the cheese. Sprinkle with a little salt. Set aside.
\end{enumerate}
\end{minipage}\vspace{8mm}
\noindent\begin{minipage}[t]{\linewidth}%
{\Large\textbf{Lemon Chicken With White Wine}} \label{lemon-chicken-with-white-wine}\hfill\textit{Sue Dunn}\\
\textbf{Yield:} \textit{serves 6}\\
\noindent\begin{minipage}[t]{0.78\linewidth}%
\textbf{Ingredients}:\vspace{-3mm}
\begin{multicols}{2}
\begin{itemize}\setlength\itemsep{-1mm}
\item 3/4 cup flour
\item 1 1/2 tsp salt
\item 1 1/2 tsp ground black pepper
\item 3/4 tsp paprika
\item 3/4 tsp dried parsley
\item 3/8 cup butter, unsalted
\item 3 Tbsp olive oil
\item 6 chicken breasts, pounded thin
\item 1 1/2 lemon juice and zest
\item 1 1/2 lemon, sliced
\item 3/8 cup white wine
\item 1 Tbsp honey (optional)
\end{itemize}
\end{multicols}
\end{minipage}
\noindent\begin{minipage}[t]{0.18\linewidth}
\centering \strut\vspace*{-\baselineskip}\newline
\includegraphics[width=0.9\linewidth]{/home/tim/Documents/projects/recipes/img/none.jpg}\\
\end{minipage}\vspace{3mm}
\textbf{Directions}:
\vspace{-3mm}\begin{enumerate}\setlength\itemsep{-1mm}
\item Combine flour, salt, pepper, paprika, and parsley in a shallow bowl. Set aside. Heat oil and butter over medium-high heat. Coat both sides of each chicken breast in the flour mixture and add to the hot skillet. Add the lemon juice, white wine, lemon zest, and rosemary to the skillet.
\item Cook chicken about 3 minutes on each side until no longer pink in center, then remove from skillet and set aside.
\item Allow the sauce in the skillet to reduce by half. Add the lemon slices after reduced until they're softened.
\item Add the chicken back into the skillet to serve. If you want a sweeter sauce, add 1 Tbsp honey. Serve over rice or pasta.
\end{enumerate}
\end{minipage}\vspace{8mm}
\noindent\begin{minipage}[t]{\linewidth}%
{\Large\textbf{Onion Cream Chicken}} \label{onion-cream-chicken}\hfill\textit{Sue Dunn}\\
\noindent\begin{minipage}[t]{0.78\linewidth}%
\textbf{Ingredients}:\vspace{-3mm}
\begin{multicols}{2}
\begin{itemize}\setlength\itemsep{-1mm}
\item 4 chicken breasts
\item salt and pepper
\item 1/2 tsp garlic powder
\item 1 Tbsp olive oil
\item 1/2 cup onions, chopped
\item 2 Tbsp butter
\item 1/2 tsp thyme, freshly chopped
\item pinch of red pepper flakes
\item 1 tsp sugar
\item 2 tsp balsamic vinegar
\item 1/2 cup chicken broth
\item 1/2 cup heavy cream
\end{itemize}
\end{multicols}
\end{minipage}
\noindent\begin{minipage}[t]{0.18\linewidth}
\centering \strut\vspace*{-\baselineskip}\newline
\includegraphics[width=0.9\linewidth]{/home/tim/Documents/projects/recipes/img/none.jpg}\\
\end{minipage}\vspace{3mm}
\textbf{Directions}:
\vspace{-3mm}\begin{enumerate}\setlength\itemsep{-1mm}
\item Cook the chicken: season both sides of the chicken breast with salt, pepper, and garlic powder. Heat the oil in a large skillet over medium-high heat and cook the chicken fully.
\item Caramelize the onions: add the butter to the skillet along with the onions. Push the onions around the pan so they pick up all the flavor bits left behind by the chicken. Lower the heat to medium-low and allow the onions to cook for 12-15 minutes, stirring them as needed to prevent from sticking. The onions are done when they soften completely and darken in color. Add the thyme, red pepper flakes, sugar, and balsamic vinegar. Let the vinegar cook out for a few minutes.
\item Make the sauce: Slowly pour in the chicken broth in a steady stream while you whisk; this will help deglaze the pan. Kick up the heat to high and let the sauce reduce for 2-3 minutes or until thicker. Once the sauce reduces, lower the heat again and add in the cream. You don't want the cream to start boiling immediately or the sauce will split. Let the sauce come to a gentle simmer.
\end{enumerate}
\end{minipage}\vspace{8mm}
\noindent\begin{minipage}[t]{\linewidth}%
{\Large\textbf{Seared Chicken With Avocado}} \label{seared-chicken-with-avocado}\hfill\textit{allrecipes.com}\\
\noindent\begin{minipage}[t]{0.78\linewidth}%
\textbf{Ingredients}:\vspace{-3mm}
\begin{multicols}{2}
\begin{itemize}\setlength\itemsep{-1mm}
\item 1 1/2 tsp blackened seasoning
\item 4 (4 oz) skinless, boneless chicken breast halves
\item 1 tsp olive oil
\item 1 diced, peeled avocado
\item 2 Tbsp chopped fresh cilantro
\item 1 jalapeno pepper, seeded and finely chopped
\item 2 Tbsp freh lime juice (anout 1 lime)
\item 1/4 tsp salt
\item 1 lime, cut into quarters
\end{itemize}
\end{multicols}
\end{minipage}
\noindent\begin{minipage}[t]{0.18\linewidth}
\centering \strut\vspace*{-\baselineskip}\newline
\includegraphics[width=0.9\linewidth]{/home/tim/Documents/projects/recipes/img/F487A08A-06D4-4929-8018-9121063658C4.jpg}\\
\end{minipage}\vspace{3mm}
\textbf{Directions}:
\vspace{-3mm}\begin{enumerate}\setlength\itemsep{-1mm}
\item Sprinkle seasoning on both sides of chicken
\item Heat oil in a large nonstick skillet over high heat. Add chicken, smooth side down, to pan; cook 1 minute or until seared.
\item Reduce hate to medium; cook 3 minutes on each side or until lightly browned.
\item Combine avocado, cilantro, pepper, lime juice and salt.
\item Squeeze one-quarter lime over each piece of chicken before serving. Serve with avocado mixture.
\end{enumerate}
\end{minipage}\vspace{8mm}
\noindent\begin{minipage}[t]{\linewidth}%
{\Large\textbf{Sour Cream Salsa Chicken}} \label{sour-cream-salsa-chicken}\hfill\textit{Sue Dunn}\\
\noindent\begin{minipage}[t]{0.78\linewidth}%
\textbf{Ingredients}:\vspace{-3mm}
\begin{multicols}{2}
\begin{itemize}\setlength\itemsep{-1mm}
\item 5 boneless chicken breasts
\item 1 package taco seasoning mix
\item 1 cup salsa
\item 2 Tbsp corn starch
\item 1/4 cup sour cream
\end{itemize}
\end{multicols}
\end{minipage}
\noindent\begin{minipage}[t]{0.18\linewidth}
\centering \strut\vspace*{-\baselineskip}\newline
\includegraphics[width=0.9\linewidth]{/home/tim/Documents/projects/recipes/img/sour-cream-salsa-chicken.jpeg}\\
\end{minipage}\vspace{3mm}
\textbf{Directions}:
\vspace{-3mm}\begin{enumerate}\setlength\itemsep{-1mm}
\item Spray crockpot with cooking spray and line with chicken. Sprinkle taco mix on top and top with salsa, then cook on low for 6-8 hours.
\item Remove chicken from pot. Mix corn starch with a little water and add to salsa. Stir in sour cream.
\item Return chicken to pot for 30-45 more minutes or until hot again.
\end{enumerate}
\end{minipage}\vspace{8mm}
\noindent\begin{minipage}[t]{\linewidth}%
{\Large\textbf{Coconut Chicken Curry}} \label{coconut-chicken-curry}\hfill\textit{Cameron Behar}\\
\noindent\begin{minipage}[t]{0.78\linewidth}%
\textbf{Ingredients}:\vspace{-3mm}
\begin{multicols}{2}
\begin{itemize}\setlength\itemsep{-1mm}
\item 6 cloves garlic
\item chili powder
\item sriracha (optional)
\item 2 cans coconut milk
\item cilantro
\item 1/2 onion
\item 1 red bell pepper
\item 1 orange bell pepper
\item 1 zucchini
\item 3 chicken breasts
\item lime juice
\item 1 (4 oz) jar curry paste
\item 2 Tbsp tomato paste
\end{itemize}
\end{multicols}
\end{minipage}
\noindent\begin{minipage}[t]{0.18\linewidth}
\centering \strut\vspace*{-\baselineskip}\newline
\includegraphics[width=0.9\linewidth]{/home/tim/Documents/projects/recipes/img/coconut-chicken-curry.jpg}\\
\end{minipage}\vspace{3mm}
\textbf{Directions}:
\vspace{-3mm}\begin{enumerate}\setlength\itemsep{-1mm}
\item TBD
\end{enumerate}
\end{minipage}\vspace{8mm}

{\newpage \LARGE \textbf{Cookies}} \label{cookies}\vspace{4mm}\\
\noindent\begin{minipage}[t]{\linewidth}%
{\Large\textbf{Almond Macaroons}} \label{almond-macaroons}\hfill\textit{}\\
\noindent\begin{minipage}[t]{0.78\linewidth}%
\textbf{Ingredients}:\vspace{-3mm}
\begin{multicols}{2}
\begin{itemize}\setlength\itemsep{-1mm}
\item 10 oz blanched whole almonds (about 2 full cups)
\item 2 3/4 Cups granulated sugar
\item 3 Large egg whites
\item 1/2 tsp pure almond extract
\item 6 oz good quality bittersweet chocolate, chopped (optional)
\end{itemize}
\end{multicols}
\end{minipage}
\noindent\begin{minipage}[t]{0.18\linewidth}
\centering \strut\vspace*{-\baselineskip}\newline
\includegraphics[width=0.9\linewidth]{/home/tim/Documents/projects/recipes/img/DD993399-6C6B-4C84-ADE9-9D35E6F5E3F3.jpg}\\
\end{minipage}\vspace{3mm}
\textbf{Directions}:
\vspace{-3mm}\begin{enumerate}\setlength\itemsep{-1mm}
\item Preheat oven to 350 degrees. Line one or two cookie sheets with parchment paper.
\item Combine the almonds and 1/4 cup of the sugar in a food processor and process until the almonds are finely ground. Add the egg whites and almond extract and process until blended.
\item Add the remaining 1 cup of sugar and process until thoroughly combined, about 15 seconds, or until the dough is a thick, sticky paste.
\item Drop the dough by level tablespoonfuls, arranging about 2 inches apart on the prepared sheet(s). Using a pastry brush lightly moistened with water, brush the tops and sides of the macaroons, gently pressing down on them to form smooth rounds about 1/2 inch thick and 1 3/4 inches in diameter.
\item Bake for about 15 minutes or until the macaroons are pale golden. They should feel crisp on the outside but still soft inside. (If using two cookie sheets, rotate them from top to bottom and front to back about halfway through baking.) Remove the sheets from the oven and slide the parchment onto racks. Cool for about 5 minutes, then use a thin metal spatula to remove the macaroons from the paper. Place on rack to cool.
\item For chocolate-dipped macaroons, melt the chocolate in a metal or glass bowl set over a pan of barely simmering water, stirring frequently, until fully melted. Alternatively, melt it in a microwave-safe bowl in microwave, using 20-to-30-second bursts at medium power, stirring well after each interval. Line baking sheet with wax paper. With a silicone pastry brush, brush the melted chocolate on half the cookie, both top side and bottom, in a semicircle. Let the excess drip off or gently scrape it off the bottom using the brush. Place the macaroons on the wax paper and let stand until the chocolate is completely set.
\item Store macaroons in an airtight container, layered between sheets of wax paper, for up to five days at room temperature. Macaroons without chocolate can be frozen for up to two months.
\end{enumerate}
\end{minipage}\vspace{8mm}
\noindent\begin{minipage}[t]{\linewidth}%
{\Large\textbf{Bunco Bites}} \label{bunco-bites}\hfill\textit{Mary Reynolds}\\
\noindent\begin{minipage}[t]{0.78\linewidth}%
\textbf{Ingredients}:\vspace{-3mm}
\begin{multicols}{2}
\begin{itemize}\setlength\itemsep{-1mm}
\item mini square party pretzels
\item M and Ms
\item white chocolate wafers
\end{itemize}
\end{multicols}
\end{minipage}
\noindent\begin{minipage}[t]{0.18\linewidth}
\centering \strut\vspace*{-\baselineskip}\newline
\includegraphics[width=0.9\linewidth]{/home/tim/Documents/projects/recipes/img/E80AF034-6CDF-4993-A223-474B6DFBA2BA.jpg}\\
\end{minipage}\vspace{3mm}
\textbf{Directions}:
\vspace{-3mm}\begin{enumerate}\setlength\itemsep{-1mm}
\item Preheat oven to 225F. Place single layer of pretzels on cookie sheet. Place one wafer on center of each pretzel. Warm in oven for 3-5 minutes to soften wafer. Remove from oven and place one M and M in center of each wafer. Cool in fridge until wafer hardens.
\end{enumerate}
\end{minipage}\vspace{8mm}
\noindent\begin{minipage}[t]{\linewidth}%
{\Large\textbf{Chocolate Chip Cookies}} \label{chocolate-chip-cookies}\hfill\textit{Sue Dunn}\\
\textit{``from the ABC cookbook''}\\
\textbf{Yield:} \textit{48 cookies}\\
\noindent\begin{minipage}[t]{0.78\linewidth}%
\textbf{Ingredients}:\vspace{-3mm}
\begin{multicols}{2}
\begin{itemize}\setlength\itemsep{-1mm}
\item 3/4 cup granulated sugar
\item 3/4 cup packed brown sugar
\item 1 cup melted butter
\item 1 egg
\item 2 1/4 cups all-purpose flour
\item 1 tsp baking soda
\item 1/2 tsp salt
\item 1 cup semisweet chocolate chips
\item 1 cup white chocolate chips
\end{itemize}
\end{multicols}
\end{minipage}
\noindent\begin{minipage}[t]{0.18\linewidth}
\centering \strut\vspace*{-\baselineskip}\newline
\includegraphics[width=0.9\linewidth]{/home/tim/Documents/projects/recipes/img/26E62107-6551-417E-8826-98D0F0756BBE.jpg}\\
\end{minipage}\vspace{3mm}
\textbf{Directions}:
\vspace{-3mm}\begin{enumerate}\setlength\itemsep{-1mm}
\item Heat the oven to 375F. 
\item Mix both sugars, butter, and egg in a large bowl. Stir in flour, baking soda and salt (dough will be stiff). Stir in chocolate chips.
\item Drop dough by rounded tablespoonfuls about 3 inches apart onto an ungreased cookie sheet. 
\item Bake until light brown, 8-10 minutes (centers will be soft). Let cookies cool slightly, then remove from cookie sheet with spatula.
\end{enumerate}
\end{minipage}\vspace{8mm}
\noindent\begin{minipage}[t]{\linewidth}%
{\Large\textbf{Christmas Crackle}} \label{christmas-crackle}\hfill\textit{Aimee Rinere}\\
\noindent\begin{minipage}[t]{0.78\linewidth}%
\textbf{Ingredients}:\vspace{-3mm}
\begin{multicols}{2}
\begin{itemize}\setlength\itemsep{-1mm}
\item 14 cups microwave popcorn, popped (about 2 bags)
\item 3 cups Rice Krispies cereal
\item 2 cups mixed salted nuts
\item 1 lb white chocolate
\item 3 Tbsp peanut butter
\end{itemize}
\end{multicols}
\end{minipage}
\noindent\begin{minipage}[t]{0.18\linewidth}
\centering \strut\vspace*{-\baselineskip}\newline
\includegraphics[width=0.9\linewidth]{/home/tim/Documents/projects/recipes/img/F6310104-FB28-461F-A8A3-24990DCB4E48.jpg}\\
\end{minipage}\vspace{3mm}
\textbf{Directions}:
\vspace{-3mm}\begin{enumerate}\setlength\itemsep{-1mm}
\item Mix popcorn, Rice Krispies and nuts in a large bowl.
\item In the microwave, melt the white chocolate and peanut butter. Start with 1 minute then stir and repeat until melted and smooth. Pour over the popcorn and mix well. Spread onto wax paper and let set for about 2 hours. Break apart into pieces.
\end{enumerate}
\end{minipage}\vspace{8mm}
\noindent\begin{minipage}[t]{\linewidth}%
{\Large\textbf{Cookies and Cream Oreo Bark}} \label{cookies-and-cream-oreo-bark}\hfill\textit{Brian Dunn}\\
\noindent\begin{minipage}[t]{0.78\linewidth}%
\textbf{Ingredients}:\vspace{-3mm}
\begin{multicols}{2}
\begin{itemize}\setlength\itemsep{-1mm}
\item 10 oz Ghirardelli white chocolate chips
\item 15 regular size oreos, plus 3 more for topping
\end{itemize}
\end{multicols}
\end{minipage}
\noindent\begin{minipage}[t]{0.18\linewidth}
\centering \strut\vspace*{-\baselineskip}\newline
\includegraphics[width=0.9\linewidth]{/home/tim/Documents/projects/recipes/img/D023DA6E-65E3-4F3C-8CF0-B47A6E79E5D2.jpg}\\
\end{minipage}\vspace{3mm}
\textbf{Directions}:
\vspace{-3mm}\begin{enumerate}\setlength\itemsep{-1mm}
\item Preparation: Line an 8x8 pan with enough parchment or wax paper for a 1 inch overhang on each side.
\item Place chocolate in a double boiler over low heat and stir continuously, until chocolate is completely melted. Transfer chocolate to a heat proof bowl and cool for 5 minutes. Add chopped Oreos and stir to combine. Pour mixture into pan. Use a spatula to smooth out top.
\item Finely chop remaining Oreos and sprinkle on top. Chill for about 10 minutes until chocolate becomes solid.
\item Lift whole bark out of pan by holding onto parchment or wax overhang. Split bark into pieces with a fork.
\end{enumerate}
\end{minipage}\vspace{8mm}
\noindent\begin{minipage}[t]{\linewidth}%
{\Large\textbf{Mint Meringues}} \label{mint-meringues}\hfill\textit{Sue Dunn}\\
\noindent\begin{minipage}[t]{0.78\linewidth}%
\textbf{Ingredients}:\vspace{-3mm}
\begin{multicols}{2}
\begin{itemize}\setlength\itemsep{-1mm}
\item 2 egg whites
\item 1/8 tsp salt
\item 1/8 tsp cream of tartar
\item 1/8 tsp peppermint extract
\item 6 drops green food coloring, optional
\item 1/2 cup sugar
\item 1/3 cup miniature semisweet chocolate chips
\end{itemize}
\end{multicols}
\end{minipage}
\noindent\begin{minipage}[t]{0.18\linewidth}
\centering \strut\vspace*{-\baselineskip}\newline
\includegraphics[width=0.9\linewidth]{/home/tim/Documents/projects/recipes/img/2F3BD676-F090-43F5-A86A-CA51F2F6D1A4.jpg}\\
\end{minipage}\vspace{3mm}
\textbf{Directions}:
\vspace{-3mm}\begin{enumerate}\setlength\itemsep{-1mm}
\item In a small bowl, beat the egg whites, salt, cream of tartar, extract and food coloring if desired on medium speed until soft peaks form. Gradually add sugar, 1 tablespoon at a time, beating on high until stiff glossy peaks form and sugar is dissolved, about 6 minutes. Gently fold in chocolate chips.
\item Drop by rounded teaspoonfuls 2 in. apart onto parchment paper-lined baking sheets. Bake at 250 for 40-45 minutes or until firm to the touch. Turn oven off; leave meringues in oven for 1-1/2 hours. Remove to wire racks. Store in an airtight container. Yield: 32 cookies.
\end{enumerate}
\end{minipage}\vspace{8mm}
\noindent\begin{minipage}[t]{\linewidth}%
{\Large\textbf{Monster Cookies}} \label{monster-cookies}\hfill\textit{Carolyn Benjamin}\\
\noindent\begin{minipage}[t]{0.78\linewidth}%
\textbf{Ingredients}:\vspace{-3mm}
\begin{multicols}{2}
\begin{itemize}\setlength\itemsep{-1mm}
\item 3 eggs
\item 1 cup brown sugar
\item 3/4 tsp vanilla
\item 1 cup granulated sugar
\item 2 tsp baking soda
\item 1/2 cup butter
\item 2 cups peanut butter
\item 4 1/2 cups oatmeal
\item 1/2 cup chocolate chips (chocolate chunks)
\item 1/2 cup M and Ms
\item 1/8 cup chopped nuts
\end{itemize}
\end{multicols}
\end{minipage}
\noindent\begin{minipage}[t]{0.18\linewidth}
\centering \strut\vspace*{-\baselineskip}\newline
\includegraphics[width=0.9\linewidth]{/home/tim/Documents/projects/recipes/img/338BA54A-1A96-4D53-B39E-7AD0183F00F2.jpg}\\
\end{minipage}\vspace{3mm}
\textbf{Directions}:
\vspace{-3mm}\begin{enumerate}\setlength\itemsep{-1mm}
\item Combine all ingredients. Use an ice cream scoop to form balls of dough, then gently flatten dough on cookie sheet. Bake at 350 for 10-12 minutes.
\end{enumerate}
\end{minipage}\vspace{8mm}
\noindent\begin{minipage}[t]{\linewidth}%
{\Large\textbf{Oatmeal Peanut Butter Chocolate Chip Cookies}} \label{oatmeal-peanut-butter-chocolate-chip-cookies}\hfill\textit{Sue Dunn}\\
\noindent\begin{minipage}[t]{0.78\linewidth}%
\textbf{Ingredients}:\vspace{-3mm}
\begin{multicols}{2}
\begin{itemize}\setlength\itemsep{-1mm}
\item 2 cups all-purpose flour
\item 2 tsp baking powder
\item 1/2 tsp salt
\item 1 cup (2 sticks) butter, softened
\item 3/4 cup peanut butter
\item 1 cup granulated sugar
\item 1 cup light brown sugar
\item 1 tsp vanilla extract
\item 2 eggs
\item 2 cups (12 oz pkg) chocolate chips
\item 1/4 cup dark chocolate morsels
\item 1 1/2 cup quick oats
\end{itemize}
\end{multicols}
\end{minipage}
\noindent\begin{minipage}[t]{0.18\linewidth}
\centering \strut\vspace*{-\baselineskip}\newline
\includegraphics[width=0.9\linewidth]{/home/tim/Documents/projects/recipes/img/none.jpg}\\
\end{minipage}\vspace{3mm}
\textbf{Directions}:
\vspace{-3mm}\begin{enumerate}\setlength\itemsep{-1mm}
\item Preheat oven to 375F. Combine flour, baking powder, and salt in small bowl.
\item Beat butter, peanut butter, granulated sugar, brown sugar, and vanilla extract in large mixer bowl until creamy. Add eggs, one at a time, beating well after each addition. Gradually beat in flour mixture. Stir in chocolate, morsels and oats. Drop by tablespoon onto baking sheets.
\item Bake for 10-12 minutes until light golden brown. Cool on baking sheets for 2 minutes.
\end{enumerate}
\end{minipage}\vspace{8mm}
\noindent\begin{minipage}[t]{\linewidth}%
{\Large\textbf{Peanut Butter Balls}} \label{peanut-butter-balls}\hfill\textit{Sue Dunn}\\
\noindent\begin{minipage}[t]{0.78\linewidth}%
\textbf{Ingredients}:\vspace{-3mm}
\begin{multicols}{2}
\begin{itemize}\setlength\itemsep{-1mm}
\item 2 cups creamy peanut butter
\item 3 cups Rice Krispies cereal
\item 4 cups powdered sugar
\item 1/4 tsp vanilla extract
\item 1 cup chocolate chips
\item 1/2 cup butter
\end{itemize}
\end{multicols}
\end{minipage}
\noindent\begin{minipage}[t]{0.18\linewidth}
\centering \strut\vspace*{-\baselineskip}\newline
\includegraphics[width=0.9\linewidth]{/home/tim/Documents/projects/recipes/img/86C864D7-F478-4755-A203-63FC3FB9B6ED.jpg}\\
\end{minipage}\vspace{3mm}
\textbf{Directions}:
\vspace{-3mm}\begin{enumerate}\setlength\itemsep{-1mm}
\item In a medium sized bowl, mix peanut butter, butter, sugar and rice krispies. Blend well until mixture forms a dough. Roll into 1-inch balls. Dip the balls into the melted chocolate until well coated. Place onto a cookie sheet lined with wax paper. Refrigerate for 30 minutes.
\end{enumerate}
\end{minipage}\vspace{8mm}
\noindent\begin{minipage}[t]{\linewidth}%
{\Large\textbf{Peanut Butter Blossoms}} \label{peanut-butter-blossoms}\hfill\textit{Sue Dunn}\\
\noindent\begin{minipage}[t]{0.78\linewidth}%
\textbf{Ingredients}:\vspace{-3mm}
\begin{multicols}{2}
\begin{itemize}\setlength\itemsep{-1mm}
\item 48 HERSHEY'S KISSES Brand Milk Chocolates
\item 1/2 Cup shortening
\item 3/4 Cup REESE'S Creamy Peanut Butter
\item 1/3 Cup granulated sugar
\item 1/3 Cup packed light brown sugar
\item 1 egg
\item 2 Tbsp milk
\item 1 tsp vanilla extract
\item 1 1/2 Cups all - purpose flour
\item 1 tsp baking soda
\item 1/2 tsp salt
\item Additional granulated sugar
\end{itemize}
\end{multicols}
\end{minipage}
\noindent\begin{minipage}[t]{0.18\linewidth}
\centering \strut\vspace*{-\baselineskip}\newline
\includegraphics[width=0.9\linewidth]{/home/tim/Documents/projects/recipes/img/CAC98C1D-BE6E-490D-9E92-FEB5A2D75A95.jpg}\\
\end{minipage}\vspace{3mm}
\textbf{Directions}:
\vspace{-3mm}\begin{enumerate}\setlength\itemsep{-1mm}
\item 1. Heat oven to 375F. Remove wrappers from chocolates.
\item 2. Beat shortening and peanut butter in large bowl until well blended. Add 1/3 cup granulated sugar and brown sugar; beat until fluffy. Add egg, milk and vanilla; beat well. Stir together flour, baking soda and salt; gradually beat into peanut butter mixture.
\item 3. Shape dough into 1-inch balls. Roll in granulated sugar; place on ungreased cookie sheet.
\item 4. Bake 8 to 10 minutes or until lightly browned. Immediately press a chocolate into center of each
\item cookie; cookie will crack around edges. Remove from cookie sheet to wire rack. Cool completely.
\item About 4 dozen cookies.
\item Nutritional Information per serving (1 cookie):
\item Calories: 90, Total Fat: 6g, Saturated Fat: 2g, Cholesterol: 5mg, Sodium: 75mg,
\end{enumerate}
\end{minipage}\vspace{8mm}
\noindent\begin{minipage}[t]{\linewidth}%
{\Large\textbf{Peanut Butter Cookies}} \label{peanut-butter-cookies}\hfill\textit{Grandma Claire Dunn}\\
\noindent\begin{minipage}[t]{0.78\linewidth}%
\textbf{Ingredients}:\vspace{-3mm}
\begin{multicols}{2}
\begin{itemize}\setlength\itemsep{-1mm}
\item 1/2 cup butter
\item 1 cup peanut butter
\item 3/4 cup brown sugar
\item 3/4 cup white sugar
\item 2 eggs
\item 2 cups sifted flour
\item 2 tsp baking soda
\end{itemize}
\end{multicols}
\end{minipage}
\noindent\begin{minipage}[t]{0.18\linewidth}
\centering \strut\vspace*{-\baselineskip}\newline
\includegraphics[width=0.9\linewidth]{/home/tim/Documents/projects/recipes/img/8FDB8D74-041F-4A11-8008-C299FAD4835A.jpg}\\
\end{minipage}\vspace{3mm}
\textbf{Directions}:
\vspace{-3mm}\begin{enumerate}\setlength\itemsep{-1mm}
\item Preheat oven to 375F.
\item Combine all ingredients and line on ungreased cookie sheets.
\item Bake for 10 minutes.
\end{enumerate}
\end{minipage}\vspace{8mm}
\noindent\begin{minipage}[t]{\linewidth}%
{\Large\textbf{Peppermint Bark}} \label{peppermint-bark}\hfill\textit{}\\
\noindent\begin{minipage}[t]{0.78\linewidth}%
\textbf{Ingredients}:\vspace{-3mm}
\begin{multicols}{2}
\begin{itemize}\setlength\itemsep{-1mm}
\item 1 cup (crushed) candy canes
\item 2 lb white chocolate
\item peppermint flavoring (optional)
\end{itemize}
\end{multicols}
\end{minipage}
\noindent\begin{minipage}[t]{0.18\linewidth}
\centering \strut\vspace*{-\baselineskip}\newline
\includegraphics[width=0.9\linewidth]{/home/tim/Documents/projects/recipes/img/43658F4A-0D1D-407D-9879-F76AD6FB686B.jpg}\\
\end{minipage}\vspace{3mm}
\textbf{Directions}:
\vspace{-3mm}\begin{enumerate}\setlength\itemsep{-1mm}
\item Place candy canes in a plastic bag and hammer into 1/4-inch chunks or smaller. Melt the chocolate in a double boiler. Combine candy cane chunks with chocolate (add peppermint flavoring at this point if desired.) Pour mixture onto a cookie sheet layered with parchment or waxed paper and place in the refrigerator for 45 minutes or until firm. Remove from cookie sheet and break into pieces (like peanut brittle.)
\end{enumerate}
\end{minipage}\vspace{8mm}
\noindent\begin{minipage}[t]{\linewidth}%
{\Large\textbf{Russian Tea Cakes}} \label{russian-tea-cakes}\hfill\textit{Sue Dunn}\\
\noindent\begin{minipage}[t]{0.78\linewidth}%
\textbf{Ingredients}:\vspace{-3mm}
\begin{multicols}{2}
\begin{itemize}\setlength\itemsep{-1mm}
\item 1 cup butter, softened
\item 1/2 cup powdered sugar
\item 1 teaspoon vanilla
\item 2 1/4 cups all-purpose flour
\item 3/4 cup finely chopped nuts
\item 1/4 teaspoon salt
\item powdered sugar
\end{itemize}
\end{multicols}
\end{minipage}
\noindent\begin{minipage}[t]{0.18\linewidth}
\centering \strut\vspace*{-\baselineskip}\newline
\includegraphics[width=0.9\linewidth]{/home/tim/Documents/projects/recipes/img/05B2E5B1-94ED-4006-BB76-8CE6F0BE54F0.jpg}\\
\end{minipage}\vspace{3mm}
\textbf{Directions}:
\vspace{-3mm}\begin{enumerate}\setlength\itemsep{-1mm}
\item Heat oven to 375 F.
\item Mix butter, 1/2 cup powdered sugar and the vanilla in large bowl. Stir in flour, nuts and salt until dough holds together.
\item Shape dough into 1-inch balls. Place about 1 inch apart on ungreased cookie sheet.
\item Bake 10 to 12 minutes or until set but not brown. Remove from cookie sheet. Cool slightly on wire rack.
\item Roll warm cookies in powdered sugar; cool on wire rack. Roll in powdered sugar again.
\end{enumerate}
\end{minipage}\vspace{8mm}
\noindent\begin{minipage}[t]{\linewidth}%
{\Large\textbf{Sugar Cookies}} \label{sugar-cookies}\hfill\textit{Sue Dunn}\\
\noindent\begin{minipage}[t]{0.78\linewidth}%
\textbf{Ingredients}:\vspace{-3mm}
\begin{multicols}{2}
\begin{itemize}\setlength\itemsep{-1mm}
\item 1 1/2 cups butter, softened
\item 2 cups white sugar
\item 4 eggs
\item 1 tsp vanilla extract
\item 5 cups all-purpose flour
\item 2 tsp baking powder
\item 2 tsp salt
\end{itemize}
\end{multicols}
\end{minipage}
\noindent\begin{minipage}[t]{0.18\linewidth}
\centering \strut\vspace*{-\baselineskip}\newline
\includegraphics[width=0.9\linewidth]{/home/tim/Documents/projects/recipes/img/sugar-cookies.jpeg}\\
\end{minipage}\vspace{3mm}
\textbf{Directions}:
\vspace{-3mm}\begin{enumerate}\setlength\itemsep{-1mm}
\item In a large bowl, cream together butter and sugar until smooth. Beat in eggs and vanilla. Stir in the flour, baking powder, and salt. Cover, and chill dough for at least one hour (or overnight).
\item Preheat oven to 400F. Roll out dough on floured surface 1/4 to 1/2 inch thick. Cut into shapes with any cookie cutter. Place cookies 1 inch apart on ungreased cookie sheets.
\item Bake 6 to 8 minutes in preheated oven. Cool completely.
\item Optionally, top with icing made with confectioner's sugar and milk.
\end{enumerate}
\end{minipage}\vspace{8mm}

{\newpage \LARGE \textbf{Desserts}} \label{desserts}\vspace{4mm}\\
\noindent\begin{minipage}[t]{\linewidth}%
{\Large\textbf{Blueberry Cobbler}} \label{blueberry-cobbler}\hfill\textit{Sue Dunn}\\
\textbf{Yield:} \textit{8-10 servings}\\
\noindent\begin{minipage}[t]{0.78\linewidth}%
\textbf{Ingredients}:\vspace{-3mm}
\begin{multicols}{2}
\begin{itemize}\setlength\itemsep{-1mm}
\item 6 cups fresh blueberries
\item 1 cup granulated sugar
\item 2 tsp lemon zest
\item 3 Tbsp all-purpose flour
\item 1 cup, 5 Tbsp all-purpose flour
\item 6 Tbsp granulated sugar
\item 1 1/2 tsp baking powder
\item 1/4 tsp salt
\item 6 Tbsp unsalted butter
\item 1 egg, beaten
\item 1 tsp vanilla extract
\item 1 Tbsp granulated sugar
\item 1 tsp ground cinnamon
\item 1/2 tsp ground nutmeg
\end{itemize}
\end{multicols}
\end{minipage}
\noindent\begin{minipage}[t]{0.18\linewidth}
\centering \strut\vspace*{-\baselineskip}\newline
\includegraphics[width=0.9\linewidth]{/home/tim/Documents/projects/recipes/img/none.jpg}\\
\end{minipage}\vspace{3mm}
\textbf{Directions}:
\vspace{-3mm}\begin{enumerate}\setlength\itemsep{-1mm}
\item Arrange oven rack in lower third and preheat to 375F. Lightly butter 10 inch pie plate.
\item Prepare the blueberry filling: Place blueberries into prepared baking dish. In a small bowl, combine the sugar and lemon zest. Add 3 Tbsp flour and whisk until thoroughly combined. Sprinkle mixture evenly over berries in prepared baking dish and toss gently. Set aside.
\item Prepare the crumble topping: In a medium bowl, whisk together flour, sugar, baking powder, and salt until well-combined. Using a pastry blender, cut buttler into flour mixure until it resembles a coarse meal. In a small bowl, whisk vanilla into beaten egg. With a fork, gently toss beaten egg and vanilla into flour mixture until moistened and dough starts to hold together.
\item Bake the cobbler: Sprinkle biscuit crumble topping evenly over fruit filling. Sprinkle biscuit crumble topping with sugar and dust with nutmeg/cinnamon. Bake until golden brown and bubbly, about 40-45 minutes. To prevent over-browning of topping, cover with sheet of aluminum foil after 25 minutes.
\end{enumerate}
\end{minipage}\vspace{8mm}
\noindent\begin{minipage}[t]{\linewidth}%
{\Large\textbf{Blueberry Kuchen}} \label{blueberry-kuchen}\hfill\textit{Sue Dunn}\\
\textbf{Yield:} \textit{8 servings}\\
\noindent\begin{minipage}[t]{0.78\linewidth}%
\textbf{Ingredients}:\vspace{-3mm}
\begin{multicols}{2}
\begin{itemize}\setlength\itemsep{-1mm}
\item 1 cup flour
\item 1/8 tsp salt
\item 2 Tbsp sugar
\item 1/2 cup butter
\item 1 Tbsp white vinegar
\item 5 cups fresh blueberries
\item 1/2 cup sugar
\item 1/8 tsp cinnamon
\item 2 Tbsp Flour
\item 1 tsp almond extract
\end{itemize}
\end{multicols}
\end{minipage}
\noindent\begin{minipage}[t]{0.18\linewidth}
\centering \strut\vspace*{-\baselineskip}\newline
\includegraphics[width=0.9\linewidth]{/home/tim/Documents/projects/recipes/img/F3625974-C436-4C8A-B7FB-4B55F0391AEF.jpg}\\
\end{minipage}\vspace{3mm}
\textbf{Directions}:
\vspace{-3mm}\begin{enumerate}\setlength\itemsep{-1mm}
\item Preheat oven to 350 F. In medium bowl, mix 1 cup flour, salt and 2 tablespoons sugar. Cut in butter until it resembles coarse crumbs. Sprinkle with vinegar and shape into dough. With lightly floured fingers, press dough into 9-inch springform pan about 1/4 inch thickness on bottom, less thick and 1 inch high around the sides.
\item Mix 3 cups blueberries, 2 tablespoon flour, 1/2 cup sugar and cinnamon. Pour and spread the blueberries onto the crust and bake on the middle rack for 20 minutes or until crust is golden brown and filling bubbles. Remove from oven and place on cooling rack. Sprinkle with remaining 2 cups blueberries. Cool for at least 30 minutes. Run a pairing knife around the crust edge to separate from the pan before opening springform.  Serve with a scoop of vanilla ice cream.
\end{enumerate}
\end{minipage}\vspace{8mm}
\noindent\begin{minipage}[t]{\linewidth}%
{\Large\textbf{Blueberry Maple Muffins}} \label{blueberry-maple-muffins}\hfill\textit{Sue Dunn}\\
\textbf{Yield:} \textit{12 muffins}\\
\noindent\begin{minipage}[t]{0.78\linewidth}%
\textbf{Ingredients}:\vspace{-3mm}
\begin{multicols}{2}
\begin{itemize}\setlength\itemsep{-1mm}
\item 0.2 cup ground flaxseeds
\item 1 cup whole-wheat flour
\item 3/4 cup plus 2 Tbsp all-purpose flour
\item 1 1/2 tsp baking powder
\item 1 tsp ground cinnamon
\item 1/2 tsp baking soda
\item 1/4 tsp salt
\item 2 large eggs
\item 1/2 cup pure maple syrup
\item 1 cup nonfat buttermilk
\item 1/4 cup canola oil
\item 2 tsp freshly grated orange zest
\item 1 Tbsp orange juice
\item 1 tsp vanilla extract
\item 1 1/2 cups blueberries
\item 1 Tbsp sugar
\end{itemize}
\end{multicols}
\end{minipage}
\noindent\begin{minipage}[t]{0.18\linewidth}
\centering \strut\vspace*{-\baselineskip}\newline
\includegraphics[width=0.9\linewidth]{/home/tim/Documents/projects/recipes/img/blueberry-maple-muffins.jpeg}\\
\end{minipage}\vspace{3mm}
\textbf{Directions}:
\vspace{-3mm}\begin{enumerate}\setlength\itemsep{-1mm}
\item Preheat oven to 400F. Coat 12 muffin cups with cooking spray.
\item Add whole-wheat flour, all-purpose flour, flaxseed, baking powder, cinnamon, baking soda and salt together; whisk to blend. Whisk eggs and maple syrup in a medium bowl until smooth. Add buttermilk, oil, orange zest, orange juice and vanilla; whisk until blended.
\item Make a well in the dry ingredients and stir in the wet ingredients with a rubber spatula just until moistened. Fold in blueberries. Scoop the batter into the prepared muffin cups. Sprinkle the tops with sugar.
\item Bake the muffins until the tops are golden brown and spring back when touched lightly, 15 to 25 minutes. Let cool in the pan for 5 minutes. Loosen edges and turn muffins out onto a wire rack to cool slightly.
\end{enumerate}
\end{minipage}\vspace{8mm}
\noindent\begin{minipage}[t]{\linewidth}%
{\Large\textbf{Chocolate Almond Mousse}} \label{chocolate-almond-mousse}\hfill\textit{Donna Knights}\\
\textbf{Yield:} \textit{4 servings}\\
\noindent\begin{minipage}[t]{0.78\linewidth}%
\textbf{Ingredients}:\vspace{-3mm}
\begin{multicols}{2}
\begin{itemize}\setlength\itemsep{-1mm}
\item 2 cups heavy whipping cream
\item 1/3 cup dark cocoa powder
\item 1/3 cup monkfruit sweetener
\item 2 tsp almond extract
\end{itemize}
\end{multicols}
\end{minipage}
\noindent\begin{minipage}[t]{0.18\linewidth}
\centering \strut\vspace*{-\baselineskip}\newline
\includegraphics[width=0.9\linewidth]{/home/tim/Documents/projects/recipes/img/chocolate-almond-mousse.jpeg}\\
\end{minipage}\vspace{3mm}
\textbf{Directions}:
\vspace{-3mm}\begin{enumerate}\setlength\itemsep{-1mm}
\item Mix heavy whipping cream and sweetener until thick. Add cocoa powder and almond extract, and beat until stiff peaks form. Chill.
\end{enumerate}
\end{minipage}\vspace{8mm}
\noindent\begin{minipage}[t]{\linewidth}%
{\Large\textbf{Chocolate Brownies}} \label{chocolate-brownies}\hfill\textit{Sue Dunn}\\
\noindent\begin{minipage}[t]{0.78\linewidth}%
\textbf{Ingredients}:\vspace{-3mm}
\begin{multicols}{2}
\begin{itemize}\setlength\itemsep{-1mm}
\item 1/4 cup water
\item 2/3 cup vegetable oil
\item 2 eggs
\item 1 bag of Betty Crocker fudge brownie mix
\end{itemize}
\end{multicols}
\end{minipage}
\noindent\begin{minipage}[t]{0.18\linewidth}
\centering \strut\vspace*{-\baselineskip}\newline
\includegraphics[width=0.9\linewidth]{/home/tim/Documents/projects/recipes/img/2C9BA575-680B-4B61-A6FE-35815E015DB6.jpg}\\
\end{minipage}\vspace{3mm}
\textbf{Directions}:
\vspace{-3mm}\begin{enumerate}\setlength\itemsep{-1mm}
\item Preheat oven to 350F. Grease or use cooking spray.
\item Stir brownie mix, water, oil and eggs, in medium bowl until well blended. Spread in pan.
\item Bake as directed above, or until toothpick inserted 2 inches from side of pan comes out almost clean; cool. To cut warm brownies easily, cut with plastic knife using short sawing motions. Store tightly covered.
\end{enumerate}
\end{minipage}\vspace{8mm}
\noindent\begin{minipage}[t]{\linewidth}%
{\Large\textbf{Chocolate Chip Muffins}} \label{chocolate-chip-muffins}\hfill\textit{Brian Dunn}\\
\textbf{Yield:} \textit{12-18 muffins}\\
\noindent\begin{minipage}[t]{0.78\linewidth}%
\textbf{Ingredients}:\vspace{-3mm}
\begin{multicols}{2}
\begin{itemize}\setlength\itemsep{-1mm}
\item 2 cups all-purpose flour
\item 1/3 cup light-brown sugar, packed
\item 1/3 cup sugar
\item 2 tsp baking powder
\item 1/2 tsp salt
\item 2/3 cup milk
\item 1/2 cup butter, melted and cooled
\item 2 eggs, lightly beaten
\item 1 tsp vanilla
\item 1 (11 1/2 oz) package milk chocolate chips
\end{itemize}
\end{multicols}
\end{minipage}
\noindent\begin{minipage}[t]{0.18\linewidth}
\centering \strut\vspace*{-\baselineskip}\newline
\includegraphics[width=0.9\linewidth]{/home/tim/Documents/projects/recipes/img/choc_muffins.jpg}\\
\end{minipage}\vspace{3mm}
\textbf{Directions}:
\vspace{-3mm}\begin{enumerate}\setlength\itemsep{-1mm}
\item Preheat oven to 400F. Coat 12 muffin cups with cooking spray.
\item In a large bowl, stir together flour, sugars, baking powder, and salt.
\item In another bowl, stir together milk, eggs, butter, and vanilla until blended. Make a well in center of dry ingredients. Add milk mixture and stir just to combine. Stir in chocolate chips and spoon batter into prepared muffin cups.
\item Bake for 15-20 minutes or until a cake tester inserted in center of one muffin comes out clean. Remove muffin tin to wire rack. Cool for 5 minutes. Remove from tins to finish cooling
\end{enumerate}
\end{minipage}\vspace{8mm}
\noindent\begin{minipage}[t]{\linewidth}%
{\Large\textbf{Dark Chocolate Brownies}} \label{dark-chocolate-brownies}\hfill\textit{Aunt Carolyn Benjamin}\\
\textit{``Mix by hand, without a mixer.''}\\
\noindent\begin{minipage}[t]{0.78\linewidth}%
\textbf{Ingredients}:\vspace{-3mm}
\begin{multicols}{2}
\begin{itemize}\setlength\itemsep{-1mm}
\item 2 sticks of butter
\item 1 cup cocoa powder (unsweetened)
\item 4 eggs
\item 2 cups sugar
\item 1/4 tsp salt
\item 1 tsp vanilla
\item 1 cup flour
\item 1 cup chocolate chips
\end{itemize}
\end{multicols}
\end{minipage}
\noindent\begin{minipage}[t]{0.18\linewidth}
\centering \strut\vspace*{-\baselineskip}\newline
\includegraphics[width=0.9\linewidth]{/home/tim/Documents/projects/recipes/img/8C974ABB-FAF0-4069-ABC9-A068D28D8FCA.jpg}\\
\end{minipage}\vspace{3mm}
\textbf{Directions}:
\vspace{-3mm}\begin{enumerate}\setlength\itemsep{-1mm}
\item Melt butter and stir in the cocoa; set aside. Beat eggs with sugar, salt, and vanilla. Combine with butter-cocoa mixture. Add flour and chopped chocolate. Bake in greased 9x13 inch pan at 350F for 20-30 minutes.
\item Mix by hand! No mixer
\end{enumerate}
\end{minipage}\vspace{8mm}
\noindent\begin{minipage}[t]{\linewidth}%
{\Large\textbf{Flourless Chocolate Torte}} \label{flourless-chocolate-torte}\hfill\textit{}\\
\noindent\begin{minipage}[t]{0.78\linewidth}%
\textbf{Ingredients}:\vspace{-3mm}
\begin{multicols}{2}
\begin{itemize}\setlength\itemsep{-1mm}
\item 1/2 Cup Butter
\item 8 oz Semisweet Chocolate (Chopped, baking squares are best)
\item 5 eggs (Separated)
\item 3/4 Cup White Sugar
\item 1 Cup Almonds (Ground)
\end{itemize}
\end{multicols}
\end{minipage}
\noindent\begin{minipage}[t]{0.18\linewidth}
\centering \strut\vspace*{-\baselineskip}\newline
\includegraphics[width=0.9\linewidth]{/home/tim/Documents/projects/recipes/img/4695E71E-B7E7-4968-AE56-31D2D6D9CC1D.jpg}\\
\end{minipage}\vspace{3mm}
\textbf{Directions}:
\vspace{-3mm}\begin{enumerate}\setlength\itemsep{-1mm}
\item Preheat over to 350 degrees. Line bottom and sides of 9" spring-form pan with foil. Grease foil.
\item Melt butter and chocolate over low heat.
\item Stir until smooth and let cool.
\item In a medium bowl, beat whites until stiff; about 2 minutes. In a separate bowl, beat together yolks and sugar until thick and pale; about 1 minute. Blend in chocolate mixture and stir in almonds. Fold in beaten whites, 1/3 at a time, into chocolate until no streaks of white remain. Scrape into prepared pan.
\item Place an 8" baking pan with 1" water in it on the bottom rack of the oven to make the torte more moist. Bake torte on center rack for 45-50 minutes or until sides begin to pull away from pan and top is set in center.
\item Cover the torte loosely with foil for the last 20 minutes of baking.
\item Cool on wire rack for 10 minutes and then carefully remove sides of pan. Invert onto a serving plate and cool completely.
\end{enumerate}
\end{minipage}\vspace{8mm}
\noindent\begin{minipage}[t]{\linewidth}%
{\Large\textbf{Fudge}} \label{fudge}\hfill\textit{}\\
\noindent\begin{minipage}[t]{0.78\linewidth}%
\textbf{Ingredients}:\vspace{-3mm}
\begin{multicols}{2}
\begin{itemize}\setlength\itemsep{-1mm}
\item 1 lb confectioners sugar (1/2 of a bag)
\item 1/2 cup cocoa powder
\item 1/4 cup milk
\item 1 stick butter
\item 1 Tbsp vanilla
\end{itemize}
\end{multicols}
\end{minipage}
\noindent\begin{minipage}[t]{0.18\linewidth}
\centering \strut\vspace*{-\baselineskip}\newline
\includegraphics[width=0.9\linewidth]{/home/tim/Documents/projects/recipes/img/FE81D3DB-35AB-475B-AFFA-0AFBADA81F7F.jpg}\\
\end{minipage}\vspace{3mm}
\textbf{Directions}:
\vspace{-3mm}\begin{enumerate}\setlength\itemsep{-1mm}
\item Put sugar in a large bowl ( I used a glass bowl and it worked well). Add cocoa but do not mix. Make a hole in the center of the sugar and pour milk into the hole. Lay the stick of butter across the top. Put in microwave for 2-5 minutes until butter is melted and milk foams. Remove from microwave. Use a mixer to wet all ingredients. Add vanilla and then keep beating with the mixer until thick. Line an 8x8 or 9x9 pan with foil before pouring. Cool overnight.
\item Optional add-ins: 1 cup mini marshmallows, 1 cup chocolate chips, 1 cup chopped walnuts, 1/2 cup peanut butter
\end{enumerate}
\end{minipage}\vspace{8mm}
\noindent\begin{minipage}[t]{\linewidth}%
{\Large\textbf{Island Gem Bars}} \label{island-gem-bars}\hfill\textit{Sue Dunn}\\
\textit{``Also known as 7-layer bars.''}\\
\noindent\begin{minipage}[t]{0.78\linewidth}%
\textbf{Ingredients}:\vspace{-3mm}
\begin{multicols}{2}
\begin{itemize}\setlength\itemsep{-1mm}
\item 1/2 cup butter, softened
\item 1/2 cup packed brown sugar
\item 1 tsp vanilla extract
\item 1 1/2 cups all-purpose flour
\item 1 cup packed brown sugar
\item 1/4 cup all-purpose flour
\item 1/4 tsp salt
\item 1 tsp vanilla extract
\item 2 large eggs
\item 1/3 to 1/2 oz. can of flaked sweetened coconut
\item 1 cup semisweet chocolate chips
\end{itemize}
\end{multicols}
\end{minipage}
\noindent\begin{minipage}[t]{0.18\linewidth}
\centering \strut\vspace*{-\baselineskip}\newline
\includegraphics[width=0.9\linewidth]{/home/tim/Documents/projects/recipes/img/island-gem-bars.jpg}\\
\end{minipage}\vspace{3mm}
\textbf{Directions}:
\vspace{-3mm}\begin{enumerate}\setlength\itemsep{-1mm}
\item Preheat oven to 350F. Lightly butter a 13x9 inch baking pan.
\item Cookie Layer: Cream butter, brown sugar, and vanilla with an electric mixer until light. Add flour and blend. Press cookie dough into prepared pan. Bake for 10 minutes or until edges are golden brown. Cool on a wire rack. Leave oven on.
\item Coconut Layer: Beat brown sugar, flour, salt, vanilla, and eggs with an electric mixer. Stir in coconut. Spread over cooled cookie layer. Sprinkle top with chocolate chips. 
\item Bake 20-25 minutes or until edges are golden brown. Cool thoroughly before cutting it into bars.
\end{enumerate}
\end{minipage}\vspace{8mm}
\noindent\begin{minipage}[t]{\linewidth}%
{\Large\textbf{Lemon Bars}} \label{lemon-bars}\hfill\textit{Sue Dunn}\\
\textbf{Yield:} \textit{24 bars}\\
\noindent\begin{minipage}[t]{0.78\linewidth}%
\textbf{Ingredients}:\vspace{-3mm}
\begin{multicols}{2}
\begin{itemize}\setlength\itemsep{-1mm}
\item 1 cup unsalted butter, melted
\item 1/2 cup granulated sugar
\item 2 tsp vanilla extract
\item 1/2 tsp salt
\item 2 cups all-purpose flour
\item 2 Tbsp all-purpose flour
\item 2 cups granulated sugar
\item 6 Tbsp all-purpose flour
\item 6 eggs
\item 1 cup lemon juice (4-6 lemons)
\item confectioner's sugar (optional)
\end{itemize}
\end{multicols}
\end{minipage}
\noindent\begin{minipage}[t]{0.18\linewidth}
\centering \strut\vspace*{-\baselineskip}\newline
\includegraphics[width=0.9\linewidth]{/home/tim/Documents/projects/recipes/img/none.jpg}\\
\end{minipage}\vspace{3mm}
\textbf{Directions}:
\vspace{-3mm}\begin{enumerate}\setlength\itemsep{-1mm}
\item Preheat the oven to 325F. Line the bottom and sides of a 9x13 inch baking pan with parchment paper. Set aside.
\item Make the crust: mix the melted butter, sugar, vanilla extract, and salt together in a medium bowl. Add the flour and stir to completely combine. The dough will be thick. Press firmly into prepared pan. Bake for 20-22 minutes or until the edges are lightly browned. Remove from oven. Using a fork, poke holes all over the top of the warm crust (which helps the filling stick in place). Set aside.
\item Make the filling: sift the sugar and flour together in a large bowl. Whisk in the egs, then the lemon juice until completely combined.
\item Pour filling over warm crust. Bake the bars for 22-26 minutes or until the center is relatively set (no jiggle). Remove bars from the oven and cool completely, then chill. Optionally dust with confectioner's sugar.
\end{enumerate}
\end{minipage}\vspace{8mm}
\noindent\begin{minipage}[t]{\linewidth}%
{\Large\textbf{Lemon Poppy Seed Muffins}} \label{lemon-poppy-seed-muffins}\hfill\textit{Sue Dunn}\\
\textbf{Yield:} \textit{12 muffins}\\
\noindent\begin{minipage}[t]{0.78\linewidth}%
\textbf{Ingredients}:\vspace{-3mm}
\begin{multicols}{2}
\begin{itemize}\setlength\itemsep{-1mm}
\item 3 cups all-purpose flour
\item 1 cup sugar
\item 2 Tbsp poppy seeds
\item 1 Tbsp baking powder
\item 1/2 tsp baking soda
\item 1/2 tsp salt
\item 1 1/2 cup plain whole milk yogurt
\item 2 Tbsp fresh lemon juice
\item 1 1/2 Tbsp grated lemon zest
\item 2 eggs
\item 8 Tbsp unsalted butter, melted and cooled
\item 1/4 cup sugar
\item 1/4 cup lemon juice
\end{itemize}
\end{multicols}
\end{minipage}
\noindent\begin{minipage}[t]{0.18\linewidth}
\centering \strut\vspace*{-\baselineskip}\newline
\includegraphics[width=0.9\linewidth]{/home/tim/Documents/projects/recipes/img/none.jpg}\\
\end{minipage}\vspace{3mm}
\textbf{Directions}:
\vspace{-3mm}\begin{enumerate}\setlength\itemsep{-1mm}
\item Adjust the oven rack to the middle position and heat to 375F. Grease a 12 cup muffin tin.
\item Whisk flour, sugar, poppy seeds, baking powder, baking soda, and salt together in a large bowl.
\item In a separate bowl, whisk yogurt, lemon zest, and eggs until smooth. Gently fold yogurt mixture into flour mixture until just combined. Fold into melted butter. Do not overmix or else muffins will be dense.
\item Divide batter evenly among prepared muffin cups. Bake until golden brown and inserted toothpick comes out clean, 20-25 minutes. Rotate muffin tin halfway through baking.
\item While muffins bake, simmer sugar and lemon juice together in a small saucepan over medium heat until it turns into a light syrup, about 3-5 minutes.
\item Remove muffin tin from oven. Brush with lemon syrup. Sprinkle with coarse sugar. Let muffins cool before serving.
\end{enumerate}
\end{minipage}\vspace{8mm}
\noindent\begin{minipage}[t]{\linewidth}%
{\Large\textbf{No Bake Brownies}} \label{no-bake-brownies}\hfill\textit{Nancy Feth (Grandma)}\\
\noindent\begin{minipage}[t]{0.78\linewidth}%
\textbf{Ingredients}:\vspace{-3mm}
\begin{multicols}{2}
\begin{itemize}\setlength\itemsep{-1mm}
\item 12 oz chocolate chips
\item 1 cup evaporated milk
\item 3 cups crushed vanilla wafers
\item 2 cups miniature marshmallows
\item 1 cup chopped nuts
\item 1 cup powdered sugar
\item 1/2 tsp Salt
\end{itemize}
\end{multicols}
\end{minipage}
\noindent\begin{minipage}[t]{0.18\linewidth}
\centering \strut\vspace*{-\baselineskip}\newline
\includegraphics[width=0.9\linewidth]{/home/tim/Documents/projects/recipes/img/3E7638E8-6016-4515-B245-DC058DFAA008.jpg}\\
\end{minipage}\vspace{3mm}
\textbf{Directions}:
\vspace{-3mm}\begin{enumerate}\setlength\itemsep{-1mm}
\item Stir chocolate chips and evaporated milk (save 2 tsp for later) on low heat until smooth. Remove from heat.
\item Mix together Vanilla wafers, marshmallows, chopped nuts, powdered sugar, salt and half of chocolate mixture (save rest for top).
\item Stir 2 tsp of evaporated milk into rest of chocolate. Spread over mixture. Chill.
\item Use 9x9 glass pan.
\end{enumerate}
\end{minipage}\vspace{8mm}
\noindent\begin{minipage}[t]{\linewidth}%
{\Large\textbf{Oreo Truffles}} \label{oreo-truffles}\hfill\textit{Brian Dunn}\\
\noindent\begin{minipage}[t]{0.78\linewidth}%
\textbf{Ingredients}:\vspace{-3mm}
\begin{multicols}{2}
\begin{itemize}\setlength\itemsep{-1mm}
\item 1 lb Oreo cookies (3 sleeves)
\item 8 oz cream cheese, room temperature
\item 1 tsp vanilla extract
\item 1 lb milk chocolate
\item 1/2 lb white chocolate
\end{itemize}
\end{multicols}
\end{minipage}
\noindent\begin{minipage}[t]{0.18\linewidth}
\centering \strut\vspace*{-\baselineskip}\newline
\includegraphics[width=0.9\linewidth]{/home/tim/Documents/projects/recipes/img/589CE24C-8D10-4225-9115-20F0F97F9F80.jpg}\\
\end{minipage}\vspace{3mm}
\textbf{Directions}:
\vspace{-3mm}\begin{enumerate}\setlength\itemsep{-1mm}
\item Using a food processor, grind cookies to a fine powder. With a mixer, blend cookie powder, cream cheese and vanilla extract until thoroughly mixed (there should be no white traces of cream cheese).
\item Roll into small balls and place on wax-lined cookie sheet. Refrigerate for 45 minutes.
\item Line two cookie sheets with wax paper. In double-boiler, melt milk chocolate. Dip balls and coat thoroughly. With slotted spoon, lift balls out of chocolate and let excess chocolate drip off. Place on wax-paper-lined cookie sheet.
\item In separate double boiler, melt white chocolate. Using a fork, drizzle white chocolate over balls. Refrigerate in an airtight container.
\end{enumerate}
\end{minipage}\vspace{8mm}
\noindent\begin{minipage}[t]{\linewidth}%
{\Large\textbf{Pecan Tassies}} \label{pecan-tassies}\hfill\textit{Nancy Feth}\\
\noindent\begin{minipage}[t]{0.78\linewidth}%
\textbf{Ingredients}:\vspace{-3mm}
\begin{multicols}{2}
\begin{itemize}\setlength\itemsep{-1mm}
\item 1/2 cup butter, softened
\item 3 oz cream cheese
\item 1 cup flour
\item 2 eggs
\item 7/8 cup brown sugar
\item 1 tsp vanilla
\item 1 Tbsp butter, melted
\item 1 dash salt
\item 1/2 cup coarsely chopped pecans
\end{itemize}
\end{multicols}
\end{minipage}
\noindent\begin{minipage}[t]{0.18\linewidth}
\centering \strut\vspace*{-\baselineskip}\newline
\includegraphics[width=0.9\linewidth]{/home/tim/Documents/projects/recipes/img/pecan-tassies.jpeg}\\
\end{minipage}\vspace{3mm}
\textbf{Directions}:
\vspace{-3mm}\begin{enumerate}\setlength\itemsep{-1mm}
\item Combine crust ingredients together. Form into large ball. Fill cups with dough and press around edges of each cup.
\item Combine filling ingredients together. Fill cups to top with filling. Bake in oven for 15 minutes at 350. Then turn oven down to 250 for 10 minutes. 
\end{enumerate}
\end{minipage}\vspace{8mm}
\noindent\begin{minipage}[t]{\linewidth}%
{\Large\textbf{Raspberry Crisp}} \label{raspberry-crisp}\hfill\textit{Sue Dunn}\\
\noindent\begin{minipage}[t]{0.78\linewidth}%
\textbf{Ingredients}:\vspace{-3mm}
\begin{multicols}{2}
\begin{itemize}\setlength\itemsep{-1mm}
\item 4 cups fresh raspberries
\item 3/4 cup sugar
\item 2 Tbsp corn starch
\item 1 3/4 cups quick oats
\item 1 cup flour
\item 1 cup brown sugar
\item 1/2 tsp baking soda
\item 1/2 cup cold butter
\item 3/4 cup crushed almonds
\item 1 tsp almond extract
\item 2 Tbsp amaretto
\end{itemize}
\end{multicols}
\end{minipage}
\noindent\begin{minipage}[t]{0.18\linewidth}
\centering \strut\vspace*{-\baselineskip}\newline
\includegraphics[width=0.9\linewidth]{/home/tim/Documents/projects/recipes/img/raspberry-crisp.jpeg}\\
\end{minipage}\vspace{3mm}
\textbf{Directions}:
\vspace{-3mm}\begin{enumerate}\setlength\itemsep{-1mm}
\item Preheat oven to 350F. Crush 1 cup berries; add enough water to measure 1 cup. In a saucepan, mix sugar and corn starch, stirring in the raspberry mixture. Bring to a boil; cook and stir for 2 minutes. Remove from heat, stir in remaining berries, and let cool.
\item In a bowl, combine the oats, flour, brown sugar, baking soda, and almonds. Cut in the butter to make crumbs, adding the amaretto and almond extract. Press half into a greased 9x9 baking dish. Spread with berries, sprinkle in remaining crumbs, and bake at 350F for 25 minutes.
\end{enumerate}
\end{minipage}\vspace{8mm}
\noindent\begin{minipage}[t]{\linewidth}%
{\Large\textbf{Rock Candy}} \label{rock-candy}\hfill\textit{Bill Nye}\\
\textbf{Yield:} \textit{one rock}\\
\noindent\begin{minipage}[t]{0.78\linewidth}%
\textbf{Ingredients}:\vspace{-3mm}
\begin{multicols}{2}
\begin{itemize}\setlength\itemsep{-1mm}
\item 1 cup water
\item 2 cups of granulated sugar
\item candy flavoring (peppermint, cherry, lemon, vanilla)
\end{itemize}
\end{multicols}
\end{minipage}
\noindent\begin{minipage}[t]{0.18\linewidth}
\centering \strut\vspace*{-\baselineskip}\newline
\includegraphics[width=0.9\linewidth]{/home/tim/Documents/projects/recipes/img/rock-candy.jpeg}\\
\end{minipage}\vspace{3mm}
\textbf{Directions}:
\vspace{-3mm}\begin{enumerate}\setlength\itemsep{-1mm}
\item Tie one end of a piece of string around the middle of a stick or pencil. Cut the string, if necessary, so that it is a little shorter than your jar. 
\item Moisten the string with a little water and roll it in the sugar. Put the paper clip on the end of the string to help it hang straight down. Lay the stick over the top of the glass jar so that the string hangs down inside the jar. The end of the string should not touch the bottom of the jar. 
\item Cook the candy mixture: Put the cup of water into the sauce pan and heat until it boils. Add 2 cups of sugar to the boiling water while stirring. Keep stirring until the sugar dissolves (or 240F). Remove pan from heat. If you want to add flavoring or color, stir it in now. Pour the hot mixture into the jar. 
\item Let the sugar water sit for a few days. The crystals will begin to form along the string in a few hours. Let them grow for 3 to 10 days (or more).
\item When you're ready to eat the rock candy, take the candy-covered string out of the jar. Break the pieces apart and enjoy.
\end{enumerate}
\end{minipage}\vspace{8mm}
\noindent\begin{minipage}[t]{\linewidth}%
{\Large\textbf{Sticky Buns}} \label{sticky-buns}\hfill\textit{Nancy Wood}\\
\textbf{Yield:} \textit{24 sticky buns}\\
\noindent\begin{minipage}[t]{0.78\linewidth}%
\textbf{Ingredients}:\vspace{-3mm}
\begin{multicols}{2}
\begin{itemize}\setlength\itemsep{-1mm}
\item 1 cup milk
\item 1/2 cup sugar
\item 1 tsp salt
\item 1/2 cup butter
\item 1/4 cup warm water
\item 1 pkg yeast
\item 1 egg
\item 4 cups flour
\item 1 cup light brown sugar
\item 3/4 cup butter
\item 1 Tbsp water
\item some maple syrup
\end{itemize}
\end{multicols}
\end{minipage}
\noindent\begin{minipage}[t]{0.18\linewidth}
\centering \strut\vspace*{-\baselineskip}\newline
\includegraphics[width=0.9\linewidth]{/home/tim/Documents/projects/recipes/img/11142383-E023-4B50-BF54-C1AEF05840DF.jpg}\\
\end{minipage}\vspace{3mm}
\textbf{Directions}:
\vspace{-3mm}\begin{enumerate}\setlength\itemsep{-1mm}
\item Scald milk in saucepan. Add salt, sugar and butter. Stir until butter is melted. Do not boil. Cool to 110-120F. Meanwhile, stir yeast into warm water to dissolve. Add to milk mixture and stir in egg. Place milk/yeast mixture in large bowl and add 2 cups of flour. Stir until smooth. Add remaining flour to make a stiff dough. Raise in the refrigerator for 2 hours (or up to 3 days).
\item Remove dough to work area and roll 24x12 inch rectangle. Brush with melted butter and sprinkle with cinnamon or cinnamon sugar. Roll from long end and slice into 24 pieces. Pour topping in the bottom of your baking pan. Place cut pieces of dough on top and let rise 1 hour or until the sticky buns are well risen. To bake in the morning, place the pan in the fridge before the last rising, let come to room temperature and rise before baking. Bake at 350F for 20-25 minutes.
\end{enumerate}
\end{minipage}\vspace{8mm}

{\newpage \LARGE \textbf{Pastas}} \label{pastas}\vspace{4mm}\\
\noindent\begin{minipage}[t]{\linewidth}%
{\Large\textbf{Chicken and Shrimp Carbonara}} \label{chicken-and-shrimp-carbonara}\hfill\textit{Olive Garden}\\
\textit{``Marinade: olive oil - chicken; Sauce: butter - salt; Cheese Topping: Romano - parsley''}\\
\textbf{Yield:} \textit{16 servings}\\
\noindent\begin{minipage}[t]{0.78\linewidth}%
\textbf{Ingredients}:\vspace{-3mm}
\begin{multicols}{2}
\begin{itemize}\setlength\itemsep{-1mm}
\item 1 cup extra virgin olive oil
\item 1 cup hot water
\item 1 Tbsp Italian seasoning
\item 1 Tbsp garlic, chopped
\item 3 lbs (or 2) chicken strips (or pre-cooked large shrimp)
\item 1 cup butter
\item 1 1/2 tsp garlic, chopped
\item 3 Tbsp bacon
\item 3 Tbsp all-purpose flour
\item 1 cup Parmesan cheese, grated
\item 4 cups heavy cream
\item 4 cups milk
\item 1/2 tsp black pepper
\item 1/4 tsp salt
\item 2 (14 oz) boxes angel hair linguine
\item 3 Tbsp Romano cheese, grated
\item 3 Tbsp Parmesan cheese, grated
\item 1 3/4 cups Mozzarella cheese, shredded
\item 1/2 cup Panko breadcrumbs
\item 1 1/2 tsp garlic, chopped
\item 1 1/2 Tbsp butter, melted
\item 2 Tbsp parsley, chopped
\item 2 roasted red peppers, chopped
\item 1/4 cups bacon
\end{itemize}
\end{multicols}
\end{minipage}
\noindent\begin{minipage}[t]{0.18\linewidth}
\centering \strut\vspace*{-\baselineskip}\newline
\includegraphics[width=0.9\linewidth]{/home/tim/Documents/projects/recipes/img/carbonara.jpg}\\
\end{minipage}\vspace{3mm}
\textbf{Directions}:
\vspace{-3mm}\begin{enumerate}\setlength\itemsep{-1mm}
\item Preheat oven to 350F.
\item Whisk extra virgin olive oil together with hot water, seasoning, and chopped garlic. Add the meat, cover, and let marinate in the refrigerator for at least 30 minutes.
\item Cook Bacon until crispy, break apart into bits.
\item Combine cheese mixture in a mixing bowl. Stir until well-blended.
\item Melt Butter in a large saucepan over medium heat. Add garlic and bacon. Sauté for 5 minutes, stirring frequently.  Add remaining sauce ingredients and bring to a boil.  Reduce heat and allow to simmer.
\item Cook pasta. Strain and stir with oil to prevent noodles from sticking.
\item Simultaneously, preheat large skillet. Cook meat until both sides are cooked. Add chopped red peppers and remaining bacon. Cook until meat is hot (165F for chicken and 155F for shrimp).
\item Add meat to sauce. Stir until well-blended. Place pasta in casserole dishes, leaving room for sauce and topping.
\item Add sauce on top of noodles in casserole dishes. Use fork to evenly spread the sauce in between the noodles. Top evenly with cheese topping mixture. Cook in oven for 15 minutes, until golden brown.
\end{enumerate}
\end{minipage}\vspace{8mm}
\noindent\begin{minipage}[t]{\linewidth}%
{\Large\textbf{Lasagna}} \label{lasagna}\hfill\textit{Sue Dunn}\\
\textbf{Yield:} \textit{10 servings}\\
\noindent\begin{minipage}[t]{0.78\linewidth}%
\textbf{Ingredients}:\vspace{-3mm}
\begin{multicols}{2}
\begin{itemize}\setlength\itemsep{-1mm}
\item 1 lb pork sausage or ground beef
\item 1/2 onion, chopped
\item 1 clove garlic, minced
\item 1 (16 oz) can diced tomatoes
\item 1 (8 oz) can tomato sauce
\item 1 (6 oz) can tomato paste
\item 2 tsp dried basil
\item 1 tsp salt
\item 1/2 tsp pepper
\item 8 oz lasagne noodles
\item 1 Tbsp olive oil
\item 2 eggs
\item 2 1/2 cups ricotta cheese
\item 3/4 cup grated Parmesan
\item 2 Tbsp dried parsley flakes
\item 1 lb mozzarella cheese, thinly sliced
\end{itemize}
\end{multicols}
\end{minipage}
\noindent\begin{minipage}[t]{0.18\linewidth}
\centering \strut\vspace*{-\baselineskip}\newline
\includegraphics[width=0.9\linewidth]{/home/tim/Documents/projects/recipes/img/lasagna.jpeg}\\
\end{minipage}\vspace{3mm}
\textbf{Directions}:
\vspace{-3mm}\begin{enumerate}\setlength\itemsep{-1mm}
\item Cook meat, onion, and garlic until meat is browned. Drain fat. Stir in the undrained tomatoes and next four ingredients. Meanwhile, cook noodles until tender in boiling salted water with oil added to water. Drain; rinse noodles. Beat eggs; add ricotta, 1/2 cup of Parmesan, parsley, salt, and pepper.
\item Layer half of the noodles in a 13 x 9 x 2 inch baking dish; spread with half of ricotta filling. Add half of the mozzarella cheese and half of the meat sauce. Repeat layers. Sprinkle remaining Parmesan on top.
\item Bake in a 375F oven for 30-35 minutes or until heated through. Let stand for 10 minutes.
\end{enumerate}
\end{minipage}\vspace{8mm}
\noindent\begin{minipage}[t]{\linewidth}%
{\Large\textbf{Mac and Cheese}} \label{mac-and-cheese}\hfill\textit{Sue Dunn}\\
\textbf{Yield:} \textit{6 servings}\\
\noindent\begin{minipage}[t]{0.78\linewidth}%
\textbf{Ingredients}:\vspace{-3mm}
\begin{multicols}{2}
\begin{itemize}\setlength\itemsep{-1mm}
\item 1 lb elbow macaroni
\item 4 cups shredded sharp cheddar cheese
\item 1 cup grated parmesan cheese
\item 6 cups milk
\item 1/2 cup butter
\item 5 Tbsp all-purpose flour
\item 4 Tbsp butter
\item 1 cup bread crumbs
\item 2 pinch paprika
\item 1 pinch nutmeg
\end{itemize}
\end{multicols}
\end{minipage}
\noindent\begin{minipage}[t]{0.18\linewidth}
\centering \strut\vspace*{-\baselineskip}\newline
\includegraphics[width=0.9\linewidth]{/home/tim/Documents/projects/recipes/img/5912D818-08DA-4FC5-9BD6-33E9D7C12A43.jpg}\\
\end{minipage}\vspace{3mm}
\textbf{Directions}:
\vspace{-3mm}\begin{enumerate}\setlength\itemsep{-1mm}
\item Cook macaroni; drain.
\item In a saucepan, melt butter or margarine over medium heat. Stir in enough flour to make a roux. Add milk to roux slowly, stirring constantly. Stir in cheeses, and cook over low heat until cheese is melted and the sauce is a little thick. Put macaroni in large casserole dish, and pour sauce over macaroni. Stir well.
\item Melt butter or margarine in a skillet over medium heat. Add breadcrumbs and brown. Spread over the macaroni and cheese to cover. Sprinkle with a little paprika and nutmeg.
\item Bake at 350F for 30 minutes.
\end{enumerate}
\end{minipage}\vspace{8mm}
\noindent\begin{minipage}[t]{\linewidth}%
{\Large\textbf{Meatballs}} \label{meatballs}\hfill\textit{Sue Dunn}\\
\textbf{Yield:} \textit{14 meatballs}\\
\noindent\begin{minipage}[t]{0.78\linewidth}%
\textbf{Ingredients}:\vspace{-3mm}
\begin{multicols}{2}
\begin{itemize}\setlength\itemsep{-1mm}
\item 1 cup panko bread crumbs
\item 2 eggs
\item 1/2 cup water
\item 1/3 cup grated Pecorino Romano cheese
\item 4 cloves garlic, minced
\item 2 Tbsp parsley
\item 1 tsp Italian seasoning
\item 1 tsp salt
\item 1/2 tsp black pepper
\item 1 lb ground beef
\end{itemize}
\end{multicols}
\end{minipage}
\noindent\begin{minipage}[t]{0.18\linewidth}
\centering \strut\vspace*{-\baselineskip}\newline
\includegraphics[width=0.9\linewidth]{/home/tim/Documents/projects/recipes/img/none.jpg}\\
\end{minipage}\vspace{3mm}
\textbf{Directions}:
\vspace{-3mm}\begin{enumerate}\setlength\itemsep{-1mm}
\item Preheat oven to 450F. Add bread crumbs, eggs, water, cheese, garlic, parsley, seasoning, salt, and pepper to large bowl. Stir to combine. Add beef, mix by hand until just combined (overmixing will toughen meatballs).
\item Scoop about 3 Tbsp meat mixture to form balls. Roll in cupped hands until smooth. Arrange meatballs on baking sheet 1 1/2 inches apart.
\item Bake for 15 minutes or until internal temperature reaches 160F. Transfer to clean platter.
\end{enumerate}
\end{minipage}\vspace{8mm}
\noindent\begin{minipage}[t]{\linewidth}%
{\Large\textbf{Spaghetti Squash Gratin}} \label{spaghetti-squash-gratin}\hfill\textit{Sue Dunn}\\
\textbf{Yield:} \textit{5 cups}\\
\noindent\begin{minipage}[t]{0.78\linewidth}%
\textbf{Ingredients}:\vspace{-3mm}
\begin{multicols}{2}
\begin{itemize}\setlength\itemsep{-1mm}
\item 1 (2-3 lb) spaghetti squash, halved and seeded
\item 1 clove garlic, chopped
\item 1 Tbsp fresh thyme
\item 2 Tbsp fresh parsley
\item 1/2 tsp salt
\item 1/4 tsp fresh cracked pepper
\item 1 (8 oz) pkg creme fraiche
\item 1 cup shredded asiago cheese
\end{itemize}
\end{multicols}
\end{minipage}
\noindent\begin{minipage}[t]{0.18\linewidth}
\centering \strut\vspace*{-\baselineskip}\newline
\includegraphics[width=0.9\linewidth]{/home/tim/Documents/projects/recipes/img/D7FA3935-A406-4952-9C6A-CFFC0F7BC23F.jpg}\\
\end{minipage}\vspace{3mm}
\textbf{Directions}:
\vspace{-3mm}\begin{enumerate}\setlength\itemsep{-1mm}
\item Preheat oven to 450 degrees. 
\item Microwave squash (one half at a time, skin side up) on high until tender (10-12 minutes). Set aside, covered, until cool enough to handle (10-15 minutes).
\item Run tines of fork lengthwise over cut surface of squash to loosen spaghetti-like strands; scoop out strands. If necessary, drain excess liquid. Set aside. 
\item Combine garlic, thyme, parsley, salt, pepper, creme fraiche, and 2/3 cup cheese in small bowl. Fold into squash; place in casserole dish. Top with remaining cheese.
\item Bake 20 minutes or until lightly browned.
\end{enumerate}
\end{minipage}\vspace{8mm}

{\newpage \LARGE \textbf{Pies}} \label{pies}\vspace{4mm}\\
\noindent\begin{minipage}[t]{\linewidth}%
{\Large\textbf{Apple Crumb Pie}} \label{apple-crumb-pie}\hfill\textit{BLC LYO}\\
\textbf{Yield:} \textit{1 pie}\\
\noindent\begin{minipage}[t]{0.78\linewidth}%
\textbf{Ingredients}:\vspace{-3mm}
\begin{multicols}{2}
\begin{itemize}\setlength\itemsep{-1mm}
\item 1 deep dish pie crust
\item 7 apples, peeled and sliced
\item 1/2 cup sugar
\item 1 tsp cinnamon
\item dash of salt
\item 1/3 cup butter
\item 1/2 cup sugar
\item 3/4 cup flour
\end{itemize}
\end{multicols}
\end{minipage}
\noindent\begin{minipage}[t]{0.18\linewidth}
\centering \strut\vspace*{-\baselineskip}\newline
\includegraphics[width=0.9\linewidth]{/home/tim/Documents/projects/recipes/img/apple_crumb_pie.jpg}\\
\end{minipage}\vspace{3mm}
\textbf{Directions}:
\vspace{-3mm}\begin{enumerate}\setlength\itemsep{-1mm}
\item Mix apples, sugar, cinnamon, and salt; place mixture in pie shell.
\item Cream together butter, sugar, and flour. Add to top of filling.
\item Bake at 425F for 40-50 minutes.
\end{enumerate}
\end{minipage}\vspace{8mm}
\noindent\begin{minipage}[t]{\linewidth}%
{\Large\textbf{Biscuit Pie Crust}} \label{biscuit-pie-crust}\hfill\textit{Sue Dunn}\\
\noindent\begin{minipage}[t]{0.78\linewidth}%
\textbf{Ingredients}:\vspace{-3mm}
\begin{multicols}{2}
\begin{itemize}\setlength\itemsep{-1mm}
\item 2 cups flour
\item 1 Tbsp baking powder
\item 3/4 tsp salt
\item 6 Tbsp unsalted butter, cold
\item 3/4 cup milk
\item 2 Tbsp basil, chopped (optional)
\end{itemize}
\end{multicols}
\end{minipage}
\noindent\begin{minipage}[t]{0.18\linewidth}
\centering \strut\vspace*{-\baselineskip}\newline
\includegraphics[width=0.9\linewidth]{/home/tim/Documents/projects/recipes/img/none.jpg}\\
\end{minipage}\vspace{3mm}
\textbf{Directions}:
\vspace{-3mm}\begin{enumerate}\setlength\itemsep{-1mm}
\item Mix the flour, baking powder, and salt thoroughly. Add the butter and mix in until it resembles a coarse meal.
\item Add the milk and mix until it forms a dough. Split the dough in half and roll out on a well-floured surface one at a time.
\end{enumerate}
\end{minipage}\vspace{8mm}
\noindent\begin{minipage}[t]{\linewidth}%
{\Large\textbf{Grasshopper Pie}} \label{grasshopper-pie}\hfill\textit{Sue Dunn}\\
\noindent\begin{minipage}[t]{0.78\linewidth}%
\textbf{Ingredients}:\vspace{-3mm}
\begin{multicols}{2}
\begin{itemize}\setlength\itemsep{-1mm}
\item 25 oreos
\item 1/2 cup butter, melted
\item 2 cups marshmallow creme (Fluff)
\item 1/4 cup Creme de Menthe liqueur
\item 2 cups whipping cream
\end{itemize}
\end{multicols}
\end{minipage}
\noindent\begin{minipage}[t]{0.18\linewidth}
\centering \strut\vspace*{-\baselineskip}\newline
\includegraphics[width=0.9\linewidth]{/home/tim/Documents/projects/recipes/img/grasshopper_pie.jpg}\\
\end{minipage}\vspace{3mm}
\textbf{Directions}:
\vspace{-3mm}\begin{enumerate}\setlength\itemsep{-1mm}
\item Crush cookies and set aside 1/4 cup of crumbs. Place remaining crumbs in a medium bowl and mix in melted butter. Press mixture firmly into bottom and sides of a 9 inch springform pan.
\item In a large mixing bowl, whip together marshmallow creme and creme de menthe until smooth. In a separate bowl, whip cream until soft peaks form, then fold into marshmallow mixture. Pour mixture into pan and sprinkle reserved cookie crumbs on top. Freeze at least 2 hours, until firm. Remove from freezer 20 minutes before serving to soften slightly.
\end{enumerate}
\end{minipage}\vspace{8mm}

{\newpage \LARGE \textbf{Pizzas}} \label{pizzas}\vspace{4mm}\\
\noindent\begin{minipage}[t]{\linewidth}%
{\Large\textbf{Stromboli}} \label{stromboli}\hfill\textit{Sue Dunn}\\
\noindent\begin{minipage}[t]{0.78\linewidth}%
\textbf{Ingredients}:\vspace{-3mm}
\begin{multicols}{2}
\begin{itemize}\setlength\itemsep{-1mm}
\item 3/4 cup warm water
\item 1 Tbsp granulated sugar
\item 1 1/2 tsp yeast
\item 1 3/4 cup all-purpose flour
\item 1/2 tsp salt
\item 1/2 cup pizza sauce
\item 1/2 lb sliced pepperoni
\item 1 1/2 cup shredded mozzarella
\item garlic powder
\item ricotta cheese (suggested)
\item spinach (suggested)
\item onion (suggested)
\item 1 egg, beaten
\item 1/4 cup shredded parmesan
\item parsley
\end{itemize}
\end{multicols}
\end{minipage}
\noindent\begin{minipage}[t]{0.18\linewidth}
\centering \strut\vspace*{-\baselineskip}\newline
\includegraphics[width=0.9\linewidth]{/home/tim/Documents/projects/recipes/img/none.jpg}\\
\end{minipage}\vspace{3mm}
\textbf{Directions}:
\vspace{-3mm}\begin{enumerate}\setlength\itemsep{-1mm}
\item Whisk together water, sugar, and yeast. Add in flour and salt just until dough comes together. Transfer dough to a lightly-greased bowl and let rise for 1 hour.
\item Preheat oven to 400F. Turn dough out onto lightly-floured parchment paper and roll into large rectangle.
\item Spread pizza sauce over dough until 1 inch from edges. Top with pepperoni, desired fillings, and sprinkle with cheese.
\item Roll up dough and transfer to a baking sheet. Brush with egg and sprinkle with parmesan and parsley. Bake for 20-25 minutes until golden. Serve hot with extra pizza sauce.
\end{enumerate}
\end{minipage}\vspace{8mm}
\noindent\begin{minipage}[t]{\linewidth}%
{\Large\textbf{Whole Wheat Pizza Dough}} \label{whole-wheat-pizza-dough}\hfill\textit{Sue Dunn}\\
\noindent\begin{minipage}[t]{0.78\linewidth}%
\textbf{Ingredients}:\vspace{-3mm}
\begin{multicols}{2}
\begin{itemize}\setlength\itemsep{-1mm}
\item 2 1/2 cups bread flour
\item 1/3 cup whole wheat flour
\item 1 package active dry yeast
\item 1/2 tsp salt
\item 1 cup hot (115-125F) water
\item 1 tsp honey
\item 2 tsp olive oil
\end{itemize}
\end{multicols}
\end{minipage}
\noindent\begin{minipage}[t]{0.18\linewidth}
\centering \strut\vspace*{-\baselineskip}\newline
\includegraphics[width=0.9\linewidth]{/home/tim/Documents/projects/recipes/img/none.jpg}\\
\end{minipage}\vspace{3mm}
\textbf{Directions}:
\vspace{-3mm}\begin{enumerate}\setlength\itemsep{-1mm}
\item In a large bowl, combine 1 2/3 cups of the bread flour, whole wheat flour, yeast, and salt. Stir to blend dry ingredients thoroughly.
\item In a small bowl combine hot water, honey, and oil; stir to blend well. Add water mixture to flour mixture. Mix to blend, then beat at medium speed until smooth and elastic (about 5 minutes). Stir in about 1/3 cup more bread flour to make a soft dough.
\item Turn dough out onto a well-floured board or pastry cloth. Knead until dough is smooth under surface (5-10 minutes), adding just enough bread flour ( up to 1/3 cup) to prevent dough from being sticky.
\item Turn dough in a greased bowl. Cover with plastic wrap and a towel and let rise in a warm place until doubled in bulk (30-40 minutes). Punch dough down, cover with inverted bowl, and let rest 10 minutes; then use as directed in recipe.
\end{enumerate}
\end{minipage}\vspace{8mm}

{\newpage \LARGE \textbf{Pork}} \label{pork}\vspace{4mm}\\
\noindent\begin{minipage}[t]{\linewidth}%
{\Large\textbf{Barbecue Sauce}} \label{barbecue-sauce}\hfill\textit{Uncle Tim Dunn}\\
\textbf{Yield:} \textit{}\\
\noindent\begin{minipage}[t]{0.78\linewidth}%
\textbf{Ingredients}:\vspace{-3mm}
\begin{multicols}{2}
\begin{itemize}\setlength\itemsep{-1mm}
\item 1 cup water
\item 3 cups apple cider vinegar
\item 4 cups brown sugar
\item 3 Tbsp onion powder
\item 1 Tbsp paprika
\item 1 Tbsp black pepper
\item 1/2 Tbsp garlic powder
\item 1 cup ketchup
\item 2 cup BBQ sauce
\item 2 Tbsp Worchestershire sauce
\end{itemize}
\end{multicols}
\end{minipage}
\noindent\begin{minipage}[t]{0.18\linewidth}
\centering \strut\vspace*{-\baselineskip}\newline
\includegraphics[width=0.9\linewidth]{/home/tim/Documents/projects/recipes/img/barbecue-sauce.jpeg}\\
\end{minipage}\vspace{3mm}
\textbf{Directions}:
\vspace{-3mm}\begin{enumerate}\setlength\itemsep{-1mm}
\item Mix ingredients thoroughly.
\end{enumerate}
\end{minipage}\vspace{8mm}
\noindent\begin{minipage}[t]{\linewidth}%
{\Large\textbf{Rib Sauce}} \label{rib-sauce}\hfill\textit{Grandma Claire Dunn}\\
\noindent\begin{minipage}[t]{0.78\linewidth}%
\textbf{Ingredients}:\vspace{-3mm}
\begin{multicols}{2}
\begin{itemize}\setlength\itemsep{-1mm}
\item 1 cup OJ
\item 1 lemon
\item 1 cup brown sugar
\item 3/4 cup pancake syrup
\item celery
\item 1 onion
\item 1 Tbsp worcestershire sauce
\item 3 Tbsp vinegar
\item 1 Tbsp dry mustard
\item parsley
\item 1 cup ketchup
\end{itemize}
\end{multicols}
\end{minipage}
\noindent\begin{minipage}[t]{0.18\linewidth}
\centering \strut\vspace*{-\baselineskip}\newline
\includegraphics[width=0.9\linewidth]{/home/tim/Documents/projects/recipes/img/8046BA7D-6BD0-4D87-A0B7-3EC284E1A15A.jpg}\\
\end{minipage}\vspace{3mm}
\textbf{Directions}:
\vspace{-3mm}\begin{enumerate}\setlength\itemsep{-1mm}
\item Pour sauce on ribs or chicken. Cook at 325 for 2 hours. Turn off oven and let sit.
\end{enumerate}
\end{minipage}\vspace{8mm}
\noindent\begin{minipage}[t]{\linewidth}%
{\Large\textbf{Sausage Balls}} \label{sausage-balls}\hfill\textit{Sue Dunn}\\
\noindent\begin{minipage}[t]{0.78\linewidth}%
\textbf{Ingredients}:\vspace{-3mm}
\begin{multicols}{2}
\begin{itemize}\setlength\itemsep{-1mm}
\item 2 cups bisquick
\item 1 sausage
\item 12 oz sharp cheddar cheese
\end{itemize}
\end{multicols}
\end{minipage}
\noindent\begin{minipage}[t]{0.18\linewidth}
\centering \strut\vspace*{-\baselineskip}\newline
\includegraphics[width=0.9\linewidth]{/home/tim/Documents/projects/recipes/img/9A973527-B47D-4EBC-8158-F139E77AAE4E.jpg}\\
\end{minipage}\vspace{3mm}
\textbf{Directions}:
\vspace{-3mm}\begin{enumerate}\setlength\itemsep{-1mm}
\item Mix ingredients. Bake at 350 degrees for 10 minutes. Freeze. Reheat 10-15 minutes at 350 degrees.
\end{enumerate}
\end{minipage}\vspace{8mm}
\noindent\begin{minipage}[t]{\linewidth}%
{\Large\textbf{Southern Succor Rub}} \label{southern-succor-rub}\hfill\textit{Uncle Tim Dunn}\\
\textbf{Yield:} \textit{enough rub for one pork butt, with some left over}\\
\noindent\begin{minipage}[t]{0.78\linewidth}%
\textbf{Ingredients}:\vspace{-3mm}
\begin{multicols}{2}
\begin{itemize}\setlength\itemsep{-1mm}
\item 1/4 cup ground black pepper
\item 1/4 cup paprika
\item 1/4 cup Turbinado/'Sugar in the Raw'/Demerara Sugar
\item 2 Tbsp table salt
\item 2 tsp dry mustard
\item 1 tsp cayenne pepper
\end{itemize}
\end{multicols}
\end{minipage}
\noindent\begin{minipage}[t]{0.18\linewidth}
\centering \strut\vspace*{-\baselineskip}\newline
\includegraphics[width=0.9\linewidth]{/home/tim/Documents/projects/recipes/img/pork_rub.jpg}\\
\end{minipage}\vspace{3mm}
\textbf{Directions}:
\vspace{-3mm}\begin{enumerate}\setlength\itemsep{-1mm}
\item Mix ingredients thoroughly.
\end{enumerate}
\end{minipage}\vspace{8mm}

{\newpage \LARGE \textbf{Salads}} \label{salads}\vspace{4mm}\\
\noindent\begin{minipage}[t]{\linewidth}%
{\Large\textbf{Apple, Dried Cherry, and Walnut Salad Base}} \label{apple,-dried-cherry,-and-walnut-salad-base}\hfill\textit{Linda Garner}\\
\textit{``With maple dressing (next recipe)''}\\
\textbf{Yield:} \textit{serves 6}\\
\noindent\begin{minipage}[t]{0.78\linewidth}%
\textbf{Ingredients}:\vspace{-3mm}
\begin{multicols}{2}
\begin{itemize}\setlength\itemsep{-1mm}
\item 5oz bag mixed baby greens (10 cups lightly packed
\item 2 Granny Smith apples
\item 1/2 cup dried tart cherries
\item 1/2 cup chopped walnuts, toasted
\item 1/4 cup crumbled Maytag bleu cheese
\end{itemize}
\end{multicols}
\end{minipage}
\noindent\begin{minipage}[t]{0.18\linewidth}
\centering \strut\vspace*{-\baselineskip}\newline
\includegraphics[width=0.9\linewidth]{/home/tim/Documents/projects/recipes/img/apple-cherry-walnut-salad.jpg}\\
\end{minipage}\vspace{3mm}
\textbf{Directions}:
\vspace{-3mm}\begin{enumerate}\setlength\itemsep{-1mm}
\item Peel and core apples. Chop into matchstick-sized strips.
\item Toss greens, apples, cherries, and 1/4 cup walnuts in large bowl to combine. Toss with enough dressing to coat. Divide salad equally among all plates. Sprinkle remaining 1/4 cup walnuts and serve.
\end{enumerate}
\end{minipage}\vspace{8mm}
\noindent\begin{minipage}[t]{\linewidth}%
{\Large\textbf{Apple, Dried Cherry, and Walnut Salad Maple Dressing}} \label{apple,-dried-cherry,-and-walnut-salad-maple-dressing}\hfill\textit{Linda Garner}\\
\textit{``Dressing can be prepared up to 3 days ahead. Cover and refrigerate. Re-whisk before using.''}\\
\noindent\begin{minipage}[t]{0.78\linewidth}%
\textbf{Ingredients}:\vspace{-3mm}
\begin{multicols}{2}
\begin{itemize}\setlength\itemsep{-1mm}
\item 1/4 cup mayonnaise
\item 1/4 cup pure maple syrup
\item 3 Tbsp Champagne/white vinegar
\item 2 tsp sugar
\item 1/2 cup vegetable oil
\end{itemize}
\end{multicols}
\end{minipage}
\noindent\begin{minipage}[t]{0.18\linewidth}
\centering \strut\vspace*{-\baselineskip}\newline
\includegraphics[width=0.9\linewidth]{/home/tim/Documents/projects/recipes/img/none.jpg}\\
\end{minipage}\vspace{3mm}
\textbf{Directions}:
\vspace{-3mm}\begin{enumerate}\setlength\itemsep{-1mm}
\item Whisk mayonnaise, maple syrup, vinegar, and sugar in medium bowl to blend.
\item Gradually whisk in oil until mixture thickens slightly. Season to taste with salt and pepper.
\end{enumerate}
\end{minipage}\vspace{8mm}
\noindent\begin{minipage}[t]{\linewidth}%
{\Large\textbf{Candied Walnut and Goat Cheese Salad}} \label{candied-walnut-and-goat-cheese-salad}\hfill\textit{Sue Dunn}\\
\textbf{Yield:} \textit{8 servings}\\
\noindent\begin{minipage}[t]{0.78\linewidth}%
\textbf{Ingredients}:\vspace{-3mm}
\begin{multicols}{2}
\begin{itemize}\setlength\itemsep{-1mm}
\item candied walnuts
\item 1/2 cup balsamic vinegar
\item 1/4 cup pure maple syrup
\item 1/8 cup minced shallots
\item 1.5 cups olive oil
\item 2 tsp salt
\item 1 tsp pepper
\item 8 cups assorted baby field greens
\item 12 slices bacon, cooked and crumbled (optional)
\item 1/2 cup crumbled goat cheese
\end{itemize}
\end{multicols}
\end{minipage}
\noindent\begin{minipage}[t]{0.18\linewidth}
\centering \strut\vspace*{-\baselineskip}\newline
\includegraphics[width=0.9\linewidth]{/home/tim/Documents/projects/recipes/img/walnut-salad.jpeg}\\
\end{minipage}\vspace{3mm}
\textbf{Directions}:
\vspace{-3mm}\begin{enumerate}\setlength\itemsep{-1mm}
\item Make candied walnuts (see recipe).
\item Combine the vinegar, syrup, shallots, olive oil, salt, and pepper in a jar with a tight-fitting lid. Shake vigorously to mix; chill.
\item Mix the greens, bacon, cheese, and walnuts in a large salad bowl. Drizzle on vinaigrette, and toss lightly to coat.
\end{enumerate}
\end{minipage}\vspace{8mm}
\noindent\begin{minipage}[t]{\linewidth}%
{\Large\textbf{Candied Walnuts}} \label{candied-walnuts}\hfill\textit{Sue Dunn}\\
\noindent\begin{minipage}[t]{0.78\linewidth}%
\textbf{Ingredients}:\vspace{-3mm}
\begin{multicols}{2}
\begin{itemize}\setlength\itemsep{-1mm}
\item 1/4 cup packed brown sugar
\item 1 cup walnuts
\item 2 Tbsp water
\end{itemize}
\end{multicols}
\end{minipage}
\noindent\begin{minipage}[t]{0.18\linewidth}
\centering \strut\vspace*{-\baselineskip}\newline
\includegraphics[width=0.9\linewidth]{/home/tim/Documents/projects/recipes/img/candied-walnuts.jpeg}\\
\end{minipage}\vspace{3mm}
\textbf{Directions}:
\vspace{-3mm}\begin{enumerate}\setlength\itemsep{-1mm}
\item Cook the brown sugar in a large heavy skillet over medium-high heat until melted.
\item Add the walnuts and water. Cook for 10 minutes or until water evaporates, stirring constantly. Spread the walnuts on a sheet pan to dry.
\end{enumerate}
\end{minipage}\vspace{8mm}
\noindent\begin{minipage}[t]{\linewidth}%
{\Large\textbf{Dill Chicken Salad}} \label{dill-chicken-salad}\hfill\textit{Sue Dunn}\\
\noindent\begin{minipage}[t]{0.78\linewidth}%
\textbf{Ingredients}:\vspace{-3mm}
\begin{multicols}{2}
\begin{itemize}\setlength\itemsep{-1mm}
\item 2 lbs chicken breasts
\item 1/2 cup diced celery (1-2 stalks)
\item 1/2 onion
\item 1 cup plain Greek yogurt
\item 1/4 cup mayonnaise
\item 1/4 cup minched fresh dill
\item 1 green onion, minced
\item 1 Tbsp Dijon mustard
\item 1 tsp apple cider vinegar
\item 1/2 tsp salt
\item 1/4 tsp pepper
\item lemon zest
\item bread, wraps, or croissants (for serving)
\end{itemize}
\end{multicols}
\end{minipage}
\noindent\begin{minipage}[t]{0.18\linewidth}
\centering \strut\vspace*{-\baselineskip}\newline
\includegraphics[width=0.9\linewidth]{/home/tim/Documents/projects/recipes/img/none.jpg}\\
\end{minipage}\vspace{3mm}
\textbf{Directions}:
\vspace{-3mm}\begin{enumerate}\setlength\itemsep{-1mm}
\item In a large bowl, combine cooked chicken and celery. In a small bowl, combine yogurt, mayo, dill, green onion, Dijon mustard, apple cider vinegar, salt, and pepper. Taste and season with more salt and pepper as desired.
\item Stir dressing into chicken and celery until everything is coated evenly.
\item Serve immediately with bread, wraps, or croissants. The sauce will separate slightly when stored in the fridge.
\item Bring to a boil. Mix together the cornstarch and water before adding to the pan, stirring quickly, until sauce has thickened slightly. Reduce heat and simmer gently for another minute to thicken sauce. Return chicken to the skillet, and sprinkle with extra herbs if desired. Serve immediately.
\end{enumerate}
\end{minipage}\vspace{8mm}
\noindent\begin{minipage}[t]{\linewidth}%
{\Large\textbf{French Potato Salad}} \label{french-potato-salad}\hfill\textit{Sue Dunn}\\
\textbf{Yield:} \textit{4-6 servings}\\
\noindent\begin{minipage}[t]{0.78\linewidth}%
\textbf{Ingredients}:\vspace{-3mm}
\begin{multicols}{2}
\begin{itemize}\setlength\itemsep{-1mm}
\item 1 lb small white boiling potatoes
\item 1 lb small red boiling potatoes
\item 2 Tbsp dry white wine
\item 2 Tbsp chicken stock
\item 3 Tbsp vinegar
\item 1/2 tsp Dijon mustard
\item 2 tsp kosher salt
\item 3/4 tsp ground black pepper
\item 10 Tbsp olive oil
\item 1/4 cup minced scallions
\item 2 Tbsp minced fresh dill
\item 2 Tbsp minced parsley
\item 2 Tbsp minced basil
\end{itemize}
\end{multicols}
\end{minipage}
\noindent\begin{minipage}[t]{0.18\linewidth}
\centering \strut\vspace*{-\baselineskip}\newline
\includegraphics[width=0.9\linewidth]{/home/tim/Documents/projects/recipes/img/potato_salad.jpg}\\
\end{minipage}\vspace{3mm}
\textbf{Directions}:
\vspace{-3mm}\begin{enumerate}\setlength\itemsep{-1mm}
\item Cook the white and red potatoes in a large boiling pot of salted water for 20-30 minutes, until cooked through. Drain and place a towel over them, allowing them to steam for 10 more minutes. As soon as you can handle them, cut in half (or smaller) and place in a medium bowl. Toss gently with the wine and chicken stock. Alow the liquids to soak into the warm potatoes before proceeding.
\item Combine the vinegar, mustard, 1/2 tsp salt, 1/4 tsp pepper and slowly whisk in the olive oil to make an emulsion. Add the vinaigrette to the potatoes. Add the scallions, dill, parsley, basil, 1 1/2 tsp salt, and 1/2 tsp pepper and toss. Serve warm or at room temperature.
\end{enumerate}
\end{minipage}\vspace{8mm}
\noindent\begin{minipage}[t]{\linewidth}%
{\Large\textbf{Oriental Chicken Salad}} \label{oriental-chicken-salad}\hfill\textit{Lynn Neff}\\
\noindent\begin{minipage}[t]{0.78\linewidth}%
\textbf{Ingredients}:\vspace{-3mm}
\begin{multicols}{2}
\begin{itemize}\setlength\itemsep{-1mm}
\item 1 cooked chicken, shredded
\item 1/2 package wonton shells, cut in 1/4 inch strips and fried in peanut oil
\item 2 heads lettuce, shredded
\item 8 green onions, sliced
\item 4 tsp sliced almonds, toasted
\item 4 Tbsp sesame seeds
\item 1/2 cup white vinegar
\item 1/2 cup ketchup
\item 1/2 cup water
\item 12 Tbsp sugar
\item 2 tsp soy
\end{itemize}
\end{multicols}
\end{minipage}
\noindent\begin{minipage}[t]{0.18\linewidth}
\centering \strut\vspace*{-\baselineskip}\newline
\includegraphics[width=0.9\linewidth]{/home/tim/Documents/projects/recipes/img/208306E3-ECAD-4693-8AA9-7461D5210B63.jpg}\\
\end{minipage}\vspace{3mm}
\textbf{Directions}:
\vspace{-3mm}\begin{enumerate}\setlength\itemsep{-1mm}
\item Tap and Hold image
\item Touch 'Save Image' to save the image to your iPad
\item Tap here to open app importer.
\item Once the app opens and imports the recipe, upload the picture you saved from the email (iPad app only)
\end{enumerate}
\end{minipage}\vspace{8mm}
\noindent\begin{minipage}[t]{\linewidth}%
{\Large\textbf{Oriental Coleslaw Salad Base}} \label{oriental-coleslaw-salad-base}\hfill\textit{Sue Dunn}\\
\noindent\begin{minipage}[t]{0.78\linewidth}%
\textbf{Ingredients}:\vspace{-3mm}
\begin{multicols}{2}
\begin{itemize}\setlength\itemsep{-1mm}
\item 1 (16 oz) pkg cole slaw mix
\item 1-2 bunches of scallions, sliced
\item 1 pkg sunflower seeds
\item 1 pkg sliced almonds, toasted
\item 2 pkgs Ramen noodles, broken up
\end{itemize}
\end{multicols}
\end{minipage}
\noindent\begin{minipage}[t]{0.18\linewidth}
\centering \strut\vspace*{-\baselineskip}\newline
\includegraphics[width=0.9\linewidth]{/home/tim/Documents/projects/recipes/img/oriental-coleslaw-salad.jpeg}\\
\end{minipage}\vspace{3mm}
\textbf{Directions}:
\vspace{-3mm}\begin{enumerate}\setlength\itemsep{-1mm}
\item Mix all of the above ingredients.
\item Top with dressing (below).
\end{enumerate}
\end{minipage}\vspace{8mm}
\noindent\begin{minipage}[t]{\linewidth}%
{\Large\textbf{Oriental Coleslaw Salad Dressing}} \label{oriental-coleslaw-salad-dressing}\hfill\textit{Sue Dunn}\\
\textit{``Use oriental seasoning mix, not the high-sodium stuff that comes with the Ramen noodles.''}\\
\noindent\begin{minipage}[t]{0.78\linewidth}%
\textbf{Ingredients}:\vspace{-3mm}
\begin{multicols}{2}
\begin{itemize}\setlength\itemsep{-1mm}
\item 1/3 cup sugar
\item 1/4 cup white wine vinegar
\item 3/4 cup oil
\item 1 pkg Oriental seasoning mix
\item 2 pkgs Ramen noodles, broken up
\end{itemize}
\end{multicols}
\end{minipage}
\noindent\begin{minipage}[t]{0.18\linewidth}
\centering \strut\vspace*{-\baselineskip}\newline
\includegraphics[width=0.9\linewidth]{/home/tim/Documents/projects/recipes/img/coleslaw-salad-dressing.jpeg}\\
\end{minipage}\vspace{3mm}
\textbf{Directions}:
\vspace{-3mm}\begin{enumerate}\setlength\itemsep{-1mm}
\item Mix all of the above ingredients.
\item Just before serving, thoroughly mix the dressing with the salad mixture.
\end{enumerate}
\end{minipage}\vspace{8mm}
\noindent\begin{minipage}[t]{\linewidth}%
{\Large\textbf{Strawberry Salad with Poppy Seed Dressing}} \label{strawberry-salad-with-poppy-seed-dressing}\hfill\textit{Sue Dunn}\\
\textbf{Yield:} \textit{Serves 6}\\
\noindent\begin{minipage}[t]{0.78\linewidth}%
\textbf{Ingredients}:\vspace{-3mm}
\begin{multicols}{2}
\begin{itemize}\setlength\itemsep{-1mm}
\item 3 Tbsp sugar
\item 3 Tbsp mayonnaise
\item 2 Tbsp milk
\item 1 Tbsp poppy seeds
\item 1 Tbsp white wine vinegar
\item 1 bag (10 oz) romaine lettuce
\item 1 cup strawberries, sliced
\item 2 Tbsp toasted slivered almonds
\end{itemize}
\end{multicols}
\end{minipage}
\noindent\begin{minipage}[t]{0.18\linewidth}
\centering \strut\vspace*{-\baselineskip}\newline
\includegraphics[width=0.9\linewidth]{/home/tim/Documents/projects/recipes/img/strawberry-salad.jpeg}\\
\end{minipage}\vspace{3mm}
\textbf{Directions}:
\vspace{-3mm}\begin{enumerate}\setlength\itemsep{-1mm}
\item Combine first 5 ingredients in a small bowl, stirring with a whisk. Place lettuce in a large bowl; add strawberries and almonds, tossing to combine. Divide salad among 6 plates. Drizzle 1 Tbsp dressing over each serving.
\end{enumerate}
\end{minipage}\vspace{8mm}

{\newpage \LARGE \textbf{Seafood}} \label{seafood}\vspace{4mm}\\
\noindent\begin{minipage}[t]{\linewidth}%
{\Large\textbf{Bourbon Bacon Scallops}} \label{bourbon-bacon-scallops}\hfill\textit{}\\
\noindent\begin{minipage}[t]{0.78\linewidth}%
\textbf{Ingredients}:\vspace{-3mm}
\begin{multicols}{2}
\begin{itemize}\setlength\itemsep{-1mm}
\item 6 slices bacon (4-5 oz)
\item 3 Tbsp minced scallions (green onions)
\item 2 Tbsp bourbon
\item 2 Tbsp maple syrup
\item 1 Tbsp soy sauce
\item 1 Tbsp Dijon mustard
\item 1/4 tsp fresh ground black pepper or 1/4 teaspoon fresh ground tricolor pepper 24 large sea scallops cooking spray (about 1 1/2 pounds)
\item 24 Large sea scallops (about 1 1/2 pounds)
\item cooking spray
\item 4 metal skewers or 4 water-soaked bamboo skewers (12 inch)
\end{itemize}
\end{multicols}
\end{minipage}
\noindent\begin{minipage}[t]{0.18\linewidth}
\centering \strut\vspace*{-\baselineskip}\newline
\includegraphics[width=0.9\linewidth]{/home/tim/Documents/projects/recipes/img/7BA081E5-C2CE-4A48-A94B-73154236B71B.jpg}\\
\end{minipage}\vspace{3mm}
\textbf{Directions}:
\vspace{-3mm}\begin{enumerate}\setlength\itemsep{-1mm}
\item 1 Heat a skillet over high temperature and saute the bacon for 4-5 minutes, until limp and partially browned; remove from skillet, drain, and set aside to cool. 
\item 2 In a bowl, combine the green onions, bourbon, maple syrup, soy sauce, mustard, and pepper, and stir well; remove about 2 tablespoons of marinade to another container and set aside. 
\item 3 Add the sea scallops to the marinade in the bowl and toss gently to coat. 
\item 4 Cover and place in refrigerator to marinate for 1 hour, stirring occasionally. 
\item 5 Preheat your oven broiler; pan spray a broiler pan. 
\item 6 Cut the partially cooked bacon strips into 4 sections apiece. 
\item 7 Remove the scallops from the marinade (reserve marinade) and wrap a piece of cut bacon around each scallop - if the scallops are very large, they might only reach halfway around. 
\item 8 Thread the wrapped scallops onto the skewers (going through each end of bacon strips if they only reach halfway around), making sure to leave space between each scallop so that the bacon will cook well. 
\item 9 Place the completed skewers on the pan-sprayed boiler pan and broil for 8 minutes or until the bacon is crisp and the scallops are opaque, occasionally basting with the marinade used with the scallops (how long they need to cook depends on the size of the scallops). 
\item 10 Remove skewers from broiler, place them on a serving platter, and brush or drizzle with the reserved marinade (that which was not combined with the scallops in the refrigerator). 
\item 11 Note: these can also be cooked on the grill if you watch them carefully so that the scallops do not overcook.
\end{enumerate}
\end{minipage}\vspace{8mm}
\noindent\begin{minipage}[t]{\linewidth}%
{\Large\textbf{Coconut Shrimp}} \label{coconut-shrimp}\hfill\textit{Sue Dunn}\\
\noindent\begin{minipage}[t]{0.78\linewidth}%
\textbf{Ingredients}:\vspace{-3mm}
\begin{multicols}{2}
\begin{itemize}\setlength\itemsep{-1mm}
\item 2 egg whites
\item 3/4 cup all-purpose flour
\item 6 oz beer
\item 1 1/2 tsp baking powder
\item 1/4 tsp table salt
\item 2 cups sweetened coconut flakes
\item 24 large uncooked shrimp, peeled and deveined (leave tails on)
\end{itemize}
\end{multicols}
\end{minipage}
\noindent\begin{minipage}[t]{0.18\linewidth}
\centering \strut\vspace*{-\baselineskip}\newline
\includegraphics[width=0.9\linewidth]{/home/tim/Documents/projects/recipes/img/623F2F31-CC8C-40B2-9D59-DED8927C64FD.jpg}\\
\end{minipage}\vspace{3mm}
\textbf{Directions}:
\vspace{-3mm}\begin{enumerate}\setlength\itemsep{-1mm}
\item Preheat oven to 450F. Coat a large baking sheet with cooking spray.
\item In a medium bowl, whisk together egg whites, 1/2 cup of flour, beer, baking powder and salt. Place remaining 1/4 cup of flour and coconut in two separate shallow bowls.
\item Holding shrimp by their tails, dredge each shrimp in flour and shake off any excess. Dip flour-coated shrimp into egg batter and allow excess to drip off. Roll shrimp in coconut and turn to coat both sides (press coconut onto shrimp to make it stick).
\item Transfer shrimp to prepared baking sheet and spray surface of shrimp with cooking spray.
\item Bake until coconut is golden brown and shrimp are bright pink and cooked through (10-12 minutes).
\end{enumerate}
\end{minipage}\vspace{8mm}
\noindent\begin{minipage}[t]{\linewidth}%
{\Large\textbf{Crab Casserole}} \label{crab-casserole}\hfill\textit{Mom}\\
\noindent\begin{minipage}[t]{0.78\linewidth}%
\textbf{Ingredients}:\vspace{-3mm}
\begin{multicols}{2}
\begin{itemize}\setlength\itemsep{-1mm}
\item 1 Tbsp margerine
\item 2 scallion, chopped
\item 1 tsp cornstarch
\item 1/2 cup half-and-half cream
\item 2 tsp lemon juice
\item 1/8 tsp salt
\item 1/8 tsp pepper
\item 2/3 cup cooked crabmeat
\item 4 mushrooms (Sliced)
\item 2 Tbsp dry bread crumbs
\item 1 1/3 Tbsp grated Parmesan cheese
\item 1 Tbsp melted margerine
\end{itemize}
\end{multicols}
\end{minipage}
\noindent\begin{minipage}[t]{0.18\linewidth}
\centering \strut\vspace*{-\baselineskip}\newline
\includegraphics[width=0.9\linewidth]{/home/tim/Documents/projects/recipes/img/AE88C0C2-4A05-4022-A9F4-C438F2053528.jpg}\\
\end{minipage}\vspace{3mm}
\textbf{Directions}:
\vspace{-3mm}\begin{enumerate}\setlength\itemsep{-1mm}
\item Put margarine and onion in a bowl. Microwave uncovered until crisp.
\item Mix in cornstarch, stir in half and half, lemon juice, salt, and pepper. Microwave uncovered for one minute. Stir, and microwave for another minute. Stir in the crab and mushroom.
\item Mix topping, sprinkle on top, cover loosely, and microwave until hot.
\end{enumerate}
\end{minipage}\vspace{8mm}
\noindent\begin{minipage}[t]{\linewidth}%
{\Large\textbf{Honey Balsamic Salmon}} \label{honey-balsamic-salmon}\hfill\textit{}\\
\noindent\begin{minipage}[t]{0.78\linewidth}%
\textbf{Ingredients}:\vspace{-3mm}
\begin{multicols}{2}
\begin{itemize}\setlength\itemsep{-1mm}
\item 1/4 tsp olive oil
\item 1/2 cup leek(s), julienne-cut (about 1 small)
\item 1 1/2 lb salmon fillets, with or without skin, four 6-oz. pieces
\item 1/2 tsp table salt, divided
\item 1/4 tsp black pepper, divided
\item 1/2 cup balsamic vinegar
\item 1 Tbsp honey
\end{itemize}
\end{multicols}
\end{minipage}
\noindent\begin{minipage}[t]{0.18\linewidth}
\centering \strut\vspace*{-\baselineskip}\newline
\includegraphics[width=0.9\linewidth]{/home/tim/Documents/projects/recipes/img/53AC6E19-DE06-4A08-87E0-2EBE21E19FB6.jpg}\\
\end{minipage}\vspace{3mm}
\textbf{Directions}:
\vspace{-3mm}\begin{enumerate}\setlength\itemsep{-1mm}
\item Coat a large nonstick skillet with cooking spray; add oil. Place over medium-high heat; add leaks and saute 3 to 4 minutes or until soft. Remove from pan and set aside.
\item Sprinkle fish with 1/4 tsp salt and 1/8 tsp pepper. Add fish to pan; cook 3 to 4 minutes on each side or until lightly browned and fish flakes easily when tested with a fork. Remove from pan; set aside, and keep warm.
\item Add vinegar, honey, 1/4 tsp salt, and 1/8 tsp pepper to pan. Cook over medium-high heat 3-4 minutes or until reduced by half. Divide leaks evenly over fish; drizzle with sauce. 
\end{enumerate}
\end{minipage}\vspace{8mm}
\noindent\begin{minipage}[t]{\linewidth}%
{\Large\textbf{Pan-Seared Scallops}} \label{pan-seared-scallops}\hfill\textit{}\\
\textbf{Yield:} \textit{4 servings}\\
\noindent\begin{minipage}[t]{0.78\linewidth}%
\textbf{Ingredients}:\vspace{-3mm}
\begin{multicols}{2}
\begin{itemize}\setlength\itemsep{-1mm}
\item 1 lb fresh sea scallops
\item salt and pepper
\item pan searing flour
\item 1 Tbsp olive oil
\item 2 Tbsp shallot-thyme finishing butter
\end{itemize}
\end{multicols}
\end{minipage}
\noindent\begin{minipage}[t]{0.18\linewidth}
\centering \strut\vspace*{-\baselineskip}\newline
\includegraphics[width=0.9\linewidth]{/home/tim/Documents/projects/recipes/img/7A83B49C-ECC8-4B52-B5AD-C50C8B011C37.jpg}\\
\end{minipage}\vspace{3mm}
\textbf{Directions}:
\vspace{-3mm}\begin{enumerate}\setlength\itemsep{-1mm}
\item Season scallops with salt and pepper; dust with pan-searing flour. Heat olive oil in pan on medium-high; add scallops. Sear until golden brown (2-3 minutes). Turn scallops.
\item Reduce heat to medium-low. Cook for 2-3 minutes, until internal temperature reaches 120F.
\item Add butter and baste scallops for 1-2 minutes until internal temp reaches 130F. Let scallops rest for 2 minutes.
\end{enumerate}
\end{minipage}\vspace{8mm}
\noindent\begin{minipage}[t]{\linewidth}%
{\Large\textbf{Shrimp Bisque}} \label{shrimp-bisque}\hfill\textit{Sue Dunn}\\
\noindent\begin{minipage}[t]{0.78\linewidth}%
\textbf{Ingredients}:\vspace{-3mm}
\begin{multicols}{2}
\begin{itemize}\setlength\itemsep{-1mm}
\item 2 lbs shrimp, peeled and deveined
\item 1 shallot
\item 4 cups seafood stock
\item 3 Tbsp olive oil
\item 2 cups chopped leeks (3 leeks)
\item 3 cloves garlic
\item pinch of cayenne pepper
\item 1/2 cup cooking sherry
\item 4 Tbsp unsalted butter
\item 1/4 cup all-purpose flour
\item 2 cups half-and-half
\item 1/3 cup tomato paste
\item 2 tsp salt
\item 1 tsp pepper
\end{itemize}
\end{multicols}
\end{minipage}
\noindent\begin{minipage}[t]{0.18\linewidth}
\centering \strut\vspace*{-\baselineskip}\newline
\includegraphics[width=0.9\linewidth]{/home/tim/Documents/projects/recipes/img/none.jpg}\\
\end{minipage}\vspace{3mm}
\textbf{Directions}:
\vspace{-3mm}\begin{enumerate}\setlength\itemsep{-1mm}
\item Set aside a few shrimp for chopping so that final soup contains large chunks. Place the remaining shrimp and seafood stock in a saucepan and simmer for 15 minutes. Strain and reserve the stock. Add enough water to make 3 3/4 cups. Meanwhile, heat the olive oil in a large pot. Add the leeks and cook them for 10 minutes over medium-low heat, or until the leeks are tender but not browned. Add the garlic and cook 1 more minute.
\item Add the cayenne pepper and shrimp and cook over medium-low heat for 3 minutes, stirring occasionally. Add the sherry and cook for 3 more minutes. Transfer the shrimp and leeks to a food processor and process until finely pureed.
\item In the same pot, melt the butter. Add the flour and cook over medium-low heat for 1 minute, stirring with a wooden spoon. Add the half-and-half and cook, stirring with a whisk, until thickened (about 3 minutes).
\item Chop the shrimp set aside in Step 1. Stir in the chopped shrimp, pureed shrimp, stock, tomato paste, salt, and pepper. Heat gently until got but not bioling.
\end{enumerate}
\end{minipage}\vspace{8mm}

{\newpage \LARGE \textbf{Sides}} \label{sides}\vspace{4mm}\\
\noindent\begin{minipage}[t]{\linewidth}%
{\Large\textbf{Baked Asparagus}} \label{baked-asparagus}\hfill\textit{Sue Dunn}\\
\textbf{Yield:} \textit{serves 4}\\
\noindent\begin{minipage}[t]{0.78\linewidth}%
\textbf{Ingredients}:\vspace{-3mm}
\begin{multicols}{2}
\begin{itemize}\setlength\itemsep{-1mm}
\item 1 lb asparagus
\item 2 Tbsp olive oil
\item 1/2 tsp salt
\item 1/8 tsp black pepper
\item 1/2 cup grated parmesan cheese
\end{itemize}
\end{multicols}
\end{minipage}
\noindent\begin{minipage}[t]{0.18\linewidth}
\centering \strut\vspace*{-\baselineskip}\newline
\includegraphics[width=0.9\linewidth]{/home/tim/Documents/projects/recipes/img/none.jpg}\\
\end{minipage}\vspace{3mm}
\textbf{Directions}:
\vspace{-3mm}\begin{enumerate}\setlength\itemsep{-1mm}
\item Prep the asparagus: Preheat the oven to 400F. Break or cut off the woody ends of the asparagus spears.
\item Toss with olive oil, salt, pepper, parmesan. Arrange the asparagus on a foil-lined baking sheet and coat with olive oil. Sprinkle with salt, pepper, and parmesan.
\item Bake at 400F until the cheese begins to brown, 8-10 minutes.
\end{enumerate}
\end{minipage}\vspace{8mm}
\noindent\begin{minipage}[t]{\linewidth}%
{\Large\textbf{Blueberry Muffins}} \label{blueberry-muffins}\hfill\textit{}\\
\noindent\begin{minipage}[t]{0.78\linewidth}%
\textbf{Ingredients}:\vspace{-3mm}
\begin{multicols}{2}
\begin{itemize}\setlength\itemsep{-1mm}
\item 3 cups flour
\item 1 cup sugar
\item 4 tsp baking powder
\item 1 tsp salt
\item 2 eggs, lightly beaten
\item 1/2 cup oil
\item 1 cup milk
\item 1 1/2 cups blueberries
\end{itemize}
\end{multicols}
\end{minipage}
\noindent\begin{minipage}[t]{0.18\linewidth}
\centering \strut\vspace*{-\baselineskip}\newline
\includegraphics[width=0.9\linewidth]{/home/tim/Documents/projects/recipes/img/C369CD3B-128B-453F-8459-2D584B138E81.jpg}\\
\end{minipage}\vspace{3mm}
\textbf{Directions}:
\vspace{-3mm}\begin{enumerate}\setlength\itemsep{-1mm}
\item Preheat oven to 400F.
\item Mix together flour, sugar, baking powder and salt. Combine eggs and milk and stir into dry ingredients until moistened.
\item Add and stir the berries through the mixture. Spoon into muffin tins about half full.
\item Bake for 20 minutes.
\end{enumerate}
\end{minipage}\vspace{8mm}
\noindent\begin{minipage}[t]{\linewidth}%
{\Large\textbf{Mashed Sweet Potatoes}} \label{mashed-sweet-potatoes}\hfill\textit{}\\
\noindent\begin{minipage}[t]{0.78\linewidth}%
\textbf{Ingredients}:\vspace{-3mm}
\begin{multicols}{2}
\begin{itemize}\setlength\itemsep{-1mm}
\item 5 medium (4 lb) sweet potatoes, peeled, 1 in. dice (10 cups)
\item 1 1/4 cups whole milk
\item 1 stick unsalted butter
\item 1/3 cup brown sugar
\item 1 1/2 tsp ground cinnamon
\item salt and pepper to taste
\end{itemize}
\end{multicols}
\end{minipage}
\noindent\begin{minipage}[t]{0.18\linewidth}
\centering \strut\vspace*{-\baselineskip}\newline
\includegraphics[width=0.9\linewidth]{/home/tim/Documents/projects/recipes/img/2E22A92B-794C-45CB-9160-95C5F86B3074.jpg}\\
\end{minipage}\vspace{3mm}
\textbf{Directions}:
\vspace{-3mm}\begin{enumerate}\setlength\itemsep{-1mm}
\item Place potatoes in stock pot; cover with cold water. Bring to boil on high; reduce heat to medium. Cover and simmer potatoes about 15 mins, until tender when pierced with knife tip. Drain; return to stockpot. Heat on low 3-5 mins, tossing gently, to remove excess moisture.
\item Combine potatoes, milk, butter, brown sugar, and cinnamon in stockpot. Mash with with handheld potato masher. Season to taste with salt and pepper. 
\end{enumerate}
\end{minipage}\vspace{8mm}
\noindent\begin{minipage}[t]{\linewidth}%
{\Large\textbf{Twice Baked Potatoes}} \label{twice-baked-potatoes}\hfill\textit{Sue Dunn}\\
\noindent\begin{minipage}[t]{0.78\linewidth}%
\textbf{Ingredients}:\vspace{-3mm}
\begin{multicols}{2}
\begin{itemize}\setlength\itemsep{-1mm}
\item 4 large baking potatoes
\item 8 slices bacon
\item 1 cup sour cream
\item 1/2 cup milk
\item 1/2 tsp salt
\item 1/2 tsp pepper
\item 1 cup cheddar, shredded
\item 8 green onions, sliced
\end{itemize}
\end{multicols}
\end{minipage}
\noindent\begin{minipage}[t]{0.18\linewidth}
\centering \strut\vspace*{-\baselineskip}\newline
\includegraphics[width=0.9\linewidth]{/home/tim/Documents/projects/recipes/img/none.jpg}\\
\end{minipage}\vspace{3mm}
\textbf{Directions}:
\vspace{-3mm}\begin{enumerate}\setlength\itemsep{-1mm}
\item Preheat the oven to 350F. Bake potatoes in preheated oven until easily pierced with fork, about 1 hour. Cool for 10 minutes, leaving oven on.
\item Meanwhile, cook bacon in a large skillet over medium heat until evenly browned. Drain, crumble, and set aside.
\item Slice potatoes in half lengthwise and scoop flesh into a large bowl, saving the skins. Mix in sour cream, milk, salt, pepper, half the cheddar, and half the green onions. Blend with a handheld mixer until creamy. Spoon back into potatos skins and top with remaining cheese, green onions, and bacon.
\item Bake in oven until filling is hot and cheese is melted, about 15 minutes.
\end{enumerate}
\end{minipage}\vspace{8mm}

{\newpage \LARGE \textbf{Soups}} \label{soups}\vspace{4mm}\\
\noindent\begin{minipage}[t]{\linewidth}%
{\Large\textbf{Butternut Squash Bisque}} \label{butternut-squash-bisque}\hfill\textit{Sue Dunn}\\
\textbf{Yield:} \textit{Serves 6}\\
\noindent\begin{minipage}[t]{0.78\linewidth}%
\textbf{Ingredients}:\vspace{-3mm}
\begin{multicols}{2}
\begin{itemize}\setlength\itemsep{-1mm}
\item 2 tsp unsalted butter
\item 1 1/2 cup onion, chopped
\item 1 1/2 tsp kosher salt
\item 2 tsp minced garlic
\item 1 pinch nutmeg
\item 1 pinch cayenne pepper
\item 4 lbs uncooked butternut squash, cubed
\item 4 cups chicken/vegetable broth
\item 3 Tbsp plain Greek yogurt
\item 1 Tbsp packed light brown sugar
\item 2 tsp fresh sage, chopped
\item 1/4 tsp black pepper
\item 1 1/2 tsp cinnamon
\end{itemize}
\end{multicols}
\end{minipage}
\noindent\begin{minipage}[t]{0.18\linewidth}
\centering \strut\vspace*{-\baselineskip}\newline
\includegraphics[width=0.9\linewidth]{/home/tim/Documents/projects/recipes/img/squash_bisque.jpg}\\
\end{minipage}\vspace{3mm}
\textbf{Directions}:
\vspace{-3mm}\begin{enumerate}\setlength\itemsep{-1mm}
\item Heat butter in soup pot over medium heat. Add onion and salt; cook, stirring occasionally, until onion is softened (5-7 minutes).
\item Add garlic, nutmeg and cayenne; stir and cook 1 minute. Add squash and broth; bring to a boil over high heat.
\item Reduce heat to low and simmer, uncovered, until squash is soft, about 30 minutes; stir in yogurt and sugar.
\item Remove from heat and puree soup in pot. Serve garnished with fresh sage, black pepper, and cinnamon.
\end{enumerate}
\end{minipage}\vspace{8mm}
\noindent\begin{minipage}[t]{\linewidth}%
{\Large\textbf{Chicken Taco Stew}} \label{chicken-taco-stew}\hfill\textit{Sue Dunn}\\
\noindent\begin{minipage}[t]{0.78\linewidth}%
\textbf{Ingredients}:\vspace{-3mm}
\begin{multicols}{2}
\begin{itemize}\setlength\itemsep{-1mm}
\item 1 onion, chopped
\item 1 (16 oz) can cannellini beans, drained and rinsed
\item 1 (16 oz) can black beans, drained and rinsed
\item 1 (16 oz) can sweet corn, drained
\item 3 (10 oz) cans of diced tomatoes with green chillies
\item 1 packet taco seasoning
\item 3 chicken breasts, cut lengthwise in half
\item 2 avocados
\item sour cream
\end{itemize}
\end{multicols}
\end{minipage}
\noindent\begin{minipage}[t]{0.18\linewidth}
\centering \strut\vspace*{-\baselineskip}\newline
\includegraphics[width=0.9\linewidth]{/home/tim/Documents/projects/recipes/img/taco_stew.jpg}\\
\end{minipage}\vspace{3mm}
\textbf{Directions}:
\vspace{-3mm}\begin{enumerate}\setlength\itemsep{-1mm}
\item Mix everything together in a slow cooker except chicken. Lay chicken on top and cover. Cook on low for 6-8 hours or on high for 4 hours.
\item 30 minutes before serving, remove chicken and shred. Return shredded chicken to slow cooker and stir in. Serve with sour cream and avocado.
\end{enumerate}
\end{minipage}\vspace{8mm}
\noindent\begin{minipage}[t]{\linewidth}%
{\Large\textbf{Chili}} \label{chili}\hfill\textit{Doug}\\
\noindent\begin{minipage}[t]{0.78\linewidth}%
\textbf{Ingredients}:\vspace{-3mm}
\begin{multicols}{2}
\begin{itemize}\setlength\itemsep{-1mm}
\item 2 (28 oz) cans crushed tomatoes
\item 1 lb ground beef, cooked and drained
\item 1 red onion, chopped
\item 2 stalks celery, chopped
\item 1 green pepper, chopped
\item 1 jalapeno, chopped (optional)
\item 1 1/2 Tbsp chili powder
\item 4 cloves garlic, chopped
\item 1 tsp ground cumin
\item 1/2 tsp red pepper
\item 1/2 tsp sugar
\item 1/2 tsp ground mustard
\item 1/2 tsp tabasco sauce
\item 1 tsp Worchestershire sauce
\item 1 can black beans
\item 1/2 cup sherry
\item beer or water for consistency
\item salt to taste
\end{itemize}
\end{multicols}
\end{minipage}
\noindent\begin{minipage}[t]{0.18\linewidth}
\centering \strut\vspace*{-\baselineskip}\newline
\includegraphics[width=0.9\linewidth]{/home/tim/Documents/projects/recipes/img/none.jpg}\\
\end{minipage}\vspace{3mm}
\textbf{Directions}:
\vspace{-3mm}\begin{enumerate}\setlength\itemsep{-1mm}
\item Starting with the tomatoes and ground beef, add the uncooked veggies and spices. Simmer for two hours.
\item Add the black beans and sherry, leaving just enough time to heat. Add salt and sugar for taste and beer for consistency as needed.
\end{enumerate}
\end{minipage}\vspace{8mm}
\noindent\begin{minipage}[t]{\linewidth}%
{\Large\textbf{Corn Chowder}} \label{corn-chowder}\hfill\textit{Sue Dunn}\\
\textbf{Yield:} \textit{18 cups}\\
\noindent\begin{minipage}[t]{0.78\linewidth}%
\textbf{Ingredients}:\vspace{-3mm}
\begin{multicols}{2}
\begin{itemize}\setlength\itemsep{-1mm}
\item 1 pkg (4 oz) diced pancetta
\item 1 cup chopped onion
\item 1 cup diced celery
\item 3 cloves garlic, minced
\item 6 Tbsp salted butter
\item 1/2 cup all-purpose flour
\item 2 (32 oz) cartons chicken stock
\item 8 ears fresh corn, shucked, kernels removed
\item 1 red bell pepper, diced
\item 1 large russet potato, peeled, diced
\item 1 Tbsp chopped thyme
\item 1 1/4 tsp sea salt
\item black pepper
\item 2 (8 oz) cartons light cream
\item 1 tsp tabasco sauce
\item 1 tsp Old Bay seasoning
\end{itemize}
\end{multicols}
\end{minipage}
\noindent\begin{minipage}[t]{0.18\linewidth}
\centering \strut\vspace*{-\baselineskip}\newline
\includegraphics[width=0.9\linewidth]{/home/tim/Documents/projects/recipes/img/corn_chowder.jpg}\\
\end{minipage}\vspace{3mm}
\textbf{Directions}:
\vspace{-3mm}\begin{enumerate}\setlength\itemsep{-1mm}
\item Add pancetta to pot on medium. Cook 2-3 min, until crispy. Add onion, celery, and garlic; stir. Cook about 4 min, until soft but not browned.
\item Add butter; stir until melted and it begins to bubble. Add flour, stirring until completely blended. Cook, stirring and scraping bottom of pot every 30 seconds, about 3 min.
\item Whisk in chicken stock. Add corn; whisk. Add peppers, potato, and thyme; season to taste with sea salt and freshly ground pepper. Increase heat to medium-high. Cook, stirring occasionally, about 10 min until it comes to a boil.
\item Reduce heat to medium. Simmer 10-15 min until potato is tender. Add cream gradually; stir. Return to simmer, about 2-3 min. Season with Tabasco and Old Bay.
\end{enumerate}
\end{minipage}\vspace{8mm}
\noindent\begin{minipage}[t]{\linewidth}%
{\Large\textbf{Country Vegetable Soup}} \label{country-vegetable-soup}\hfill\textit{Carolyn Benjamin}\\
\noindent\begin{minipage}[t]{0.78\linewidth}%
\textbf{Ingredients}:\vspace{-3mm}
\begin{multicols}{2}
\begin{itemize}\setlength\itemsep{-1mm}
\item 1 tsp olive oil
\item 1 cup chopped onion
\item 3 garlic cloves, minced
\item 1/2 cup carrots, sliced
\item 1/2 cup celery, sliced
\item 2 cups red potatoes, chopped
\item 2 cups acorn squash or cooking pumpkin, cubed
\item 1 tsp dried basil
\item 1/4 tsp cinnamon
\item 1/4 tsp dried thyme
\item 28 oz chopped tomatoes
\item 2 cans vegetable broth
\item 4 cups chopped kale
\item 1 can drained chickpeas
\item 1 can drained cannelini beans
\end{itemize}
\end{multicols}
\end{minipage}
\noindent\begin{minipage}[t]{0.18\linewidth}
\centering \strut\vspace*{-\baselineskip}\newline
\includegraphics[width=0.9\linewidth]{/home/tim/Documents/projects/recipes/img/D438F69A-B08F-4D9F-8163-552F4EE4E321.jpg}\\
\end{minipage}\vspace{3mm}
\textbf{Directions}:
\vspace{-3mm}\begin{enumerate}\setlength\itemsep{-1mm}
\item Heat oil in a Dutch oven over medium-high heat. Add onion and garlic and saute for 3 minutes. Add carrots and the next 6 ingredients (through thyme), stirring to combine. Cook 4 minutes, stirring occasionally. Add tomatoes, cook 2 minutes.
\item Stir in broth and bring to a boil. Reduce heat, simmer 8 minutes. Add kale, simmer 5 minutes. Add beans, simmer 4 minutes or until potato and kale are tender. 
\end{enumerate}
\end{minipage}\vspace{8mm}
\noindent\begin{minipage}[t]{\linewidth}%
{\Large\textbf{Egg Drop Soup}} \label{egg-drop-soup}\hfill\textit{Carolyn Benjamin}\\
\noindent\begin{minipage}[t]{0.78\linewidth}%
\textbf{Ingredients}:\vspace{-3mm}
\begin{multicols}{2}
\begin{itemize}\setlength\itemsep{-1mm}
\item 3 cups chicken stock
\item 1 egg, beaten
\item 1 scallion, chopped
\item 1 Tbsp cornstarch
\item 1 Tbsp water
\end{itemize}
\end{multicols}
\end{minipage}
\noindent\begin{minipage}[t]{0.18\linewidth}
\centering \strut\vspace*{-\baselineskip}\newline
\includegraphics[width=0.9\linewidth]{/home/tim/Documents/projects/recipes/img/FD6D1C2E-051B-4FA3-98AC-2E4FF82BBA09.jpg}\\
\end{minipage}\vspace{3mm}
\textbf{Directions}:
\vspace{-3mm}\begin{enumerate}\setlength\itemsep{-1mm}
\item Put stock on high heat. While boiling, stir in cornstarch until stock thickens and becomes clear.
\item Slowly pour in egg and stir once gently. Turn off heat. Pour in bowls and top with chopped scallions.
\end{enumerate}
\end{minipage}\vspace{8mm}
\noindent\begin{minipage}[t]{\linewidth}%
{\Large\textbf{New England Clam Chowder}} \label{new-england-clam-chowder}\hfill\textit{Carolyn Benjamin}\\
\textbf{Yield:} \textit{8 servings}\\
\noindent\begin{minipage}[t]{0.78\linewidth}%
\textbf{Ingredients}:\vspace{-3mm}
\begin{multicols}{2}
\begin{itemize}\setlength\itemsep{-1mm}
\item 1 lb bacon
\item 2 cups sweet onions, chopped
\item 1 1/2 cups celery, chopped
\item 3/4 cups carrots, chopped
\item 1/3 cups flour
\item 2 large potatoes, diced (small)
\item 10 (6.5 oz) cans of clams, save juice from 8
\item 2 bay leaves
\item 1 Tbsp garlic, minced
\item 1 Tbsp parsley
\item 1/2 tsp dill
\item 4 cups half 'n half, or whole milk
\item salt and pepper
\item Tabasco, to taste
\item 3 Tbsp butter
\item 1 lemon, squeezed (optional)
\end{itemize}
\end{multicols}
\end{minipage}
\noindent\begin{minipage}[t]{0.18\linewidth}
\centering \strut\vspace*{-\baselineskip}\newline
\includegraphics[width=0.9\linewidth]{/home/tim/Documents/projects/recipes/img/clam-chowder.jpeg}\\
\end{minipage}\vspace{3mm}
\textbf{Directions}:
\vspace{-3mm}\begin{enumerate}\setlength\itemsep{-1mm}
\item Cook bacon slowly until dark and crispy. Reserve bacon grease for cooking vegetables.
\item  Chop onions, celery, and carrots.
\item  Saute vegetables in pot with bacon grease. Slowly add flour.
\item Add potatoes and clam juice. Stir until hot.
\item Add clams, bay leaves, garlic, parsley, and dill. Pour in Half ‘n Half. Stir until potatoes are soft.
\item Add salt, pepper, and tabasco sauce to taste. Add butter and lemon juice. Top with bacon.
\end{enumerate}
\end{minipage}\vspace{8mm}
\noindent\begin{minipage}[t]{\linewidth}%
{\Large\textbf{Potato Leek Soup}} \label{potato-leek-soup}\hfill\textit{Sue Dunn}\\
\textbf{Yield:} \textit{serves 8-10}\\
\noindent\begin{minipage}[t]{0.78\linewidth}%
\textbf{Ingredients}:\vspace{-3mm}
\begin{multicols}{2}
\begin{itemize}\setlength\itemsep{-1mm}
\item 1/4 cup unsalted butter
\item 2 lb leeks, white portions only, trimmed, washed, sliced
\item 6 cups chicken/vegetable stock
\item 2 lb baking potatoes, peeled, quartered lengthwise, sliced
\item salt and white pepper
\item 2 Tbsp fresh chives, chopped
\end{itemize}
\end{multicols}
\end{minipage}
\noindent\begin{minipage}[t]{0.18\linewidth}
\centering \strut\vspace*{-\baselineskip}\newline
\includegraphics[width=0.9\linewidth]{/home/tim/Documents/projects/recipes/img/511D727A-8A5C-4025-B07A-B7168B6B53CF.jpg}\\
\end{minipage}\vspace{3mm}
\textbf{Directions}:
\vspace{-3mm}\begin{enumerate}\setlength\itemsep{-1mm}
\item In a large saucepan, melt the butter over medium heat. Add the leeks and saute just until they begin to soften, 3-5 minutes. Add the stock and potatoes, bring to a boil, reduce heat to low, cover, and simmer until the potatoes are very tender, about 20 minutes. 
\item Use handheld blender to puree the soup. Season to taste with salt and white pepper. Ladle into warmed bowls and garnish with the chives.
\end{enumerate}
\end{minipage}\vspace{8mm}
\noindent\begin{minipage}[t]{\linewidth}%
{\Large\textbf{Roasted Cauliflower Soup}} \label{roasted-cauliflower-soup}\hfill\textit{Sue Dunn}\\
\textbf{Yield:} \textit{8 bowls}\\
\noindent\begin{minipage}[t]{0.78\linewidth}%
\textbf{Ingredients}:\vspace{-3mm}
\begin{multicols}{2}
\begin{itemize}\setlength\itemsep{-1mm}
\item 2 lb cauliflower florets
\item 3 Tbsp olive oil
\item 1 tsp salt
\item 1/4 tsp cayenne pepper
\item 1/2 cup onion, diced
\item 2 cloves garlic, minced
\item 3-4 cups chicken broth
\item 1/2 cup heavy cream
\item 4 oz (1/2 pkg) cream cheese
\item 2 cups cheddar cheese, grated
\end{itemize}
\end{multicols}
\end{minipage}
\noindent\begin{minipage}[t]{0.18\linewidth}
\centering \strut\vspace*{-\baselineskip}\newline
\includegraphics[width=0.9\linewidth]{/home/tim/Documents/projects/recipes/img/none.jpg}\\
\end{minipage}\vspace{3mm}
\textbf{Directions}:
\vspace{-3mm}\begin{enumerate}\setlength\itemsep{-1mm}
\item Preheat oven to 400F. Place cauliflower on a baking sheet and drizzle with 2 Tbsp olive oil. Sprinkle with salt and cayenne pepper and stir to coat. Roast cauliflower for 20 minutes, stirring once halfway through cooking.
\item When cauliflower has roasted, add the remaining tablespoon of olive oil to a large soup pot on medium heat. Add the onion and garlic; cook for 5 minutes until onions are soft. Add the chicken broth to the pot along with the roasted cauliflower and bring to a boil. Reduce to a simmer and cook for 15 minutes.
\item Use a handheld blender to mash the cauliflower. Stir in the heavy cream, cream cheese, and grated cheddar until smooth and creamy.
\item Bring to a boil. Mix together the cornstarch and water before adding to the pan, stirring quickly, until sauce has thickened slightly. Reduce heat and simmer gently for another minute to thicken sauce. Return chicken to the skillet, and sprinkle with extra herbs if desired. Serve immediately.
\end{enumerate}
\end{minipage}\vspace{8mm}
\noindent\begin{minipage}[t]{\linewidth}%
{\Large\textbf{Spicy White Turkey Chili Soup}} \label{spicy-white-turkey-chili-soup}\hfill\textit{Sue Dunn}\\
\textbf{Yield:} \textit{15 cups}\\
\noindent\begin{minipage}[t]{0.78\linewidth}%
\textbf{Ingredients}:\vspace{-3mm}
\begin{multicols}{2}
\begin{itemize}\setlength\itemsep{-1mm}
\item 2 Tbsp butter
\item 3 Tbsp olive oil
\item 1 1/2 cups diced celery and onions
\item 3/4 cup all-purpose flour
\item 3 Tbsp chili powder
\item 1 tsp cumin
\item 1/8 tsp cayenne pepper
\item 2 (32 oz) cartons chicken stock
\item 1 red bell pepper, diced
\item 2 (15.5 oz) cans cannellini beans
\item 1 green bell pepper, diced
\item 1 cup (6 oz) cooked turkey, diced
\item 1/4 cup heavy cream
\item salt and pepper to taste
\end{itemize}
\end{multicols}
\end{minipage}
\noindent\begin{minipage}[t]{0.18\linewidth}
\centering \strut\vspace*{-\baselineskip}\newline
\includegraphics[width=0.9\linewidth]{/home/tim/Documents/projects/recipes/img/turkey-chili.jpeg}\\
\end{minipage}\vspace{3mm}
\textbf{Directions}:
\vspace{-3mm}\begin{enumerate}\setlength\itemsep{-1mm}
\item Melt butter and olive oil in large stockpot on medium, until oil/butter mixture faintly smokes. Add celery and onions. Cool, stirring occasionally for 4-5 minutes, until translucent but not brown. Stir in flour, chili powder, cumin and cayenne. Cook 1 more minute, stirring continuously.
\item Add stock. Bring to a simmer; simmer 10 minutes. Add cannellini beans, peppers, turkey and heavy cream. Season to taste with salt and pepper. Simmer 2-3 minutes until heated through. Ladle into warmed bowls.
\end{enumerate}
\end{minipage}\vspace{8mm}
\noindent\begin{minipage}[t]{\linewidth}%
{\Large\textbf{Sweet Potato Soup}} \label{sweet-potato-soup}\hfill\textit{Sue Dunn}\\
\textbf{Yield:} \textit{8 servings}\\
\noindent\begin{minipage}[t]{0.78\linewidth}%
\textbf{Ingredients}:\vspace{-3mm}
\begin{multicols}{2}
\begin{itemize}\setlength\itemsep{-1mm}
\item 2 Tbsp flour
\item 2 Tbsp unsalted butter
\item 3 cups chicken/vegetable broth
\item 2 Tbsp light brown sugar
\item 3 cups cooked sweet potatoes
\item 1/2 tsp ground ginger
\item 1/4 tsp ground cinnamon
\item 1/4 tsp ground nutmeg
\item 2 cups milk
\item salt
\end{itemize}
\end{multicols}
\end{minipage}
\noindent\begin{minipage}[t]{0.18\linewidth}
\centering \strut\vspace*{-\baselineskip}\newline
\includegraphics[width=0.9\linewidth]{/home/tim/Documents/projects/recipes/img/sweet-potato-soup.jpeg}\\
\end{minipage}\vspace{3mm}
\textbf{Directions}:
\vspace{-3mm}\begin{enumerate}\setlength\itemsep{-1mm}
\item In a saucepot, over medium-low heat, cook the flour and butter, stirring constantly until a light caramel color is achieved. Add the broth and brown sugar, bring to a boil, then lower to a simmer. Stir in sweet potatoes and spices, bring to a simmer again, and cook for 5 more minutes. In a blender, puree the soup in batches and return to saucepot. Add the milk, and reheat soup. Season with salt and pepper.
\end{enumerate}
\end{minipage}\vspace{8mm}
\noindent\begin{minipage}[t]{\linewidth}%
{\Large\textbf{Shakshuka}} \label{shakshuka}\hfill\textit{Cameron Behar}\\
\noindent\begin{minipage}[t]{0.78\linewidth}%
\textbf{Ingredients}:\vspace{-3mm}
\begin{multicols}{2}
\begin{itemize}\setlength\itemsep{-1mm}
\item 3 Tbsp olive oil
\item 1/2 red bell pepper
\item 1/2 yellow onion
\item 1/2 tsp ground cumin
\item 1/3 tsp chili powder
\item 1/2 tsp smoked paprika
\item 1/2 tsp dried oregano
\item 1/4 tsp ground pepper
\item 1/2 tsp crushed red pepper flakes
\item 1/2 tsp salt
\item 3 cloves garlic, minced
\item 1 (14 oz) can crushed tomatoes
\item 4 eggs
\item 1/4 cup crumbled feta cheese
\item 1/4 cup chopped cilantro
\item toast or naan bread
\end{itemize}
\end{multicols}
\end{minipage}
\noindent\begin{minipage}[t]{0.18\linewidth}
\centering \strut\vspace*{-\baselineskip}\newline
\includegraphics[width=0.9\linewidth]{/home/tim/Documents/projects/recipes/img/shakshuka.jpg}\\
\end{minipage}\vspace{3mm}
\textbf{Directions}:
\vspace{-3mm}\begin{enumerate}\setlength\itemsep{-1mm}
\item Preheat oven to 375F. In a cast iron skillet over medium heat, add olive oil, chopped onions, and red bell pepper. Saute until soft, then add the spices. Lastly, add the minced garlic and saute until fragrant. Add the crushed tomatoes and simmer for 10 minutes.
\item Create 4 small wells in the sauce and crack the eggs into the wells. Place the skillet in the oven and bake for 8-10 minutes until the eggs are cooked.
\item Remove from oven and top with crumbled feta and cilantro. Serve with toasted bread.
\end{enumerate}
\end{minipage}\vspace{8mm}

{\newpage \LARGE \textbf{Steak}} \label{steak}\vspace{4mm}\\
\noindent\begin{minipage}[t]{\linewidth}%
{\Large\textbf{Beef Tenderloin}} \label{beef-tenderloin}\hfill\textit{Cheryl Barker}\\
\noindent\begin{minipage}[t]{0.78\linewidth}%
\textbf{Ingredients}:\vspace{-3mm}
\begin{multicols}{2}
\begin{itemize}\setlength\itemsep{-1mm}
\item 5 lbs beef tenderloin
\item 1 cup soy sauce
\item 1/2 cup sherry
\item 1/2 cup olive oil
\item 3 cloves garlic, minced
\end{itemize}
\end{multicols}
\end{minipage}
\noindent\begin{minipage}[t]{0.18\linewidth}
\centering \strut\vspace*{-\baselineskip}\newline
\includegraphics[width=0.9\linewidth]{/home/tim/Documents/projects/recipes/img/beef-tenderloin.jpeg}\\
\end{minipage}\vspace{3mm}
\textbf{Directions}:
\vspace{-3mm}\begin{enumerate}\setlength\itemsep{-1mm}
\item Combine ingredients into a marinade, and then marinate meat for up to 24 hours. Reserve the liquid.
\item Bake the tenderloin at 275F for 1 hour and 15 minutes, or internal temperature reaches 120F. Wrap tightly in doubled foil with marinade poured over for 2 hours. Slice thinly and serve.
\end{enumerate}
\end{minipage}\vspace{8mm}
\noindent\begin{minipage}[t]{\linewidth}%
{\Large\textbf{Beef Wellington}} \label{beef-wellington}\hfill\textit{Jeff Benjamin}\\
\textbf{Yield:} \textit{6 servings}\\
\noindent\begin{minipage}[t]{0.78\linewidth}%
\textbf{Ingredients}:\vspace{-3mm}
\begin{multicols}{2}
\begin{itemize}\setlength\itemsep{-1mm}
\item 2 lbs beef tenderloin
\item salt
\item black pepper, freshly ground
\item olive oil, for greasing
\item 2 Tbsp dijon mustard
\item 1 1/2 lb mushrooms, chopped
\item 1 shallot, chopped
\item leaves from 1 thyme sprig
\item 2 Tbsp unsalted butter
\item 12 thin slices prosciutto
\item flour, for dusting
\item 14 oz frozen puff pastry, thawed
\item 1 egg, beaten
\item flaky salt, for sprinkling
\end{itemize}
\end{multicols}
\end{minipage}
\noindent\begin{minipage}[t]{0.18\linewidth}
\centering \strut\vspace*{-\baselineskip}\newline
\includegraphics[width=0.9\linewidth]{/home/tim/Documents/projects/recipes/img/none.jpeg}\\
\end{minipage}\vspace{3mm}
\textbf{Directions}:
\vspace{-3mm}\begin{enumerate}\setlength\itemsep{-1mm}
\item Using kitchen twine, tie tenderloin in 4 places. Season generously with salt and pepper.
\item Over high heat, coat bottom of a heavy skillet with olive oil. Once pan is nearly smoking, sear tenderloin until well-browned on all sides, including the ends, about 2 minutes per side (12 minutes total). Transfer to a plate. When cool enough to handle, snip off twine and coat all sides with mustard. Let cool in fridge.
\item Meanwhile, make duxelles: in a food processor, pulse mushrooms, shallots, and thyme until finely chopped.
\item To skillet, add butter and melt over medium heat. Add mushroom mixutre and cook until liquid has evaporated, about 25 minutes. Season with salt and pepper, then let cool in fridge.
\item Place plastic wrap down on a work surface, overlapping so that it's twice the length and width of the tenderloin. Shingle the prosciutto on the plastic wrap into a rectangle that's big enough to over the whole tenderloin. Spread the duxelles evenly and thinly over the prosciutto.
\item Season tenderloin, then place it at the bottom of the prosciutto. Roll meat into mixture, using plastic warap to roll tightly. Tuck ends of prosciutto as you roll, then twist ends of plastic wrap tightly into a log and transfer to fridge to chill (this helps maintain shape).
\item Heat oven to 425F. Lightly flour your work surface, then spread out puff pastry and roll into a rectangle that will cover the tenderloin (slightly larger than prosciutto rectangle). Remove tenderloin from plastic wrap and place on bottom of puff pastry. Brush the other three edges of the pastry with egg wash, then tightly roll beef into pastry.
\item Once the log is fully covered in puff pastry, trim any extra pastry, then crimp edges with a fork to seal well. Wrap roll in plastic wrap to get a really right cylinder, then chill for 20 minutes.
\item Remove plastic wrap, then transfer roll to a foil-lined baking sheet. Brush with egg wash and sprinkle with flaky salt.
\item Bake until pastry is golden and the center registers 120F for medium-rare, about 40-45 minutes. Let rest for 10 minutes before carving and serving.
\end{enumerate}
\end{minipage}\vspace{8mm}
\noindent\begin{minipage}[t]{\linewidth}%
{\Large\textbf{Steak Teriyaki Marinade}} \label{steak-teriyaki-marinade}\hfill\textit{Aunt Nancy Feth}\\
\noindent\begin{minipage}[t]{0.78\linewidth}%
\textbf{Ingredients}:\vspace{-3mm}
\begin{multicols}{2}
\begin{itemize}\setlength\itemsep{-1mm}
\item 3/4 cup soy sauce
\item 1/4 cup water or wine
\item 1 tsp garlic powder
\item 2 tsp sugar
\item 1/2 tsp ginger
\end{itemize}
\end{multicols}
\end{minipage}
\noindent\begin{minipage}[t]{0.18\linewidth}
\centering \strut\vspace*{-\baselineskip}\newline
\includegraphics[width=0.9\linewidth]{/home/tim/Documents/projects/recipes/img/041AD6E5-03E7-484A-B58E-E8678B694B63.jpg}\\
\end{minipage}\vspace{3mm}
\textbf{Directions}:
\vspace{-3mm}\begin{enumerate}\setlength\itemsep{-1mm}
\item Mix all together and pour over meat to be marinated. Double recipe to desired quantity to cover meat.
\end{enumerate}
\end{minipage}\vspace{8mm}

{\newpage \LARGE \textbf{Quiches}} \label{quiches}\vspace{4mm}\\
\noindent\begin{minipage}[t]{\linewidth}%
{\Large\textbf{Quiche Lorraine}} \label{quiche-lorraine}\hfill\textit{Sue Dunn}\\
\textbf{Yield:} \textit{two quiches}\\
\noindent\begin{minipage}[t]{0.78\linewidth}%
\textbf{Ingredients}:\vspace{-3mm}
\begin{multicols}{2}
\begin{itemize}\setlength\itemsep{-1mm}
\item 2 frozen deep dish pie crusts
\item 1 lb bacon (12 slices)
\item 2 onions
\item 6 eggs, beaten
\item 3 cups milk
\item 1/2 tsp salt
\item 3 cups Swiss cheese, shredded
\item 2 Tbsp flour
\item 1/4 tsp ground nutmeg
\end{itemize}
\end{multicols}
\end{minipage}
\noindent\begin{minipage}[t]{0.18\linewidth}
\centering \strut\vspace*{-\baselineskip}\newline
\includegraphics[width=0.9\linewidth]{/home/tim/Documents/projects/recipes/img/quiche_lorraine.jpg}\\
\end{minipage}\vspace{3mm}
\textbf{Directions}:
\vspace{-3mm}\begin{enumerate}\setlength\itemsep{-1mm}
\item Preheat oven to 325F. In a large skillet, cook bacon until crisp. Drain and reserve 2 Tbsp of drippings. Crumble the bacon and set aside.
\item Cook onion in skillet with reserved dripipings; cook until onions are tender and then drain.
\item In a large bowl, mix together milk, salt, and eggs. Stir in bacon and onion.
\item In a separate bowl, toss cheese and flour together, then add to egg mixture. Be sure to mix well. Pour egg mixture into pie crust.
\item Bake in preheated oven for 35-40 minutes.
\end{enumerate}
\end{minipage}\vspace{8mm}
\noindent\begin{minipage}[t]{\linewidth}%
{\Large\textbf{Spinach Quiche}} \label{spinach-quiche}\hfill\textit{Sue Dunn}\\
\textbf{Yield:} \textit{2 quiches}\\
\noindent\begin{minipage}[t]{0.78\linewidth}%
\textbf{Ingredients}:\vspace{-3mm}
\begin{multicols}{2}
\begin{itemize}\setlength\itemsep{-1mm}
\item 2 frozen deep dish pie crusts
\item 2 (12 oz) boxes frozen chopped spinach
\item 1 ham steak, diced
\item 1 chopped vidalia (white) onion
\item 2 cups cheddar cheese, shredded
\item 9 eggs
\item 1 cup milk
\item 1 tsp salt
\item ground pepper
\end{itemize}
\end{multicols}
\end{minipage}
\noindent\begin{minipage}[t]{0.18\linewidth}
\centering \strut\vspace*{-\baselineskip}\newline
\includegraphics[width=0.9\linewidth]{/home/tim/Documents/projects/recipes/img/spinach_quiche.jpg}\\
\end{minipage}\vspace{3mm}
\textbf{Directions}:
\vspace{-3mm}\begin{enumerate}\setlength\itemsep{-1mm}
\item Preheat oven to 375F.
\item Thaw spinach in microwave. When thawed, totally squeeze out all moisture.
\item In a medium bowl, whisk together the eggs, milk, salt, and pepper. Thoroughly stir in the spinach, cheese, onion, and ham. Pour half into each pie crust, and bake for 30-40 minutes.
\end{enumerate}
\end{minipage}\vspace{8mm}
\noindent\begin{minipage}[t]{\linewidth}%
{\Large\textbf{Vegetable Gruyere Quiche}} \label{vegetable-gruyere-quiche}\hfill\textit{Sue Dunn}\\
\textit{``also known as `stinky cheese quiche'''}\\
\noindent\begin{minipage}[t]{0.78\linewidth}%
\textbf{Ingredients}:\vspace{-3mm}
\begin{multicols}{2}
\begin{itemize}\setlength\itemsep{-1mm}
\item 2 cups leek
\item 4 cups zucchini
\item 4 cups yellow squash
\item 2 cups red bell pepper
\item 2 frozen deep-dish pie crusts
\item 4 eggs
\item 2 egg white
\item 4 Tbsp milk
\item 2 tsp Italian seasoning
\item 1 tsp salt
\item 1/2 tsp black pepper
\item 1/2 cup shredded Gruyere cheese
\end{itemize}
\end{multicols}
\end{minipage}
\noindent\begin{minipage}[t]{0.18\linewidth}
\centering \strut\vspace*{-\baselineskip}\newline
\includegraphics[width=0.9\linewidth]{/home/tim/Documents/projects/recipes/img/none.jpg}\\
\end{minipage}\vspace{3mm}
\textbf{Directions}:
\vspace{-3mm}\begin{enumerate}\setlength\itemsep{-1mm}
\item In a large pan, bring 2 inches of water to a boil. Add leeks; cover and simmer for 5 minutes.
\item Add zucchini, squash, and bell pepper. Simmer another 5 minutes. Drain well and pat vegetables dry.
\item Preheat oven to 375F. Spoon vegetables into pie shells.
\item Combine eggs, egg whites, milk, and spices; stir well until mixed.
\item Pour into pie shell. Bake at 375F for 30 minutes.
\item Sprinkle with cheese. Bake for an additional 5 minutes or until cheese is melted.
\end{enumerate}
\end{minipage}\vspace{8mm}

\end{document}
